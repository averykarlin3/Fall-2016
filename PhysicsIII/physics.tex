\documentclass[11 pt, twoside]{article}
\usepackage{textcomp}
\usepackage[margin=1in]{geometry}
\usepackage[utf8]{inputenc}
\usepackage{color}
\usepackage{indentfirst} %Comment out for no first paragraph indent
\usepackage[parfill]{parskip}
\usepackage{setspace}
\usepackage{tikz}
\usepackage{amsmath}
\usepackage{amsfonts}
\usepackage{amssymb}
\usepackage{enumitem}
\usepackage{outlines}

\usepackage{fancyhdr}
\pagestyle{fancy}
\cfoot{\hyperlink{content}{\thepage}}
\lhead{}
\chead{}
\rfoot{}
\lfoot{}
\rhead{}
\renewcommand{\headrulewidth}{0pt}
\renewcommand{\footrulewidth}{0pt}


\usepackage{hyperref}
\hypersetup {
	colorlinks,
	citecolor=black,
	filecolor=black,
	linkcolor=black,
	urlcolor=black
}

\newcommand{\sepitem}{0pt} %Added room between items on the list, not including a list and its sublist
\newcommand{\seppar}{1pt} %Between items and lists overall

\setenumerate[1]{itemsep=\sepitem, parsep=\seppar}
\setenumerate[2]{itemsep=\sepitem, parsep=\seppar}
\setenumerate[3]{itemsep=\sepitem, parsep=\seppar}
\setenumerate[4]{itemsep=\sepitem, parsep=\seppar}

\newenvironment{outline*}
{
	\begin{outline}[enumerate]
	}
	{\end{outline}
}

\newcommand{\foot}[1]{\hyperlink{#1}{$_#1$}}

\begin{document}

\title{Introduction to Physics III: Thermodynamics, Waves, and Relativity}
\author{Avery Karlin}
\date{Fall 2016}
\newcommand{\textbook}{Heat and Thermodynamics by Zemansky and Dittman, 7th Edition}
\newcommand{\teacher}{Dr. Johnson}

\maketitle
\newpage
\hypertarget{content}{\tableofcontents}
\vspace{11pt}
\noindent
\underline{Primary Textbook}: \textbook\\
\underline{Secondary Textbook}: Introduction to the Physics of Waves by Freegarde \\
\underline{Secondary Textbook}: Special Relativity by Helliwell \\
\underline{Teacher}: \teacher
\newpage

\section{Chapter 1 - Temperature and the Zeroth Law of Thermodynamics}
\subsection{Definition of Thermodynamics}
\begin{outline*}
\1 The study of a natural system requires the creation of a boundary, separating a section of space and matter, or the system, from its surroundings, called closed if no matter is able to cross the boundary, open if there is an exchange of matter
\1 Systems are then studied on either a macroscopic/human scale or a microscopic/molecular scale
\2 Systems are described on a macroscopic scale by composition, mass, volume, and temperature, and other aggregate properties, acting as macroscopic coordinates for the macroscopic description
\3 Macroscopic coordinates have the properties of assuming nothing special about the matter structure, fields, or radiation, require few to describe a system, suggested by sensory observation, and can be directly measured
\3 While mechanics describes the external properties of a system and its energy, thermodynamics describes the internal properties
\2 Systems are described on a microscopic scale by statistical mechanics, describing the populations/number of particles in each energy state at equilibrium, and the interactions of particles with each other by collisions and fields, and with other systems in an ensemble
\3 The equilibrium state is the state of the highest probability, and the system is assumed to have some high number of particles
\3 Microscopic descriptions have the properties of making assumptions about the structure of matter, fields, and radiation, requires many quantities, is based mathematical models rather than observation, and must be calculated rather than measured
\2 The scales must reconcile to the same conclusion, with the macroscopic view as the average over some amount of time of the microscopic characteristics
\3 Since the microscopic view requires assumptions and models rather than observations, it is able to change as the result of increased data, unlike the macroscopic view which is used to test the assumptions
\1 Thermodynamics is the study of the macroscopic properties of nature, especially the temperature of the system, and the identification of thermodynamic relations based on the fundamental laws
\2 Mechanical coordinates used within classical mechanics are from the mechanical/external energy of the system to measure the movement of the system overall, while thermodynamics uses macroscopic coordinates to deal with the internal energy of the system, called thermodynamic coordinates
\2 Systems described by thermodynamic laws are thermodynamic systems, found within each discipline
\1 Thermodynamics is the branch of physics which describes the relationships between all forms of energy, conserved in an isolated system by the Law of Conservation of Energy
\2 Heat is energy in transit from one body at higher temperature to another with lower temperature, with 1 calorie as the heat to raise the temperature of 1 g of water by $1^oC$ at 1 atm, equal to 4.184 J
\end{outline*}
\subsection{Thermal Equilibrium and Temperature}
\begin{outline*}
\1 For some thermodynamic system of constant mass and composition, where to describe the system requires two coordinates, X and Y
\2 The system is at equilibrium state if the coordinates would remain constant if the external conditions are unchanged
\2 Equilibrium states depend on the proximity to other systems and the boundary within systems (adiabatic if any equilibrium state is present on both, such as wood, while diathermic if there must be specific combinations of equilibrium states, such as metal)
\3 Thermal equilibrium is achieved by multiple systems separated by a diathermic wall when they have reached a combined system equilibrium
\3 Diathermic walls are thus boundaries that allow heat transfers, while adiabatic do not allow heat transfers, both not permitting matter transfers (assuming it does not break due to stress from the systems)
\2 The Zeroth Law of Thermodynamics states that if two systems are each in thermal equilibrium with another system simultaneously, then they are at thermal equilibrium with each other as well
\3 It is called this due to the next two relying on this definition, found experimentally as a fundamental law of the discipline
\1 Temperature is designed as the macroscopic measure of hotness from an object observed, and microscopically as the average kinetic energy of particles, defined as some number $\geq 0$ in science (Kelvin), ignoring the terms of coldness
\2 For any given state of a system, there exists a curve, or isotherm, of the locus of all states of another system which would be in thermal equilibrium with the first system, experimentally found to generally be continuous over some region of the curve
\3 The same can be found for the other system at each set of coordinates, such that by the zeroth law, each isotherm has a corresponding isotherm in the other system, such that all points on those lines are at equilibrium with each other
\3 The property which determines if a system is at thermal equilibrium is called temperature, since all that is needed is a correspondence of equilibrium between two systems and a reference, such that it is a scalar, rather than a vector if there was not a transitive property
\2 Thus, for systems at the same temperature, they are on corresponding isotherms and thus at thermal equilibrium, such that temperature is a measure of thermal equilibrium
\3 The converse of this is also true, such that thermal equilibrium implies same temperature
\1 Temperature equality can be determined by a capillary of mercury, equally filled and sized, measuring the equality of the height the mercury rises, experimentally found to signify equilibrium, called a thermoscope
\2 Thermometers measure the temperature on an empirical scale, using some rule to assign a value to each isotherm, finding some path on the isotherm plane, with the varying variable as the thermometric property, the thermometric function, $\theta(X)$ to get the temperature
\3 The traditional mercury volume thermometer relies on the asymetrical oscillation as energy increases, such that it is more likely to be at the larger side of oscillation, creating an average larger volume
\4 This is defined by $U(r) = 4\epsilon((\frac{\omega}{r})^{12} - (\frac{\omega}{r})^6)$, such that $\epsilon$ is the minimum energy, r is the distance between molecules, and $\omega^{\frac{2}{3}} = r_0$, where $r_0$ is the minimum distance between molecules
\4 All objects expand outward as a result, such that there would be increased space in a cavity as well, expanding only outward
\2 The historic scale mercury pressure thermometer (constant volume) is based on a linear relationship, $\theta(X) = aX + c$, without a constant defining an absolute temperature scale (absolute zero at $0^o)$
\3 The same relation when applied to other thermometers or other thermometer systems of the same type (the international standard being hydrogen pressure) produces other scales, with some defined coefficient for each
\2 Thus, when a thermometer is placed in contact with a chosen standard system in a reproducible state, called the fixed point temperature
\3 Until 1954, the world standard was Celsius, but which was too difficult to measure accurately due to water surrounding ice and sensitivity to minor pressure fluctuations
\3 After, the Kelvin system is based on the triple point of water (three state equilibrium, called the standard fixed point), able to measured accurately, as an absolute system, assigned 273.16 K, equal to $0.01^o C$
\4 This is measured by putting water under such pressure in a sealed tube that it begins to boil, using a freezing mixture to form a layer of ice on the edge, after which the mixture is removed and water forms between the ice and the edge, still evaporating, at the triple point
\end{outline*}
\subsection{Thermometer Types}
\begin{outline*}
\1 For each type of thermometer, a test using the same equation would produce drastically different results from those of the standard at all values except the reference
\2 Hydrogen gas pressure thermometers tend to be negligibly different at standard pressures, acting as the reference for normal temperature and pressure
\1 Gas thermometers are made up of a measuring gas containing bulb, attached to a mercury column through a capillary, keeping the volume of the gas constant by adjusting the height of the mercury column
\2 The mercury column has two columns, one attached to the capillary, the other to the dead space, with a column of the measuring gas reaching from underneath the columns to the reservoir
\3 If high enough pressure, the mercury can rise into the dead space/nuisance volume at the top of the capillary, adjusting the mercury in the column to the point that the mercury fills the capillary column fully to keep volume constant
\2 Thus, at this point, the pressure of the main column is the atmospheric pressure added to the difference in heights between the columns
\3 It can then be measured when the bulb is surrounded by the triple point water and when surrounded by the measured system
\2 The pressures must be corrected for errors due to the gas being a different temperature than the bulb, the capillary not being uniform temperature, volume changes in any part as the temperature and pressure change
\3 It can also have error due to gradient in the capillary due to the diameter being close to the mean free path of the gas, gas adsorbed by the device, especially at low temperatures, and temperature and compressibility of the mercury creating error
\2 As a result, once corrected, the measuring gas behavior appears close to that of an ideal gas
\3 The basis of gas thermometers was the ideal-gas law, stating PV = nRT, where n is the number of moles of gas, T is the Kelvin temperature, and R is the molar gas constant
\3 This is able to be used for the reference, to derive the same relation between triple point pressure/temperature and measured pressure/temperature ($\frac{P}{P_TP} = \frac{\theta}{273.16 K}$), such that the temperature definition thus assumes an ideal gas for a Kelvin thermometer
\2 Helium is the best measuring gas, not diffusing through platinum at high temperatures, but not becoming a liquid until extremely low temperatures
\1 Vapor pressure thermometry depends only on the vapor pressure of the measuring gas at some temperature, thus easy to produce, and is easy to measure
\1 Platinum resistance thermometry is secondary, due to being fundamentally defined in terms of temperature, but with a broader range than gas thermometers
\2 This is done by winding a long, thin wire around a frame, loose enough to avoid strain when compressing, able to also be wrapped around the object being measured, drilled into the object, or bonded onto the surface
\2 Potentiometric resistance thermometers have zero direct current into the leads, while bridge circuits are balanced such that there is no alternating current through
\3 Bridge circuits are made possible as thermometers due to inductive voltage-dividers/ratio transformers and high sensitivity/signal-to-noise lock-in amplifiers
\2 Within the ideal measurement range, $R'(T) = R'_{TP}(1 + aT + bT^2)$, where a and b are constants, generally measured as $W(T) = 1 + aT + bT^2$, measuring the ratio of resistivity, which is less prone to material tension/composition error
\1 Radiation thermometry, including optical, radiation, infrared, and total radiation, are used to measure thermal/blackbody radiation, using Planck's radiation law relating temperature to radiance, depending only on the walls of a closed body, not the shape
\2 The dimensions must be larger than the wavelengths themselves, and there must be a small hole to allow radiation to escape to allow measurement, negligible enough to not disturb equilibrium
\2 Pyrometers were made to measure high temperatures without contact, comparing visible radiation over a small band with that of a standard, using a photoelectric detector, correcting for emissivity of the source at that band
\2 Total-radiation pyrometers have a larger range able to measure, but are less precise, measuring all radiation from the object, rather than just blackbody
\1 Thermocouples are an outdated method of measurement due to high inaccuracy of 0.2 K, made of two wires connected to copper wires in a melting point reference junction, connecting the two wires and measuring the EMF at various values
\2 $\epsilon = c_0 + c_1\theta + c_2\theta^2 + c_3\theta^3$, where $c_n$ are constants based on the materials of the thermocouple, though for a small range of temperature, $c_3$ approximates to 0
\2 The temperature range of the thermocouple depends on the materials used, advantageous because of the small, conductive mass allowing rapid temperature equilibrium with the material being measured
\2 The copper wires can be connected to electrical circuits for monitoring of electrical equipment temperatures or to a simple voltmeter for measurement
\end{outline*}
\subsection{Temperature Scales}
\begin{outline*}
\1 As the triple point pressure of a gas (due to lessening volume), approaches 0, the resulting temperature calculated is the ideal-gas temperature
\2 While thermometers depend on the specific measuring gas properties to provide the temperature, as it approaches the ideal gas temperature, it becomes independent of the measuring gas, though based on the properties of gases overall, rather than an individual gas, behaving ideally
\2 For the temperature region in which a gas thermometer may be used, the measured Kelvin temperature scale and ideal gas temperature scales are identical
\2 It is noted that the idea of lack of atomic motion at absolute zero cannot be assumed, due to relying on the complete equivalency of temperature and atomic motion on different scales 
\3 In addition, at absolute zero, there is some residual energy expected based on quantum mechanics, resulting in the zero-point energy quantity
\1 The Celsius scale was used prior to the Kelvin temperature scale until 1954, which slightly shifted the base from the ice point to the triple point of water (273.16 K) for accuracy and modifying to an absolute scale, with the same degree of magnitude between values
\2 Thus, the Celsius and Kelvin scales simply have a constant difference of the measured ice point, equal to 273.15 K
\2 As a result, the modern Celsius scale no longer has a fixed ice/steam point for water, but rather purely the triple point, such that Kelvin is the standard
\1 Fahrenheit and Rankine are based on a scale 5/9ths that of the Celsius and Kelvin, based on the triple point of water, such that the ice point is found to be $32^o F$, providing a relationship to Celsius
\1 The International Temperature Scale of 1990 was the creation of a practical scale for routine measurements and calibration of instruments, to ease measurement from time-consuming calibration by the single reference point,  providing a series of fixed points for comparison
\2 It also provides close measurement interpolation for the remainder, providing a close approximation to the Kelvin scale
\2 At the minimum of 0.65 K to 3.2 K, it is defined by the $^3He$ vapor pressure-temperature relation, from 1.25 K to 2.1768 K to 5 K by $^4He$ of the same type
\3 $^3He$ fails below 0.3 K due to being too small to measure, and fails above 3.32 K due to the critical point (where only gas exists), while $^4He$ fails below 1 K due to superfluid behavior causing less variation with temperature and errors, and above 5.2 K due to the critical point
\2 From 3 K to 24.5561 K by either He gases as a constant-volume gas thermometer
\2 From 13.8033 K to 1234.93 K, it is defined by platinum resistance thermometers with fixed points within and deviation functions between, creating 11 sub-ranges
\2 Above that, it is measured by an optical pyrometer, using blackbody spectral radiance concentration ratios by Planck's radiation law, using only a single reference of the freezing point of gold, silver, or copper, replacing the thermocouple
\end{outline*}
\section{Chapter 2 - Simple Thermodynamic Systems}
\subsection{Thermodynamic Equilibrium}
\begin{outline*}
\1 Change of state of a system is when the coordinates change either spontaneously or due to outside influence, isolated if the former, such that there is no outside influence, rarely found practically
\2 Systems not in mechanical equilibrium (with no unbalanced force or torque in the system interior or the surroundings/no unequal pressure) will have a change of state until it is corrected
\2 Systems not in chemical equilibrium (unequal chemical potential) result in a transfer of matter through or out/in the system, or a spontaneous change of internal structure as a reaction to restore it
\2 Systems in thermal equilibrium are those with no change in the system's thermodynamic coordinates when separated diathermically, such that there is no heat transfer (meaning temperature is equal with surroundings and within the system)
\3 Otherwise, change of state will occur until thermal equilibrium is restored
\1 Thermodynamic equilibrium is when thermal, mechanical, and chemical equilibrium are found simultaneously, such that it can be described by thermodynamic macroscopic coordinates at a particular instant, though not over time
\2 Thus, when not at mechanical, chemical, and thermal equilibrium, coordinates like composition, temperature, and pressure cannot be used to apply to the system as a whole, as it moves through nonequilibrium states
\1 Equations of state relate the coordinates of a thermodynamic equilibrium system to each other, based on the individual specifics of a system, such that it is determined by molecular theory or experiment, not general theory
\2 At low pressures, the state equation of ideal gas is PV = nRT, or Pv = RT, where v is the molar volume ($\frac{V}{n}$)
\2 At high pressures, it is represented by the van der Waals equation, based on particle interactions and sizes, such that $(P + \frac{a}{v^2})(v - b) = RT$, where v is the molar volume
\3 a and b are gas-specific constants, since equations of state are defined by existence, rather than any particular form without constants
\3 a is the molecular attraction while b is the atomic short range repulsion of the system, such that it accounts for higher pressures in which gases behave less ideally 
\2 Nonequilibrium states cannot be described purely by coordinates as a whole, such that there is no equation of state for the system
\end{outline*}
\subsection{Simple Systems}
\begin{outline*}
\1 Simple systems are those which are described by three thermodynamic coordinates in an equation of state, with the special case of PVT systems called hydrostatic systems
\1 Hydrostatic systems are also defined as an isotopic (uniform) system of constant mass and composition that exerts uniform hydrostatic pressure on the surroundings if there is no gravitational or electromagnetic effects
\2 They are divided into three catagories, pure substances (a single compound of any mixture of states), homogeneous mixture of different compounds (mixture of different compounds of the same state or a solution), and heterogeneous mixtures (mixtures of multiple homogeneous multiple compound mixtures of different states)
\3 It has been found experimentally that systems of a single state can be described by three coordinate equations of state, PVT
\2 These equations can describe some macroscopically small change in a measurement by a differential, though since they are macroscopic coordinates, it describes an amount greater than a few particles changing, rather than an infinitesimal amount
\2 Thus, for some function V = V(T, P), $dV = \frac{\partial V}{\partial T}(T, P)dT + \frac{\partial V}{\partial P}(T, P)dP$
\3 The average coefficient of volume expansion is defined as the change in volume per unit volume divided by the change in temperature at constant pressure
\4 As the change in temperature, and thus volume, approaches 0, the differential coefficient of volume expansion, or the volume expansivity, $\beta = \frac{1}{V}(\frac{\partial V}{\partial T})$
\4 It is generally a positive number, though there are exceptions such as water for $0-4^o C$ melting point at which point water reaches maximum density
\4 This is due to hydrogen bonding of ice and surface tension of water and is generally a function of P and T
\4 It often varies negligibly with respect to P and T often within a small temperature and pressure range, such that it can be considered a constant over the range
\3 The average bulk modulus is defined as $\frac{-V\Delta P}{\Delta V}$, due to the pressure increasing causing a decrease in volume, such that it is a positive value
\4 As the change in volume, and thus pressure, approaches 0 as temperature is constant, it is called the isothermal bulk modulus, $B = -V(\frac{\partial P}{\partial V})$
\4 Isothermal compressibility, or the reciprocal of isothermal bulk modulus, $\kappa = \frac{-1}{V}(\frac{\partial V}{\partial P})$, and is similarly often constant for a material over some PT range
\2 This can be similarly found for each equation of state of a different explicit variable, all of which are differentiated to determine an explicit exact differential equation, describing a slight change in state of equilibrium for that explicit differential in terms of the other variables
\3 By combining two of exact differential equations equations, it follows that if y(x, z) and dz = 0, dx $\neq 0$, then $\frac{\partial x}{\partial y}\frac{\partial y}{\partial x} = 1$
\4 Further, if dz $\neq 0$ and dx = 0, then by extension, $\frac{\partial x}{\partial y}\frac{\partial y}{\partial z}\frac{\partial z}{\partial x} = -1$
\4 By extension of that, $\frac{\partial y}{\partial z}\frac{\partial z}{\partial x} = -\frac{\partial y}{\partial x}$
\3 Thus, by those theorems, $\frac{\partial P}{\partial T} = \frac{\beta}{\kappa}$, such that each partial derivative has a physical representation by the theorems
\4 As a result, a form $dP = \frac{\beta}{\kappa}dT - \frac{1}{\kappa V}dV$ of the state equation is found, which can be integrated for each in terms of the others
\1 Wires are a one-dimensional system, usually assumed that pressure and volume are constant, such that the coordinates are tension, $\tau$, length, L, and temperature, T,
\2 While it cannot be expressed universally by a single equation, for a constant temperature in the portion of elasticity such that Hooke's law is true, $\tau = -k(L - L_0)$, where $L_0$ is the length at no tension
\3 Since equations of state assume thermodynamic equilibrium, the tension must be balanced out by some external force
\2 For some infinitesimal equilibrium state change for $L = L(T, \tau)$, $dL = \frac{\partial L}{\partial T}(T, \tau)dT + \frac{\partial L}{\partial \tau}(T, \tau)d\tau$
\3 By the limit of the average linear coefficient of expansion, linear expansivity, $\alpha = \frac{1}{L}\frac{\partial L}{\partial T}$
\4 Metals tend to have a positive linear expansivity, but other materials such as rubber may be negative
\4 Linear expansivity is mainly constant for torque change, only changing for temperature, but over a small range, can be considered constant
\3 By the limit of average Young's modulus, isothermal Young's modulus, $Y = \frac{L}{A}\frac{\partial \tau}{\partial L}$
\4 Isothermal Young's modulus is always positive, only changing for temperature, but can be considered constant over some small range as well
\1 Surfaces are a 2D system, with different properties from those of the underlying material, acting as a stretched membrane, with each point exerting a force opposite and perpendicular to the opposite side of the surface
\2 Surfaces are thus described by the area, A, the temperature, T, and the surface tension, $\gamma$, which is the force per unit length of the surface, where the unit length is parallel to the edge
\2 While the substance within the surface is relevant, the volume and pressure of it can be ignored, due to remaining fairly constant for small changes in the surface coordinates
\2 For pure liquids at equilibrium with their vapor phase, it is described by the Guggenheim equation of state, $\gamma = \gamma_0(1 - \frac{T}{T_c})^n$, where $T_c$ is critical temperature, $\gamma_0$ is surface tension at standard temperature, generally $20^o C$, and c is a constant, such that $1 \leq c \leq 2$
\3 As a result, as T approaches $T_c$, the surface tension approaches 0, after which there is unable to be any liquid form
\2 For oil films on liquid water, it is described by the equation $(\gamma - \gamma_w)A = aT$, where $\gamma - \gamma_w$ is called the surface pressure, acting as a 2D version of the ideal gas law, and where a is some constant
\end{outline*}
\section{Chapter 3 - Work}
\begin{outline*}
\1 External work is work done on or by the system as a whole, while internal work is work done by part of the system on another part, such as interaction of molecules, only the former able to be done by macroscopic thermodynamics
\2 External work can be viewed as the changing of macroscopic coordinates by the modification of the configuration/position of some external mechanical device or system, such as thermodynamic electrochemical cell systems powering a mechanical device
\2 Work done on a system is denoted positively, while work done by the system is denoted negatively within thermodynamics and mechanics
\1 
\end{outline*}
\section{Chapter 4 - Heat and the First Law of Thermodynamics}
\begin{outline*}

\end{outline*}
\end{document}
