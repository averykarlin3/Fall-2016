\documentclass[11 pt, twoside]{article}
\usepackage{textcomp}
\usepackage[margin=1in]{geometry}
\usepackage[utf8]{inputenc}
\usepackage{color}
\usepackage{indentfirst} %Comment out for no first paragraph indent
\usepackage[parfill]{parskip}
\usepackage{setspace}
\usepackage{tikz}
\usepackage{amsmath}
\usepackage{amsfonts}
\usepackage{amssymb}
\usepackage{enumitem}
\usepackage{outlines}

\usepackage{fancyhdr}
\pagestyle{fancy}
\cfoot{\hyperlink{content}{\thepage}}
\lhead{}
\chead{}
\rfoot{}
\lfoot{}
\rhead{}
\renewcommand{\headrulewidth}{0pt}
\renewcommand{\footrulewidth}{0pt}


\usepackage{hyperref}
\hypersetup {
	colorlinks,
	citecolor=black,
	filecolor=black,
	linkcolor=black,
	urlcolor=black
}

\newcommand{\sepitem}{0pt} %Added room between items on the list, not including a list and its sublist
\newcommand{\seppar}{1pt} %Between items and lists overall

\setenumerate[1]{itemsep=\sepitem, parsep=\seppar}
\setenumerate[2]{itemsep=\sepitem, parsep=\seppar}
\setenumerate[3]{itemsep=\sepitem, parsep=\seppar}
\setenumerate[4]{itemsep=\sepitem, parsep=\seppar}

\newenvironment{outline*}
{
	\begin{outline}[enumerate]
	}
	{\end{outline}
}

\newcommand{\foot}[1]{\hyperlink{#1}{$_#1$}}

\begin{document}

\title{Introduction to Physics III: Thermodynamics, Waves, and Relativity}
\author{Avery Karlin}
\date{Fall 2016}
\newcommand{\textbook}{Heat and Thermodynamics by Zemansky and Dittman, 7th Edition}
\newcommand{\teacher}{Dr. Johnson}

\maketitle
\newpage
\hypertarget{content}{\tableofcontents}
\vspace{11pt}
\noindent
\underline{Primary Textbook}: \textbook\\
\underline{Secondary Textbook}: Introduction to the Physics of Waves by Freegarde \\
\underline{Secondary Textbook}: Special Relativity by Helliwell \\
\underline{Teacher}: \teacher
\newpage

\input{thermo.tex}

\section{Freegarde Chapter 1 - Wave Motion Essence}
\begin{outline*}
\1 Physics can be viewed by Lagrangian Particle Theory or Euler Field Theory, as a duality of perspectives, most apparent on a quantum scale
\2 As a result, when considering a dynamic/time-dependent system, it becomes Kinetic Theory and Wave Theory respectively
\1 Waves are defined microscopically as a collective bulk disturbance, created at a point as a delayed response to the disturbance at adjacent points
\2 The disturbance progression to adjacent points is called the propagation, going through the medium
\2 The medium does not need to be linear, homogeneous, with sinusoidal or periodic propagation to classify as a wave
\2 Positions in the medium are defined by coordinates, each position requiring at least one additional variable to describe the disturbance
\3 The coordinates and disturbance variables are each wavefunctions, defined with respect to time
\1 Waves must be viewed in terms of both cause and effect, such that it is defined macroscopically as a time-dependent field effect due to finite speed of propagation of a causal effect
\2 Thus, it can be viewed as the solution to systems in dynamic equilibrium, more generalized than static equilibrium theory
\2 Electric Coulomb waves can thus be viewed as a result of a rotating dipole acting on a single charge at some distance
\3 If the dipole radius is assumed to be far less than the distance from the charge, it can be simplified such that $E(t) = \frac{2kq}{r_0^3}a(t - \frac{r_0}{c}$
\4 a(t) is the perpendicular height of each side of the dipole, while $r_0$ is the distance from the center of the dipole to the particle
\4 The time delay due to the relativistic limitations on which the change in force is received create the time lag/retardation, allowing the wave
\4 The magnetic component of electromagnetic waves are found by a Lorenz transformation taking into account the motion of the charges and dipoles themselves
\3 Gravitational waves have a similar wave effect for a gravitational single body, such that $g = \frac{Gm}{r_0^3}a(t - \frac{r_0}{c})$
\4 On the other hand, since the rotation of the causal body requires a third body to rotate with it similar to a dipole, it results for similar bodies in a higher order (inverse fourth root law), canceling out the effect to some degree
\4 Thus, it is found that g = $\frac{2Gma_0^2}{r_0^4}sin(2\omega t)$ assuming the distance between the two causal bodies is vastly smaller
\2 As a result, waves can be viewed as some function dependent on an action at a prior time at some other object
\1 It was believed until the 1900s that objects required a medium to influence each other in a vacuum, leading to the idea of the aether
\2 As a result of relativity and electromagnetism, it became apparent that there is no ether, such that forces can be sent through a vacuum
\2 On the other hand, for waves, there remains the concern of diffraction in a vacuum, with apparently nothing deflecting, and the concept of radiation, sending energy through a vacuum alone
\3 Current theories create no issues with these concepts, though they remain difficult to conceptualize
\1 Transverse wave motions, such as surface tension, electromagnetic, or gravitational, where the disturbance is perpendicular to the direction of wave propagation
\2 Longitudinal waves are those where the disturbance is parallel to the direction of propagation, such as sound, generally due to pressure differences on each side
\3 On the other hand, gravitational or electromagnetic waves can be created longitudinal rather than transverse
\2 Spin waves are those with both transverse and longitudinal components, such as seismology
\2 Waves can also be scalar quantity displacement, such as thermal waves (based on heat transfer), quantum wavefunction, or chemical composition waves (based on a reaction-diffusion system)
\2 Waves are also not required to be moving through a continuous medium, as long as they fulfill the definition
\end{outline*}
\section{Freegarde Chapter 2 - Basic Wave Equations}
\subsection{General Concepts}
\begin{outline*}
\1 Wave equations are partial differential equations which relate the wavefunction (derivatives of wave displacement) to time and position, derived first from the general physical situation
\1 It is found that for traveling waves, the shape remains the same though the location of the disturbance changes with time
\2 This is due to the lack of loss of energy during propagation, such that there is no dissipation/damping of the wave
\2 As a result, it is found that $\psi(x, t) = \psi(x - \delta x, t - \delta t) = \psi(x - v\delta t, t - \delta t)$, since $\delta x = v \delta t$
\3 Thus, if $\delta t = t$, $\psi(x, t) = \psi(x - vt, 0) = \psi(u)$, such that the variables must appear in that combination for a forward traveling wave
\3 Thus, since the wavefunction is the same value over a specific curve, this single variable function is found for a traveling wave
\3 This all follows from the method of viewing waves as the motions of individual particles, collectivized, such that x determines which particle is being viewed, while t is at which time
\2 In addition, both sides can be subtracted by $\phi(x, t - \delta t)$, by the x-t relation, such that $\frac{\partial \phi}{\partial t} = -v \frac{\partial \phi}{\partial x}$
\3 This assumes that $\phi$ is continuous, or at least can be approximated as a continuous function
\3 Further, it is found similarly that $\frac{\partial^n \phi}{\partial t^n} = (-v)^n \frac{\partial^n \phi}{\partial x^n}$, though the equation would have to be modified due to the sign of n changing
\1 Generally, wave motion may be done relating the derivative with respect to time at some point to that relating the derivative with respect to location at that time
\2 This is then used to determine the general form of the wave equation, followed by inputting specific parameters for the solution
\end{outline*}
\subsection{Long String Waves}
\begin{outline*}
\1 For some string with tension T, assuming gravity effect is negligible, such that for some unit length mass M with length $\delta x$, tension is acting on both sides
\2 The angle is assumed to be slightly varying on the sides of the length, with small enough total displacement that $cos\theta_1 \appx cos\theta_2 \appx 1$
\3 Since the tension on the side in the direction of the wave is toward the positive wave function, while it is toward the negative for the other at some point, $F_{perpendicular} = T(sin\theta_2 - sin\theta_1)$
\3 When this distance is infinitesimal, the change in angle for each side is infinitesimal, moving the wave toward the slightly steeper direction
\2 By definition of the angle, $tan\theta = \frac{\partial \psi}{\partial x}$ for each end of $\delta x$
\3 Since the angles are small enough that $cos\theta \appx 1$, it can be stated that $sin\theta \appx tan\theta$
\3 As a result by this and Newton's Law, $F_n = T(\frac{\partial \psi}{\partial x}_{x_2} - \frac{\psi}{\partial x}_{x_1}) = M\delta x (\frac{\partial^2 \psi}{\partial t^2}|_x)$
\4 This is due to the fact that the acceleration at any one point is equal to the wave function second derivative with respect to time
\3 Thus, as divided by the length and taken to the limit as the length approaches 0, it is found that $\frac{\partial^2 \psi}{\partial t^2} = \frac{T}{M}\frac{\partial^2 \psi}{\partial x^2}$
\1 Using the formula for traveling waves in terms of one variable combined with the chain rule and the above formula, it is found that $v^2 \frac{d^2 \psi}{du^2} = \frac{T}{M} \frac{d^2 \psi}{du^2}$
\2 Thus, for a traveling string wave, $v = \pm \sqrt{\frac{T}{M}}$, such that the velocity depends on the tension and density
\1 Since the wave equation has linearity, such that superposition applies to it, the general solution is equal to the linear combination of individual solutions, spanning the solution set
\2 Thus, the two respective solutions are for a positive and negative velocity traveling waves, such that it is equal to the solution to a wave with positive and negative velocity
\3 Thus, this forms the idea that since the velocity can be in either direction when assuming a standing wave, all waves on the string are made up of two standing wave directions
\2 Thus, it is found that $\psi(x, t) = \psi(x \pm vt) = \psi_+(x - |v|t) + \psi_-(x + |v|t)$, to provide the forward and backward wave components, taking the specific solution at t = 0
\3 This forms $\psi(x, 0) = \psi_+(x) + \psi_-(x)$ and by the derivative with respect to t, $\frac{\partial \psi}{\partial t}(x, 0) = \frac{\partia \psi_+ (x - |v|t)}{\partial t}(x, 0) + \frac{\partial \psi_- (x + |v|t)}{\partial t}(x, 0)$
\4 The first equation serves to provide the initial condition for a unique solution
\3 By the chain rule, the second equation becomes $\frac{\partial \psi}{\partial t}(x, 0) = |v|(\frac{d\psi_-(x)}{dx} - \frac{d\psi_+(x)}{dx})$
\2 For some guitar string connected at two ends, [0, l], pulled at some point $x_0$ to initial height $a_0$ at t = 0, with initial velocity of 0
\3 By integrating the velocity formula for t = 0, $\psi_-(x) = \psi_+(x) + c$, such that at t = 0, $\psi(x, 0) = 2\psi_+(x) + c = 2\psi_-(x) - c$
\1 By these formulas for $\psi_\pm$ in terms of the initial wavefunction of some point, $\psi(x, t) = \frac{1}{2}\psi(x - |v|t, 0) + \frac{1}{2}\psi(x - |v|t, 0)$
\2 This assumes that x is not at the endpoints at some nonzero time, since it is undefined for a negative displacement or a displacement greater than the length
\2 This equations can be supplemented by the fact that $\psi(0, t) = \psi(l, t) = 0$ for any time t, acting as a boundary condition
\2 
\end{outline*}
\end{document}
