\documentclass[11 pt, twoside]{article}
\usepackage{textcomp}
\usepackage[margin=1in]{geometry}
\usepackage[utf8]{inputenc}
\usepackage{color}
\usepackage{indentfirst} %Comment out for no first paragraph indent
\usepackage[parfill]{parskip}
\usepackage{setspace}
\usepackage{tikz}
\usepackage{amsmath}
\usepackage{amsfonts}
\usepackage{amssymb}
\usepackage{enumitem}
\usepackage{outlines}

\usepackage{fancyhdr}
\pagestyle{fancy}
\cfoot{\hyperlink{content}{\thepage}}
\lhead{}
\chead{}
\rfoot{}
\lfoot{}
\rhead{}
\renewcommand{\headrulewidth}{0pt}
\renewcommand{\footrulewidth}{0pt}


\usepackage{hyperref}
\hypersetup {
	colorlinks,
	citecolor=black,
	filecolor=black,
	linkcolor=black,
	urlcolor=black
}

\newcommand{\sepitem}{0pt} %Added room between items on the list, not including a list and its sublist
\newcommand{\seppar}{1pt} %Between items and lists overall

\setenumerate[1]{itemsep=\sepitem, parsep=\seppar}
\setenumerate[2]{itemsep=\sepitem, parsep=\seppar}
\setenumerate[3]{itemsep=\sepitem, parsep=\seppar}
\setenumerate[4]{itemsep=\sepitem, parsep=\seppar}

\newenvironment{outline*}
{
	\begin{outline}[enumerate]
	}
	{\end{outline}
}

\newcommand{\foot}[1]{\hyperlink{#1}{$_#1$}}

\begin{document}

\title{Introduction to Physics III: Thermodynamics, Waves, and Relativity}
\author{Avery Karlin}
\date{Fall 2016}
\newcommand{\textbook}{Heat and Thermodynamics by Zemansky and Dittman, 7th Edition}
\newcommand{\teacher}{Dr. Johnson}

\maketitle
\newpage
\hypertarget{content}{\tableofcontents}
\vspace{11pt}
\noindent
\underline{Primary Textbook}: \textbook\\
\underline{Secondary Textbook}: Introduction to the Physics of Waves by Freegarde \\
\underline{Secondary Textbook}: Special Relativity by Helliwell \\
\underline{Teacher}: \teacher
\newpage

\input{thermo.tex}

\section{Freegarde Chapter 1 - Wave Motion Essence}
\begin{outline*}
\1 Physics can be viewed by Lagrangian Particle Theory or Euler Field Theory, as a duality of perspectives, most apparent on a quantum scale
\2 As a result, when considering a dynamic/time-dependent system, it becomes Kinetic Theory and Wave Theory respectively
\1 Waves are defined microscopically as a collective bulk disturbance, created at a point as a delayed response to the disturbance at adjacent points
\2 The disturbance progression to adjacent points is called the propagation, going through the medium
\2 The medium does not need to be linear, homogeneous, with sinusoidal or periodic propagation to classify as a wave
\2 Positions in the medium are defined by coordinates, each position requiring at least one additional variable to describe the disturbance
\3 The coordinates and disturbance variables are each wavefunctions, defined with respect to time
\1 Waves must be viewed in terms of both cause and effect, such that it is defined macroscopically as a time-dependent field effect due to finite speed of propagation of a causal effect
\2 Thus, it can be viewed as the solution to systems in dynamic equilibrium, more generalized than static equilibrium theory
\2 Electric Coulomb waves can thus be viewed as a result of a rotating dipole acting on a single charge at some distance
\3 If the dipole radius is assumed to be far less than the distance from the charge, it can be simplified such that $E(t) = \frac{2kq}{r_0^3}a(t - \frac{r_0}{c}$
\4 a(t) is the perpendicular height of each side of the dipole, while $r_0$ is the distance from the center of the dipole to the particle
\4 The time delay due to the relativistic limitations on which the change in force is received create the time lag/retardation, allowing the wave
\4 The magnetic component of electromagnetic waves are found by a Lorenz transformation taking into account the motion of the charges and dipoles themselves
\3 Gravitational waves have a similar wave effect for a gravitational single body, such that $g = \frac{Gm}{r_0^3}a(t - \frac{r_0}{c})$
\4 On the other hand, since the rotation of the causal body requires a third body to rotate with it similar to a dipole, it results for similar bodies in a higher order (inverse fourth root law), canceling out the effect to some degree
\4 Thus, it is found that g = $\frac{2Gma_0^2}{r_0^4}sin(2\omega t)$ assuming the distance between the two causal bodies is vastly smaller
\2 As a result, waves can be viewed as some function dependent on an action at a prior time at some other object
\1 It was believed until the 1900s that objects required a medium to influence each other in a vacuum, leading to the idea of the aether
\2 As a result of relativity and electromagnetism, it became apparent that there is no ether, such that forces can be sent through a vacuum
\2 On the other hand, for waves, there remains the concern of diffraction in a vacuum, with apparently nothing deflecting, and the concept of radiation, sending energy through a vacuum alone
\3 Current theories create no issues with these concepts, though they remain difficult to conceptualize
\1 Transverse wave motions, such as surface tension, electromagnetic, or gravitational, where the disturbance is perpendicular to the direction of wave propagation
\2 Longitudinal waves are those where the disturbance is parallel to the direction of propagation, such as sound, generally due to pressure differences on each side
\3 On the other hand, gravitational or electromagnetic waves can be created longitudinal rather than transverse
\2 Spin waves are those with both transverse and longitudinal components, such as seismology
\2 Waves can also be scalar quantity displacement, such as thermal waves (based on heat transfer), quantum wavefunction, or chemical composition waves (based on a reaction-diffusion system)
\2 Waves are also not required to be moving through a continuous medium, as long as they fulfill the definition
\end{outline*}
\section{Freegarde Chapter 2 - Basic Wave Equations}
\begin{outline*}

\end{outline*}
\end{document}
