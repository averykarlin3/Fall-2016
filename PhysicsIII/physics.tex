\documentclass[11 pt, twoside]{article}
\usepackage{textcomp}
\usepackage[margin=1in]{geometry}
\usepackage[utf8]{inputenc}
\usepackage{color}
\usepackage{indentfirst} %Comment out for no first paragraph indent
\usepackage[parfill]{parskip}
\usepackage{setspace}
\usepackage{tikz}
\usepackage{amsmath}
\usepackage{amsfonts}
\usepackage{amssymb}
\usepackage{enumitem}
\usepackage{outlines}

\usepackage{fancyhdr}
\pagestyle{fancy}
\cfoot{\hyperlink{content}{\thepage}}
\lhead{}
\chead{}
\rfoot{}
\lfoot{}
\rhead{}
\renewcommand{\headrulewidth}{0pt}
\renewcommand{\footrulewidth}{0pt}


\usepackage{hyperref}
\hypersetup {
	colorlinks,
	citecolor=black,
	filecolor=black,
	linkcolor=black,
	urlcolor=black
}

\newcommand{\sepitem}{0pt} %Added room between items on the list, not including a list and its sublist
\newcommand{\seppar}{1pt} %Between items and lists overall

\setenumerate[1]{itemsep=\sepitem, parsep=\seppar}
\setenumerate[2]{itemsep=\sepitem, parsep=\seppar}
\setenumerate[3]{itemsep=\sepitem, parsep=\seppar}
\setenumerate[4]{itemsep=\sepitem, parsep=\seppar}

\newenvironment{outline*}
{
	\begin{outline}[enumerate]
	}
	{\end{outline}
}

\newcommand{\foot}[1]{\hyperlink{#1}{$_#1$}}

\begin{document}

\title{Introduction to Physics III: Thermodynamics, Waves, and Relativity}
\author{Avery Karlin}
\date{Fall 2016}
\newcommand{\textbook}{Heat and Thermodynamics by Zemansky and Dittman, 7th Edition}
\newcommand{\teacher}{Dr. Johnson}

\maketitle
\newpage
\hypertarget{content}{\tableofcontents}
\vspace{11pt}
\noindent
\underline{Primary Textbook}: \textbook\\
\underline{Secondary Textbook}: Introduction to the Physics of Waves by Freegarde \\
\underline{Secondary Textbook}: Special Relativity by Helliwell \\
\underline{Teacher}: \teacher
\newpage

\input{thermo.tex}

\section{Freegarde Chapter 1 - Wave Motion Essence}
\begin{outline*}
\1 Physics can be viewed by Lagrangian Particle Theory or Euler Field Theory, as a duality of perspectives, most apparent on a quantum scale
\2 As a result, when considering a dynamic/time-dependent system, it becomes Kinetic Theory and Wave Theory respectively
\1 Waves are defined microscopically as a collective bulk disturbance, created at a point as a delayed response to the disturbance at adjacent points
\2 The disturbance progression to adjacent points is called the propagation, going through the medium
\2 The medium does not need to be linear, homogeneous, with sinusoidal or periodic propagation to classify as a wave
\2 Positions in the medium are defined by coordinates, each position requiring at least one additional variable to describe the disturbance
\3 The coordinates and disturbance variables are each wavefunctions, defined with respect to time
\1 Waves must be viewed in terms of both cause and effect, such that it is defined macroscopically as a time-dependent field effect due to finite speed of propagation of a causal effect
\2 Thus, it can be viewed as the solution to systems in dynamic equilibrium, more generalized than static equilibrium theory
\2 Electric Coulomb waves can thus be viewed as a result of a rotating dipole acting on a single charge at some distance
\3 If the dipole radius is assumed to be far less than the distance from the charge, it can be simplified such that $E(t) = \frac{2kq}{r_0^3}a(t - \frac{r_0}{c}$
\4 a(t) is the perpendicular height of each side of the dipole, while $r_0$ is the distance from the center of the dipole to the particle
\4 The time delay due to the relativistic limitations on which the change in force is received create the time lag/retardation, allowing the wave
\4 The magnetic component of electromagnetic waves are found by a Lorenz transformation taking into account the motion of the charges and dipoles themselves
\3 Gravitational waves have a similar wave effect for a gravitational single body, such that $g = \frac{Gm}{r_0^3}a(t - \frac{r_0}{c})$
\4 On the other hand, since the rotation of the causal body requires a third body to rotate with it similar to a dipole, it results for similar bodies in a higher order (inverse fourth root law), canceling out the effect to some degree
\4 Thus, it is found that g = $\frac{2Gma_0^2}{r_0^4}sin(2\omega t)$ assuming the distance between the two causal bodies is vastly smaller
\2 As a result, waves can be viewed as some function dependent on an action at a prior time at some other object
\1 It was believed until the 1900s that objects required a medium to influence each other in a vacuum, leading to the idea of the aether
\2 As a result of relativity and electromagnetism, it became apparent that there is no ether, such that forces can be sent through a vacuum
\2 On the other hand, for waves, there remains the concern of diffraction in a vacuum, with apparently nothing deflecting, and the concept of radiation, sending energy through a vacuum alone
\3 Current theories create no issues with these concepts, though they remain difficult to conceptualize
\1 Transverse wave motions, such as surface tension, electromagnetic, or gravitational, where the disturbance is perpendicular to the direction of wave propagation
\2 Longitudinal waves are those where the disturbance is parallel to the direction of propagation, such as sound, generally due to pressure differences on each side
\3 On the other hand, gravitational or electromagnetic waves can be created longitudinal rather than transverse
\2 Spin waves are those with both transverse and longitudinal components, such as seismology
\2 Waves can also be scalar quantity displacement, such as thermal waves (based on heat transfer), quantum wavefunction, or chemical composition waves (based on a reaction-diffusion system)
\2 Waves are also not required to be moving through a continuous medium, as long as they fulfill the definition
\end{outline*}
\section{Freegarde Chapter 2 - Basic Wave Equations}
\subsection{General Concepts}
\begin{outline*}
\1 Wave equations are partial differential equations which relate the wavefunction (derivatives of wave displacement) to time and position, derived first from the general physical situation
\1 It is found that for traveling waves, the shape remains the same though the location of the disturbance changes with time
\2 This is due to the lack of loss of energy during propagation, such that there is no dissipation/damping of the wave
\2 As a result, it is found that $\psi(x, t) = \psi(x - \delta x, t - \delta t) = \psi(x - v\delta t, t - \delta t)$, since $\delta x = v \delta t$
\3 Thus, if $\delta t = t$, $\psi(x, t) = \psi(x - vt, 0) = \psi(u)$, such that the variables must appear in that combination for a forward traveling wave
\3 Thus, since the wavefunction is the same value over a specific curve, this single variable function is found for a traveling wave
\3 This all follows from the method of viewing waves as the motions of individual particles, collectivized, such that x determines which particle is being viewed, while t is at which time
\2 In addition, both sides can be subtracted by $\phi(x, t - \delta t)$, by the x-t relation, such that $\frac{\partial \phi}{\partial t} = -v \frac{\partial \phi}{\partial x}$
\3 This assumes that $\phi$ is continuous, or at least can be approximated as a continuous function
\3 Further, it is found similarly that $\frac{\partial^n \phi}{\partial t^n} = (-v)^n \frac{\partial^n \phi}{\partial x^n}$, though the equation would have to be modified due to the sign of n changing
\1 Generally, wave motion may be done relating the derivative with respect to time at some point to that relating the derivative with respect to location at that time
\2 This is then used to determine the general form of the wave equation, followed by inputting specific parameters for the solution
\end{outline*}
\subsection{Long String Waves}
\begin{outline*}
\1 For some string with tension T, assuming gravity effect is negligible, such that for some unit length mass M with length $\delta x$, tension is acting on both sides
\2 The angle is assumed to be slightly varying on the sides of the length, with small enough total displacement that $cos\theta_1 \appx cos\theta_2 \appx 1$
\3 Since the tension on the side in the direction of the wave is toward the positive wave function, while it is toward the negative for the other at some point, $F_{perpendicular} = T(sin\theta_2 - sin\theta_1)$
\3 When this distance is infinitesimal, the change in angle for each side is infinitesimal, moving the wave toward the slightly steeper direction
\2 By definition of the angle, $tan\theta = \frac{\partial \psi}{\partial x}$ for each end of $\delta x$
\3 Since the angles are small enough that $cos\theta \appx 1$, it can be stated that $sin\theta \appx tan\theta$
\3 As a result by this and Newton's Law, $F_n = T(\frac{\partial \psi}{\partial x}_{x_2} - \frac{\psi}{\partial x}_{x_1}) = M\delta x (\frac{\partial^2 \psi}{\partial t^2}|_x)$
\4 This is due to the fact that the acceleration at any one point is equal to the wave function second derivative with respect to time
\3 Thus, as divided by the length and taken to the limit as the length approaches 0, it is found that $\frac{\partial^2 \psi}{\partial t^2} = \frac{T}{M}\frac{\partial^2 \psi}{\partial x^2}$
\1 Using the formula for traveling waves in terms of one variable combined with the chain rule and the above formula, it is found that $v^2 \frac{d^2 \psi}{du^2} = \frac{T}{M} \frac{d^2 \psi}{du^2}$
\2 Thus, for a traveling string wave, $v = \pm \sqrt{\frac{T}{M}}$, such that the velocity depends on the tension and density
\1 Since the wave equation has linearity, such that superposition applies to it, the general solution is equal to the linear combination of individual solutions, spanning the solution set
\2 Thus, the two respective solutions are for a positive and negative velocity traveling waves, such that it is equal to the solution to a wave with positive and negative velocity
\3 Thus, this forms the idea that since the velocity can be in either direction when assuming a standing wave, all waves on the string are made up of two standing wave directions
\2 Thus, it is found that $\psi(x, t) = \psi(x \pm vt) = \psi_+(x - |v|t) + \psi_-(x + |v|t)$, to provide the forward and backward wave components, taking the specific solution at t = 0
\3 The constant multipliers of the linear multiplier are contained within the functions themselves
\3 This forms $\psi(x, 0) = \psi_+(x) + \psi_-(x)$ and by the derivative with respect to t, $\frac{\partial \psi}{\partial t}(x, 0) = \frac{\partia \psi_+ (x - |v|t)}{\partial t}(x, 0) + \frac{\partial \psi_- (x + |v|t)}{\partial t}(x, 0)$
\4 The first equation serves to provide the initial condition for a unique solution
\3 By the chain rule, the second equation becomes $\frac{\partial \psi}{\partial t}(x, 0) = |v|(\frac{d\psi_-(x)}{dx} - \frac{d\psi_+(x)}{dx})$
\2 For some guitar string connected at two ends, [0, l], pulled at some point $x_0$ to initial height $a_0$ at t = 0, with initial velocity of 0
\3 By integrating the velocity formula for t = 0, $\psi_-(x) = \psi_+(x) + c$, such that at t = 0, $\psi(x, 0) = 2\psi_+(x) + c = 2\psi_-(x) - c$
\1 By these formulas for $\psi_\pm$ in terms of the initial wavefunction of some point, $\psi(x, t) = \frac{1}{2}\psi(x - |v|t, 0) + \frac{1}{2}\psi(x + |v|t, 0)$
\2 It follows from this and the general formula that the separate wave components are equal to half of the initial wavefunction, moving in each respective direction
\2 This assumes that x is not at the endpoints at some nonzero time, since it is undefined for a negative displacement or a displacement greater than the length
\2 This equations can be supplemented by the fact that $\psi(0, t) = \psi(l, t) = 0$ for any time t, acting as a boundary condition
\2 This is used to find the restriction that for any x, t, $\psi_+(x - |v|t) = -\psi_-(2l - x + |v|t) = \psi_+(x - |v|t - 2l)$, such that the component waves are periodic triangular/sawtooth
\3 This is found by the fact that $u_+ = 2x - u_i$, giving the relationship between the component functions if the overall function is 0, such that $u_+ = -u_i$ and $u_+ = 2l - u_-$ must be true for all u by the boundary conditions
\end{outline*}
\section{Freegarde Chapter 4 - Sinusoidal Waveforms}
\begin{outline*}
\1 Sinusoidal wave solutions are often used to approximate a wave, such that while they are not the only form of motion, can be used to form any linear wavemotion
\2 Pseudo-sinusoidal equations are also used for dissipative systems, able to be described generally by a main sinusoidal component
\1 The main sinusoidal wave form, fulfilling the traveling wave criteria, is $\psi(x, t) = asin(k(x - vt) + \phi)$, where $\phi$ is the initial phase/offset (relative to an origin) and a is the amplitude
\2 As a result, the wave angular frequency, $\omega = vk = kf\lambda$, where k is the constant wavenumber, continuing infinitely
\2 This can be rewritten by several variables, oscillation period ($\tau f = 1$), frequency ($\omega = 2\pi f$), wavelength ($k\lambda = 2\pi$), or the spectroscopists' wavenumber ($\bar{v}\lambda = 1$)
\1 Wavefronts are defined as a continuous function with constant sinusoidal phase, such that for the main form, $k(x - vt) + \phi = c$ for some constant c
\2 As a result, since it describes the movement of a point of constant wavefunction over the function over time ($x = vt + (c - \phi)k$)
\2 This includes the crest of a wave, or a point where the wavefunction remains 0, or a point that moves with the wave, keeping constant displacement
\2 The velocity of the wavefront, moving with constant phase, is called the phase velocity, denoted $v_p$, commonly just as v
\1 For some traveling string wave (u = x - vt), assuming amplitude is far smaller than wavelength and tension is constant, kinetic energy is by the transverse motion ($v_t = \frac{\partial \psi}{\partial t}$)
\2 Thus, $K = \frac{1}{2}\delta m v_t^2 = \frac{M \delta x}{2}(\frac{\partial \psi}{\partial t})^2$, by chain rule, $K = \frac{v_p^2M\delta x}{2}(\frac{\partial \psi}{\partial u})^2$
\3 Due to lack of longitudinal motion, each mass item is constant per unit length, and kinetic energy is purely transverse
\2 Potential energy of a string element is the work done stretching the string to displace it, creating the tension
\3 As a result, U = $\frac{T\delta x}{2}(\frac{\partial \psi}{\partial x})^2$, by the chain rule, U = $\frac{T\delta x}{2}(\frac{\partial \psi}{\partial u})^2$
\3 Thus, potential and kinetic energy of any length element are equal, such that $E = K + U = T(\frac{\partial \psi}{\partial x})^2\delta x = T(\frac{\partial \psi}{\partial u})^2\delta x$
\2 The energy density, $\epsilon$, is the energy per unit length
\2 This is an example of the virial theorem, stating that potential and kinetic energy are related for any dynamic system
\3 In addition, this shows that both forms of energy are related to the square of the wave amplitude
\2 As a result, the power is the energy moving through some point with the wave at some moment, such that $P = Tv_p(\frac{d\psi}{du})^2 = \epsilon v_p$
\2 It is noted that for superpositioned waves, the power and energy must take into account the sums of various directions of each
\2 Intensity is defined as the energy density per time, such that $I = \frac{d\epsilon}{dt}$
\1 Sinusoidal waves are generally used as an ideal solution class to the wave equations, due to commonality, but relying on a single frequency to be able to define phase velocity
\2 On the other hand, it is ideal due to acting as solutions to linear, dispersive systems with varying phase velocity as well, and due to standing waves and orthogonality
\1 Standing waves are a class of solutions to a non-dispersive, linear string system, in which the motion is standard, differing only in magnitude throughout the string
\2 As a result, it can be written in the format of $\psi(x, t) = X(x)T(t)$, placed within the string wave function to get two second order ODEs, forming sinusoidal solutions if X, T, and k are real
\2 In this case, $X(x) = X_0sin(kx + \phi), T(t) = T_0sin(\omega t + \theta)$, where $\omega/v_p$ are constant based on string characteristics
\3 Thus, $\psi(x, t) = \psi_0 sin(kx + \phi)sin(\omega t + \theta)$, where $\psi_0 = T_0X_0$
\2 These appear as non-moving waves, though by the trigonometric identities, can be converted to the sum of sinusoidal travelling waves, creating an alternate wave format
\2 Nodes of a standing wave are points where motion is always 0, such that $kx_m + \phi = m\pi$, where m is an integer, able to be fixed at those points without changing the wave
\3 Thus, for some fixed string of length l, assuming no phase shift, $k = \frac{n\pi}{l}$, where n is some integer, or $l = \frac{n\lambda}{2}$
\3 As a result, the allowed modes of vibration of a standing wave are all lengths of integer n for some wavelength
\4 For n = 1, it is the fundamental mode of vibration/first harmonic, while higher modes are harmonics/overtones (n = 2, the second harmonic, first overtone)
\3 Nodes can also be determined by the superposition of forward and backward travelling waves, such that if $\psi_+(u) = \psi_0 cos(ku)$, for each node, it must be true by boundary conditions that $\psi_+(u - 2x)$
\4 As a result, it must be true that $kx = n\pi$ where n is some integer
\4 This is able to be taken as well, such that $\psi_-(u) = -\psi_0 cos(k(x + |v|t))$, the standing wave equation can be derived
\2 The energy over an entire string, $E = T(\int (a\frac{\partial \psi_1}{\partial x} + b\frac{\partial \psi_2}{\partial x})^2)$, expanded to find that $E = a^2E_1 + b^2E_2 + 2Tab\int^l_0 \frac{\partial \psi_1}{\partial x}\frac{\partial \psi_2}{\partial x}dx$
\3 As a result, it is found that the energy is greater than the sum of the component energies
\3 On the other hand, it is found that sinusoidal functions, $sin(k_1x), sin(k_2x)$ have the integrated product of 0 if the functions are linearly independent, called orthogonal, making them ideal for energy analysis
\end{outline*}
\section{Freegarde Chapter 5 - Complex Wavefunctions}
\begin{outline*}
\1 Sinusoidal wave forms, due to the cycle of derivatives, are those with wavefunctions of the form $\sum_{n_i, n_j, \dots} a_{n_i, n_j, \dots} \frac{\partial^{n_i + n_j + \dots}\psi}{\partial x_i^{n_i} \partial x_j^{n_j} \dots} = 0$
\2 This assumes that the order of the sum of partial derivatives are all even or all odd ($n_i + n_j + \dots$)
\2 The main wave types not with sinusoidal wave forms, are dissipative waves, thermal diffusion equation, and the Schrodinger quantum equation, which are not all odd or even, and thus not fully sinusoidal
\3 In other cases, the complex harmonic waveform is used, such that it is a well-defined sinusoidal function, but with possibly imaginary amplitudes
\2 For some complex wavefunction, $\psi(x, t) = acos(kx - \omega t) + iasin(kx - \omega t) = ae^{i(kx - \omega t)}$, such that it is a travelling wave where $u = k(x - v_pt)$, and can be broken separably
\3 This can be proven to be the solution to any wavefunction, with any partial derivative sum parity
\3 It can also thus be broken up into a separable standing wave function, breaking the exponents up
\3 It can be superimposed with other complex exponentials to produce a real sinusoidal wave, allowing for easier calculation
\2 For some dissipative system, such as a string wave with a drag force, such that $F = -\mu v_t$, solved to find $k = \pm \sqrt{\frac{M}{T}}\omega\sqrt{1 + \frac{\mu i}{M \omega}} = \pm k_0(p + iq)$
\3 This can then be solved for the real and imaginary parts, such that $p^2 = \frac{1}{2}(1 + \sqrt{1 + (\frac{\mu}{M\omega})^2}), q = \frac{\mu}{2M\omega}$
\3 Weak damping is if $p \appx 1$, such that $\mu << M\omega$ and $\psi = ae^{i(k_0x - \omega t)}e^{-k_0qx}$, where $k_0q$ is the decay constant
\3 This provides the spatial decay wave equation, though it can be rewritten to create a complex $\omega$ and real k for the temporally decaying form
\3 These equations can also be solved by complex standing wave form of each standing wave part, separating the variables, creating a single variable differential equation for the same function
\3 These display dispersion as a result, changing the shape during propagation, rather than being a constant travelling wave, such that phase velocity and angular frequency are dependent on wave number
\4 As a result, dispersive waves are only sinusoidal travelling waves if the wavenumber is constant
\1 The displacement due to sinusoidal wave motion at some point is drawn with a phasor arrow, proportional in length to amplitude
\2 The phasor is rotated to show the phase of the wave propagation, able to be represented as a complex exponential, providing the time difference between each neighbor effect
\2 The phasors are thus able to be added as vectors, providing the overall wave displacement at some point
\3 For phasors of constant phase difference, it becomes an infinite geometric series, decreasing in magnitude from the original amplitude, such as light reflecting on oil
\3 The light is then reflected partially at the bottom of the oil to the top, repeating infinitely, summing for the net effect
\end{outline*}
\section{Freegarde Chapter 6 - Huygens Wave Propagation}
\subsection{Huygens Description}
\begin{outline*}
\1 Huygen's principle states that a wavefront of light propagates from an initial disturbance at the speed of light, equally in all directions, to form a spherical wavefront
\2 This has since been expanded to apply to all forward-traveling waves, rather than just light waves
\2 It also states that as the wavefront moves, it creates secondary sources, forming secondary wavefronts, being created at all positions continuously
\2 As a result of interference, adding the amplitudes as superposition of waves from different paths to enforce or cancel out others
\3 This results in the lack of backwards travelling waves, averaging to cancel each other out in that direction, superposed purely on the advancing planar wavefront
\4 This creates a wavefront tangent to the secondary wavelets, where the set of wavelets is the envelope of the wavelets
\3 This can be shown by phasars for each wavefront at some point, the central waves at the lowest angle due to being the first, and so forth
\4 This can be mathematically analyzed by finding the exponential function, $a(x, y) = \frac{e^{ikr}}{r}, r = \sqrt{x^2 + y^2}$, such that the amplitude is the inverse square dependent on the distance
\4 This can then be taken for all points at the same x value, integrated, to get the overall amplitude at that point
\4 It is found to be proportional to the Fresnel integral, $\int^\infty_{-\infty} e^{iks^2}ds$, where $r - x = s^2$ if kx >> 1
\4 The Cornu/Euler Spiral/Clothoid is the graph on the complex plane of the result of $a_0(s_0) = \int^{s_0}_0 e^{iks^2}ds$ for all $s_0$, graphing the line from the ends of the curve for the Fresnel integral vector
\4 This as a result approximates to a resultant with a phase of $\frac{\pi}{4}$
\3 Huygen's description implies a backward-travelling wavefront similarly, acting as a snapshot mechanism, such that each wavefront is replaced by the new wavefront of subsequent sources
\4 This is due to the velocity for the wave being of equal importance to the displacement at some point, for determining the wave, providing the initial conditions for a specific solution
\4 Thus, the lack of a specific mechanism or velocity of propagation prevents a specific waveform from being determined
\4 For electromagnetic waves, the electric and magnetic fields provide the displacement and velocity, the cross product being the Poynting vector of the direction of energy propagation
\4 It can also be explained by the forward wave producing secondary waves in phase with each other, while the backwards secondary waves don't produce in phase, but rather at any point after
\3 The Fresnel integral vector is found to have a $\frac{\pi}{4}$ angle/phase, doubled for a 2D wave, as the basis of Fresnel-Kirchhoff diffraction theory
\4 It could also be attributed to some degree to an overly literally derived mathematical model of Huygen's description, but can also be derived by Maxwell's equations
\4 In addition, if the assumption of kx >> 1 is false, the value collapses towards the real value in the direction of the wavefront
\1 Huygen's principle is not independent, but rather a means of calculating wave phenomena, able to be separately derived for individual wave types
\end{outline*}
\subsection{Geometric Properties}
\begin{outline*}
\1 Reflection is explained as creating secondary wavefronts at the collision with the surface, spreading outward, the waves hitting the mirror moving further, creating a new horizontal wavefront
\2 It can be then found geometrically that the wavefronts are at equal angles to the normal of the surface of the mirror, creating the Law of Reflection
\3 The rays are considered to be propagating in the plane of incidence, perpendicular to the material boundary
\2 This prevents the backward-travelling waves from being canceled out and it blocks/disrupts the forward travelling waves
\2 Reflection at a definite angle of a smooth surface is specular reflection, while a rough surface causes diffuse reflection
\2 Reflection takes place in all types of materials, allowing them to be seen by the colors reflected
\1 Refraction is described similarly, though it only partially disrupts the forward travelling wave
\2 It must take into account the relative speeds within each medium, such that the distances are based on time, such that the wave travels at a different velocity within the new medium, causing the angle change
\2 It is thus partially reflected, but forms a wave within the border, such that the angles must be specifically to allow the same time with different velocity
\1 The relationship between the angles and velocity are embodied by Snell's law of refraction, commonly written in terms of a refractive index, $\eta$ of the medium, such that $\eta = \frac{c}{v_p} \geq 1$, such that if $\theta$ is the angle from the normal line, $\eta_1 sin(\theta_1) = \eta_2 sin(\theta_2)$
\2 Gradual changes in the velocity, rather than at an interface, causes refraction, though not reflection since there is no sudden medium shift to trigger reflection, changing due to the density shift, such as in the atmosphere, creating mirages
\2 The refractive index is distinct for different materials and since all real waves are dispersive, since the velocity depends on the wavelength/wavenumber, the index of refraction does as well
\3 It follows that since $v = \lambda f$, if $\lambda_0$ is the wavelength in a vacuum, $\lamdba = \frac{\lambda_0}{\eta}$
\2 As a result of this law, the path of a ray is reversible, moving by the same path in the other direction
\2 It is possible for all light to be reflected from a non-opaque medium shift if the resultant angle is $90^o$, called the critical angle, such that above that, none can be refracted
\3 This is called total internal reflection, only able to occur if the index of refraction of the original medium is larger than that of the other medium, such that there is no refraction
\3 This allows a hollow Porro prism (45-90-45) to internally reflect all light within to each of the legs of the prism, due to air/glass having a lower critical angle
\3 This also allows a light pipe, such as a fiber optic cable of glass or plastic, to transmit an image, built so the angles are all greater than the critical angle, acting as a thinner and more insulator than normal wires
\1 Fermat's Principle of Least Time is distinct from, but consistant with, the Huygens construction, stating that a wave follows the path that takes the least time when travelling between two points
\2 The path is defined as the ray path, either of a narrow beam from a slit blocking the light wave, larger than a wavelength, smaller than the main components, such that there are enough secondary sources for the wavefront, but a narrow wavefront
\3 This can also be viewed as the direction of energy flow or as the normal to the wavefront
\2 This is found to coorespond to Snell's Law if applied to a medium transition, such that refraction produces the path of least time
\3 This can be thought of in terms of phasars, such that phasars of longer time vary more in direction, cancelling each other out, such that only the shortest time remain
\2 This law is akin to the Principle of Least Action in mechanics, such that the same law applies to particles as well as waves to an extent
\1 Dispersion is the dependence of wave speed and index of refraction on wavelength, not taking place in a vacuum, allowing light to be dispersed/separated into a spectrum when diffracted
\2 The amount of dispersion depends on the difference between the refractive indices, causing diamonds to have large dispersion
\1 Rainbows are created partially from light being partially refracted, refracted, and dispersed, most reflected into some angle less than $\theta$, the majority at $\theta$
\2 The down-sun point is the point at which for some drop of water, the horizontal axis intersects the angle, generally outside the drop, appearing as the bright line of the rainbow
\3 The angle depends on the refractive index for each wavelength, such that it is the angle light at the top of the drop reflects to for the primary rainbow, bottom for secondary
\3 For purple light, refracted the most, the angle is approximately $42^o$
\2 The primary rainbow has one reflection and two refractions, while the secondary rainbow has two reflections and two refractions, with larger angle for each wavelength ($52^o$ for red)
\2 The secondary rainbow as a result has reversed colors, with the areas between the primary and secondary and above the secondary darker than the rest, due to far fewer light waves
\3 The second rainbow is fainter due to less light remaining after two reflections causing some diffracted
\end{outline*}
\section{Freegarde Chapter 7 - Geometric Optics}
\begin{outline*}
\1 Geometric optics are the movement of a ray/wavefront moving in a straight line, and the effects of a change in medium, predating physical optics which requires wave behavior
\2 Optics are generally the study of electromagnetic waves/light
\2 Until Newton, light was believed to be a string of particles emitted, though as it displayed properties of waves, the consensus changed until Maxwell proved it was an electromagnetic wave
\3 It was later found that light was in discrete quanta, or photons, displaying wave-particle duality, leading to the study of quantum electrodynamics
\3 As a result, propagation is generally studied as a wave, while absorption and emission are studied as particles
\2 Wavefronts, while realistically somewhat curved, can be viewed on some small area relative to the total size of the wavefront, as a planar wave
\1 For some glass sphere, the paraxial approximation states that the angle made by any ray to the sphere is small, and the thin-lens approximation states that the distance from the horizontal axis is smaller than the radius
\2 Situations where the approximations do not apply are considered optical aberrations, preventing the height from being ignored
\2 The sphere is made up of a sphere connected to a continuing medium, found geometrically that $\frac{\eta_1}{d_1} + \frac{\eta_2}{d_2} = \frac{\eta_2 - \eta_1}{R}$
\3 $d_1$ is the distance from the ray origin on the axis to the point of refraction $d_2$ is the distance from when the diffracted ray hits the axis to the image, and R is the radius
\3 It is noted that it is independent of the height from the horizontal axis at which the ray hits the sphere
\3 The focus of the rays is the point at which the rays are refracted to, reversible, such that the points are images of each other
\3 It is also found that the same point is valid for all rays from a particular initial axis point
\2 For some point at the same distance from the sphere, but either above or below the axis, it is mapped as if the axis was rotated
\3 This creates a correspondence between each point, such that it creates image formation in the sphere, flipped in orientation
\2 The existence of a focal point at which all the rays converge produces a real image, allowing it to be recorded
\3 For some critical distance, the rays are parallel, such that they intersect at infinity, and no image is produced
\4 This distance is called the focal length, $f_n = \frac{\eta_{n}R}{\eta_2 - \eta_1}$, where n is the medium, $\eta_2$ is the sphere, $\eta_1$ is the other medium
\3 For a greater distance, they diverge from some point beyond the original source, called a virtual image
\3 Due to the images being reversible, the sources/objects can be real or virtual, either from a physical object or an image object
\3 Negative sources and images are denoted with a negative distance, positive if real, with the radius positive for a source outside the interface, negative for within
\1 Two spheres of different curvatures/radii are often combined to send the image into the original medium on the opposite side, called a lens
\2 Lenses can have each side calculated for diffraction, combined with the thin lens approximation, to give the lensmaker's formula, such that $\frac{1}{s_1} + \frac{1}{p_2} = \frac{\eta_2 - \eta_1}{\eta_1}(\frac{1}{R_1} - \frac{1}{R_2})$
\3 $s_1$ is the distance from the object to the lens, while $p_2$ is the distance from the lens to the image, defined purely by the lens properties
\3 For either an infinitely distant image or source, one term can be removed, such that $\frac{1}{f} = \frac{\eta_2 - \eta_1}{\eta_1}(\frac{1}{R_1} - \frac{1}{R_2}$
\3 As a result, it can also be written as $\frac{1}{s} + \frac{1}{p} = \frac{1}{f}$
\2 Converging lenses are those with a positive focal length, while those with a negative focal length are diverging lenses
\2 This can be extended to a spherical mirror, such that $\eta_1 = -\eta_2$ to describe reflection, with $R_1 = R_2$
\3 As a result, $\frac{1}{f} = \frac{-2}{R} = \frac{1}{s} + \frac{1}{p}$, such that the focal length is half the curvature radius
\3 This applies to both forms of mirrors, such that R is positive if the image is outside the radius interface, negative otherwise
\1 The focal plane is the plane that passes perpendicular to the lens axis, such that the angle of the parallel rays to the lens axis is based on the initial height of the object in the plane
\2 Thus, it is determined by the Gaussian lens equation that $y = f\theta$
\end{outline*}
\section{Notes}
All waves can be viewed as the sum of a sinusoidal traveling wave, such that $\psi(x, t) = Acos(k(x-vt) + \phi) = Acos(kx - \omega t + \phi)$ by forier transform

Common differential trick is separation of variables
General wave equation - 
$\frac{\partial^2 \psi}{\partial t^2} = v^2\frac{\partial^2 \psi}{\partial x^2}$ then $\frac{1}{v^2}\frac{1}{T}{d^2T}{dt^2} = \frac{1}{X}\frac{d^2X}{dx^2} = C$ if $\psi = X(x)T(t)$

$X(x) = X_0 sin(\sqrt{C}X + \phi) = X_0sin(\kx + \phi)$
$T(t) = T_0 sin(kvt +  \theta)$
$\psi(x, t) = \psi_0 sin(kx + \phi) sin(\omega t + \theta)$, giving the standing wave form

Since $sinkxsin\omega t = \frac{1}{2}(cos(kx - \omega t) - cos(kx + \omega t))$, it can be written as travelling waves
For a string attached at the ends, since $\psi(0, t) = Asin\phi sin(\omega t + \theta) = 0$ for a standing wave, and $\psi(L, t) = Asin(kL + \phi)sin(\omega t + \theta) = 0$, kL is limited to specific values for a standing wave to occur
v independent for a string, such that $f\lambda$ is constant, $f_n = \frac{nv}{2L} = nf_1$, where $f_1$ is the fundemental frequency on the string
Fundemental frequency is the frequency of a standing wave equal to half a wavelength

$\psi(x, t) = \sum a_n sin(\frac{n\pi x}{L})cos(\frac{n\pi vt}{L})$, represented as a fourier series, equal to just the sine component at t = 0, from n = $a_1$ to $a_\infy$

In addition, it proceeds to get more rounded as it goes along, due to damping causing the higher frequencies to fade?? or vice versa

Each stationary standing wave point is a normal node, while normal string waves are triangular, they can be approximated by a standing wave curve, getting closer as the fourier series is allow to extend to infinity

Three terms is generally enough to approximate a string properly
Superposition of electronic states leads to electrons in many distinct states simultaneously until observed

Energy in Wave Motion - $K = \frac{1}{2}\Delta mv^2  = \frac{1}{2} \mu \Delta x (\frac{\partial \psi}{\partial t}^2$

For string, $\Delta L = \sqrt{(\Delta x)^2 + (\Delta \psi)^2} - \Delta x = \Delta x (\sqrt{1 - (\frac{\partial \psi}{\partial x})^2} - 1)$

$\sqrt{1 + x^2} = 1 + \frac{1}{2}x^2$
$U = T\Delta L$, $E = T(\frac{\partial \psi}{\partial x})^2 \Delta x$

Energy can be expressed in normal nodes, electromagnetic waves in powerpoint

Poynting vector in the direction of propogation of light, equal to the cross product of the electric and magnetic fields
$E = E_0 cos(kx - \omega t), B = B_0 cos(kx - \omega t)$
v = c = $\frac{1}{\sqrt{\mu_0 \epsilon_0}}$
At some angle, total internal reflection takes place, at which point there is no refraction, only reflection, only when moving from high index of refraction to low
Called critical angle, when the transmission angle is parallel to the medium boundary

$c = \frac{1}{\sqrt{\epsilon \mu_0}} = \frac{1}{\sqrt{\kappa \epsilon_0 \mu_0}}, n = \sqrt{\kappa} > 1$

Mirage is caused by hot air having a lower angle of refraction, such that light bends, allowing two images of the same object, one down, one straight
It is turned upside down due to the top part being bent such that the ray is below the unbent ray of the bottom

Polarization in the direction of oscillation, using a filter barrier to remove all those not oscillating in the right direction
Can be done by a metal grating, the electric charges changing to cancel out the field with electric fields in that orientation
Can also be done by long stretched chain molecules, acting similar to a metal due to fairly mobile electrons

Intensity = $\frac{EB}{\mu} = \frac{E^2}{c\mu_0}$
\end{document}
