\documentclass[11 pt, twoside]{article}
\usepackage{textcomp}
\usepackage[margin=1in]{geometry}
\usepackage[utf8]{inputenc}
\usepackage{color}
\usepackage{indentfirst} %Comment out for no first paragraph indent
\usepackage[parfill]{parskip}
\usepackage{setspace}
\usepackage{tikz}
\usepackage{amsmath}
\usepackage{amsfonts}
\usepackage{amssymb}
\usepackage{enumitem}
\usepackage{outlines}

\usepackage{fancyhdr}
\pagestyle{fancy}
\cfoot{\hyperlink{content}{\thepage}}
\lhead{}
\chead{}
\rfoot{}
\lfoot{}
\rhead{}
\renewcommand{\headrulewidth}{0pt}
\renewcommand{\footrulewidth}{0pt}


\usepackage{hyperref}
\hypersetup {
	colorlinks,
	citecolor=black,
	filecolor=black,
	linkcolor=black,
	urlcolor=black
}

\newcommand{\sepitem}{0pt} %Added room between items on the list, not including a list and its sublist
\newcommand{\seppar}{1pt} %Between items and lists overall

\setenumerate[1]{itemsep=\sepitem, parsep=\seppar}
\setenumerate[2]{itemsep=\sepitem, parsep=\seppar}
\setenumerate[3]{itemsep=\sepitem, parsep=\seppar}
\setenumerate[4]{itemsep=\sepitem, parsep=\seppar}

\newenvironment{outline*}
{
	\begin{outline}[enumerate]
	}
	{\end{outline}
}

\newcommand{\foot}[1]{\hyperlink{#1}{$_#1$}}

\begin{document}

\title{Introduction to Physics III: Thermodynamics, Waves, and Relativity}
\author{Avery Karlin}
\date{Fall 2016}
\newcommand{\textbook}{Heat and Thermodynamics by Zemansky and Dittman, 7th Edition}
\newcommand{\teacher}{Dr. Johnson}

\maketitle
\newpage
\hypertarget{content}{\tableofcontents}
\vspace{11pt}
\noindent
\underline{Primary Textbook}: \textbook\\
\underline{Secondary Textbook}: Introduction to the Physics of Waves by Freegarde \\
\underline{Secondary Textbook}: Special Relativity by Helliwell \\
\underline{Teacher}: \teacher
\newpage

\section{Chapter 1 - Temperature and the Zeroth Law of Thermodynamics}
\subsection{Definition of Thermodynamics}
\begin{outline*}
\1 The study of a natural system requires the creation of a boundary, separating a section of space and matter, or the system, from its surroundings, called closed if no matter is able to cross the boundary, open if there is an exchange of matter
\1 Systems are then studied on either a macroscopic/human scale or a microscopic/molecular scale
\2 Systems are described on a macroscopic scale by composition, mass, volume, and temperature, and other aggregate properties, acting as macroscopic coordinates for the macroscopic description
\3 Macroscopic coordinates have the properties of assuming nothing special about the matter structure, fields, or radiation, require few to describe a system, suggested by sensory observation, and can be directly measured
\2 Systems are described on a microscopic scale by statistical mechanics, describing the populations/number of particles in each energy state at equilibrium, and the interactions of particles with each other by collisions and fields, and with other systems in an ensemble
\3 The equilibrium state is the state of the highest probability, and the system is assumed to have some high number of particles
\3 Microscopic descriptions have the properties of making assumptions about the structure of matter, fields, and radiation, requires many quantities, is based mathematical models rather than observation, and must be calculated rather than measured
\2 The scales must reconcile to the same conclusion, with the macroscopic view as the average over some amount of time of the microscopic characteristics
\3 Since the microscopic view requires assumptions and models rather than observations, it is able to change as the result of increased data, unlike the macroscopic view which is used to test the assumptions
\1 Thermodynamics is the study of the macroscopic properties of nature, especially the temperature of the system, and the identification of thermodynamic relations based on the fundamental laws
\2 Mechanical coordinates used within classical mechanics are from the mechanical/external energy of the system to measure the movement of the system overall, while thermodynamics uses macroscopic coordinates to deal with the internal energy of the system, called thermodynamic coordinates
\2 Systems described by thermodynamic laws are thermodynamic systems, found within each discipline
\end{outline*}
\subsection{}
\begin{outline*}

\end{outline*}
\end{document}
