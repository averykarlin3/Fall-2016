\documentclass[11 pt, twoside]{article}
\usepackage{textcomp}
\usepackage[margin=1in]{geometry}
\usepackage[utf8]{inputenc}
\usepackage{color}
\usepackage{indentfirst} %Comment out for no first paragraph indent
\usepackage[parfill]{parskip}
\usepackage{setspace}
\usepackage{tikz}
\usepackage{amsmath}
\usepackage{amsfonts}
\usepackage{amssymb}
\usepackage{enumitem}
\usepackage{outlines}

\usepackage{fancyhdr}
\pagestyle{fancy}
\cfoot{\hyperlink{content}{\thepage}}
\lhead{}
\chead{}
\rfoot{}
\lfoot{}
\rhead{}
\renewcommand{\headrulewidth}{0pt}
\renewcommand{\footrulewidth}{0pt}


\usepackage{hyperref}
\hypersetup {
	colorlinks,
	citecolor=black,
	filecolor=black,
	linkcolor=black,
	urlcolor=black
}

\newcommand{\sepitem}{0pt} %Added room between items on the list, not including a list and its sublist
\newcommand{\seppar}{1pt} %Between items and lists overall

\setenumerate[1]{itemsep=\sepitem, parsep=\seppar}
\setenumerate[2]{itemsep=\sepitem, parsep=\seppar}
\setenumerate[3]{itemsep=\sepitem, parsep=\seppar}
\setenumerate[4]{itemsep=\sepitem, parsep=\seppar}

\newenvironment{outline*}
{
	\begin{outline}[enumerate]
	}
	{\end{outline}
}

\newcommand{\foot}[1]{\hyperlink{#1}{$_#1$}}

\begin{document}

\title{Introduction to Physics III: Thermodynamics, Waves, and Relativity}
\author{Avery Karlin}
\date{Fall 2016}
\newcommand{\textbook}{Heat and Thermodynamics by Zemansky and Dittman, 7th Edition}
\newcommand{\teacher}{Dr. Johnson}

\maketitle
\newpage
\hypertarget{content}{\tableofcontents}
\vspace{11pt}
\noindent
\underline{Primary Textbook}: \textbook\\
\underline{Secondary Textbook}: Introduction to the Physics of Waves by Freegarde \\
\underline{Secondary Textbook}: Special Relativity by Helliwell \\
\underline{Teacher}: \teacher
\newpage

\section{Chapter 1 - Temperature and the Zeroth Law of Thermodynamics}
\subsection{Definition of Thermodynamics}
\begin{outline*}
\1 The study of a natural system requires the creation of a boundary, separating a section of space and matter, or the system, from its surroundings, called closed if no matter is able to cross the boundary, open if there is an exchange of matter
\1 Systems are then studied on either a macroscopic/human scale or a microscopic/molecular scale
\2 Systems are described on a macroscopic scale by composition, mass, volume, and temperature, and other aggregate properties, acting as macroscopic coordinates for the macroscopic description
\3 Macroscopic coordinates have the properties of assuming nothing special about the matter structure, fields, or radiation, require few to describe a system, suggested by sensory observation, and can be directly measured
\3 While mechanics describes the external properties of a system and its energy, thermodynamics describes the internal properties
\2 Systems are described on a microscopic scale by statistical mechanics, describing the populations/number of particles in each energy state at equilibrium, and the interactions of particles with each other by collisions and fields, and with other systems in an ensemble
\3 The equilibrium state is the state of the highest probability, and the system is assumed to have some high number of particles
\3 Microscopic descriptions have the properties of making assumptions about the structure of matter, fields, and radiation, requires many quantities, is based mathematical models rather than observation, and must be calculated rather than measured
\2 The scales must reconcile to the same conclusion, with the macroscopic view as the average over some amount of time of the microscopic characteristics
\3 Since the microscopic view requires assumptions and models rather than observations, it is able to change as the result of increased data, unlike the macroscopic view which is used to test the assumptions
\1 Thermodynamics is the study of the macroscopic properties of nature, especially the temperature of the system, and the identification of thermodynamic relations based on the fundamental laws
\2 Mechanical coordinates used within classical mechanics are from the mechanical/external energy of the system to measure the movement of the system overall, while thermodynamics uses macroscopic coordinates to deal with the internal energy of the system, called thermodynamic coordinates
\2 Systems described by thermodynamic laws are thermodynamic systems, found within each discipline
\1 Thermodynamics is the branch of physics which describes the relationships between all forms of energy, conserved in an isolated system by the Law of Conservation of Energy
\2 Heat is energy in transit from one body at higher temperature to another with lower temperature, with 1 calorie as the heat to raise the temperature of 1 g of water by $1^oC$ at 1 atm, equal to 4.184 J
\end{outline*}
\subsection{Thermal Equilibrium and Temperature}
\begin{outline*}
\1 For some thermodynamic system of constant mass and composition, where to describe the system requires two coordinates, X and Y
\2 The system is at equilibrium state if the coordinates would remain constant if the external conditions are unchanged
\2 Equilibrium states depend on the proximity to other systems and the boundary within systems (adiabatic if any equilibrium state is present on both, such as wood, while diathermic if there must be specific combinations of equilibrium states, such as metal)
\3 Thermal equilibrium is achieved by multiple systems separated by a diathermic wall when they have reached a combined system equilibrium
\3 Diathermic walls are thus boundaries that allow heat transfers, while adiabatic do not allow heat transfers, both not permitting matter transfers (assuming it does not break due to stress from the systems)
\2 The Zeroth Law of Thermodynamics states that if two systems are each in thermal equilibrium with another system simultaneously, then they are at thermal equilibrium with each other as well
\3 It is called this due to the next two relying on this definition, found experimentally as a fundamental law of the discipline, creating the concept of temperature as a universal quantity
\1 Temperature is designed as the macroscopic measure of hotness from an object observed, and microscopically as the average kinetic energy of particles, defined as some number $\geq 0$ in science (Kelvin), ignoring the terms of coldness
\2 For any given state of a system, there exists a curve, or isotherm, of the locus of all states of another system which would be in thermal equilibrium with the first system, experimentally found to generally be continuous over some region of the curve
\3 The same can be found for the other system at each set of coordinates, such that by the zeroth law, each isotherm has a corresponding isotherm in the other system, such that all points on those lines are at equilibrium with each other
\3 The property which determines if a system is at thermal equilibrium is called temperature, since all that is needed is a correspondence of equilibrium between two systems and a reference, such that it is a scalar, rather than a vector if there was not a transitive property
\2 Thus, for systems at the same temperature, they are on corresponding isotherms and thus at thermal equilibrium, such that temperature is a measure of thermal equilibrium
\3 The converse of this is also true, such that thermal equilibrium implies same temperature
\1 Temperature equality can be determined by a capillary of mercury, equally filled and sized, measuring the equality of the height the mercury rises, experimentally found to signify equilibrium, called a thermoscope
\2 Thermometers measure the temperature on an empirical scale, using some rule to assign a value to each isotherm, finding some path on the isotherm plane, with the varying variable as the thermometric property, the thermometric function, $\theta(X)$ to get the temperature
\3 The traditional mercury volume thermometer relies on the asymetrical oscillation as energy increases, such that it is more likely to be at the larger side of oscillation, creating an average larger volume
\4 This is defined by $U(r) = 4\epsilon((\frac{\omega}{r})^{12} - (\frac{\omega}{r})^6)$, such that $\epsilon$ is the minimum energy, r is the distance between molecules, and $\omega^{\frac{2}{3}} = r_0$, where $r_0$ is the minimum distance between molecules
\4 All objects expand outward as a result, such that there would be increased space in a cavity as well, expanding only outward
\2 The historic scale mercury pressure thermometer (constant volume) is based on a linear relationship, $\theta(X) = aX + c$, without a constant defining an absolute temperature scale (absolute zero at $0^o)$
\3 The same relation when applied to other thermometers or other thermometer systems of the same type (the international standard being hydrogen pressure) produces other scales, with some defined coefficient for each
\2 Thus, when a thermometer is placed in contact with a chosen standard system in a reproducible state, called the fixed point temperature
\3 Until 1954, the world standard was Celsius, but which was too difficult to measure accurately due to water surrounding ice and sensitivity to minor pressure fluctuations
\3 After, the Kelvin system is based on the triple point of water (three state equilibrium, called the standard fixed point), able to measured accurately, as an absolute system, assigned 273.16 K, equal to $0.01^o C$
\4 This is measured by putting water under such pressure in a sealed tube that it begins to boil, using a freezing mixture to form a layer of ice on the edge, after which the mixture is removed and water forms between the ice and the edge, still evaporating, at the triple point
\end{outline*}
\subsection{Thermometer Types}
\begin{outline*}
\1 For each type of thermometer, a test using the same equation would produce drastically different results from those of the standard at all values except the reference
\2 Hydrogen gas pressure thermometers tend to be negligibly different at standard pressures, acting as the reference for normal temperature and pressure
\1 Gas thermometers are made up of a measuring gas containing bulb, attached to a mercury column through a capillary, keeping the volume of the gas constant by adjusting the height of the mercury column
\2 The mercury column has two columns, one attached to the capillary, the other to the dead space, with a column of the measuring gas reaching from underneath the columns to the reservoir
\3 If high enough pressure, the mercury can rise into the dead space/nuisance volume at the top of the capillary, adjusting the mercury in the column to the point that the mercury fills the capillary column fully to keep volume constant
\2 Thus, at this point, the pressure of the main column is the atmospheric pressure added to the difference in heights between the columns
\3 It can then be measured when the bulb is surrounded by the triple point water and when surrounded by the measured system
\2 The pressures must be corrected for errors due to the gas being a different temperature than the bulb, the capillary not being uniform temperature, volume changes in any part as the temperature and pressure change
\3 It can also have error due to gradient in the capillary due to the diameter being close to the mean free path of the gas, gas adsorbed by the device, especially at low temperatures, and temperature and compressibility of the mercury creating error
\2 As a result, once corrected, the measuring gas behavior appears close to that of an ideal gas
\3 The basis of gas thermometers was the ideal-gas law, stating PV = nRT, where n is the number of moles of gas, T is the Kelvin temperature, and R is the molar gas constant
\3 This is able to be used for the reference, to derive the same relation between triple point pressure/temperature and measured pressure/temperature ($\frac{P}{P_TP} = \frac{\theta}{273.16 K}$), such that the temperature definition thus assumes an ideal gas for a Kelvin thermometer
\2 Helium is the best measuring gas, not diffusing through platinum at high temperatures, but not becoming a liquid until extremely low temperatures
\1 Vapor pressure thermometry depends only on the vapor pressure of the measuring gas at some temperature, thus easy to produce, and is easy to measure
\1 Platinum resistance thermometry is secondary, due to being fundamentally defined in terms of temperature, but with a broader range than gas thermometers
\2 This is done by winding a long, thin wire around a frame, loose enough to avoid strain when compressing, able to also be wrapped around the object being measured, drilled into the object, or bonded onto the surface
\2 Potentiometric resistance thermometers have zero direct current into the leads, while bridge circuits are balanced such that there is no alternating current through
\3 Bridge circuits are made possible as thermometers due to inductive voltage-dividers/ratio transformers and high sensitivity/signal-to-noise lock-in amplifiers
\2 Within the ideal measurement range, $R'(T) = R'_{TP}(1 + aT + bT^2)$, where a and b are constants, generally measured as $W(T) = 1 + aT + bT^2$, measuring the ratio of resistivity, which is less prone to material tension/composition error
\1 Radiation thermometry, including optical, radiation, infrared, and total radiation, are used to measure thermal/blackbody radiation, using Planck's radiation law relating temperature to radiance, depending only on the walls of a closed body, not the shape
\2 The dimensions must be larger than the wavelengths themselves, and there must be a small hole to allow radiation to escape to allow measurement, negligible enough to not disturb equilibrium
\2 Pyrometers were made to measure high temperatures without contact, comparing visible radiation over a small band with that of a standard, using a photoelectric detector, correcting for emissivity of the source at that band
\2 Total-radiation pyrometers have a larger range able to measure, but are less precise, measuring all radiation from the object, rather than just blackbody
\1 Thermocouples are an outdated method of measurement due to high inaccuracy of 0.2 K, made of two wires connected to copper wires in a melting point reference junction, connecting the two wires and measuring the EMF at various values
\2 $\epsilon = c_0 + c_1\theta + c_2\theta^2 + c_3\theta^3$, where $c_n$ are constants based on the materials of the thermocouple, though for a small range of temperature, $c_3$ approximates to 0
\2 The temperature range of the thermocouple depends on the materials used, advantageous because of the small, conductive mass allowing rapid temperature equilibrium with the material being measured
\2 The copper wires can be connected to electrical circuits for monitoring of electrical equipment temperatures or to a simple voltmeter for measurement
\end{outline*}
\subsection{Temperature Scales}
\begin{outline*}
\1 As the triple point pressure of a gas (due to lessening volume), approaches 0, the resulting temperature calculated is the ideal-gas temperature
\2 While thermometers depend on the specific measuring gas properties to provide the temperature, as it approaches the ideal gas temperature, it becomes independent of the measuring gas, though based on the properties of gases overall, rather than an individual gas, behaving ideally
\3 Due to this ideal specific gas-independent temperature, it is an ideal state of matter for measuring of temperature
\2 For the temperature region in which a gas thermometer may be used, the measured Kelvin temperature scale and ideal gas temperature scales are identical
\2 It is noted that the idea of lack of atomic motion at absolute zero cannot be assumed, due to relying on the complete equivalency of temperature and atomic motion on different scales 
\3 In addition, at absolute zero, there is some residual energy expected based on quantum mechanics, resulting in the zero-point energy quantity
\1 The Celsius scale was used prior to the Kelvin temperature scale until 1954, which slightly shifted the base from the ice point to the triple point of water (273.16 K) for accuracy and modifying to an absolute scale, with the same degree of magnitude between values
\2 Thus, the Celsius and Kelvin scales simply have a constant difference of the measured ice point, equal to 273.15 K
\2 As a result, the modern Celsius scale no longer has a fixed ice/steam point for water, but rather purely the triple point, such that Kelvin is the standard
\1 Fahrenheit and Rankine are based on a scale 5/9ths that of the Celsius and Kelvin, based on the triple point of water, such that the ice point is found to be $32^o F$, providing a relationship to Celsius
\1 The International Temperature Scale of 1990 was the creation of a practical scale for routine measurements and calibration of instruments, to ease measurement from time-consuming calibration by the single reference point,  providing a series of fixed points for comparison
\2 It also provides close measurement interpolation for the remainder, providing a close approximation to the Kelvin scale, by other thermometer types
\3 This uses polynomial approximations of constants to approximate the scale of another thermometer for that region
\2 At the minimum of 0.65 K to 3.2 K, it is defined by the $^3He$ vapor pressure-temperature relation, from 1.25 K to 2.1768 K to 5 K by $^4He$ of the same type
\3 $^3He$ fails below 0.3 K due to being too small to measure, and fails above 3.32 K due to the critical point (where only gas exists), while $^4He$ fails below 1 K due to superfluid behavior causing less variation with temperature and errors, and above 5.2 K due to the critical point
\2 From 3 K to 24.5561 K by either He gases as a constant-volume gas thermometer
\2 From 13.8033 K to 1234.93 K, it is defined by platinum resistance thermometers with fixed points within and deviation functions between, creating 11 sub-ranges
\2 Above that, it is measured by an optical pyrometer, using blackbody spectral radiance concentration ratios by Planck's radiation law, using only a single reference of the freezing point of gold, silver, or copper, replacing the thermocouple
\end{outline*}
\section{Chapter 2 - Simple Thermodynamic Systems}
\subsection{Thermodynamic Equilibrium}
\begin{outline*}
\1 Change of state of a system is when the coordinates change either spontaneously or due to outside influence, isolated if the former, such that there is no outside influence, rarely found practically
\2 Systems not in mechanical equilibrium (with no unbalanced force or torque in the system interior or the surroundings/no unequal pressure) will have a change of state until it is corrected
\2 Systems not in chemical equilibrium (unequal chemical potential) result in a transfer of matter through or out/in the system, or a spontaneous change of internal structure as a reaction to restore it
\2 Systems in thermal equilibrium are those with no change in the system's thermodynamic coordinates when separated diathermically, such that there is no heat transfer (meaning temperature is equal with surroundings and within the system)
\3 Otherwise, change of state will occur until thermal equilibrium is restored
\2 It is noted that systems not at thermodynamic equilibrium, but static over time due to some outside influence, can be broken into small increments, where it can be considered in equilibrium
\1 Thermodynamic equilibrium is when thermal, mechanical, and chemical equilibrium are found simultaneously, such that it can be described by thermodynamic macroscopic coordinates at a particular instant, though not over time
\2 Thus, when not at mechanical, chemical, and thermal equilibrium, coordinates like composition, temperature, and pressure cannot be used to apply to the system as a whole, as it moves through nonequilibrium states
\1 Equations of state relate the coordinates of a thermodynamic equilibrium system to each other, based on the individual specifics of a system, such that it is determined by molecular theory or experiment, not general theory, but can be assumed to exist for all thermodynamic equilibrium systems since all coordinates are not independent for a system
\2 At low pressures, the state equation of ideal gas is PV = nRT, or Pv = RT, where v is the molar volume ($\frac{V}{n}$)
\2 At high pressures, it is represented by the van der Waals equation, based on particle interactions and sizes, such that $(P + \frac{a}{v^2})(v - b) = RT$, where v is the molar volume
\3 a and b are gas-specific constants, since equations of state are defined by existence, rather than any particular form without constants
\3 a is the molecular attraction while b is the atomic short range repulsion of the system, such that it accounts for higher pressures in which gases behave less ideally 
\2 Nonequilibrium states cannot be described purely by coordinates as a whole, such that there is no equation of state for the system
\end{outline*}
\subsection{Simple Systems}
\begin{outline*}
\1 Simple systems are those which are described by three thermodynamic coordinates in an equation of state, with the special case of PVT systems called hydrostatic systems
\1 Hydrostatic systems are also defined as an isotopic (uniform) system of constant mass and composition that exerts uniform hydrostatic pressure on the surroundings if there is no gravitational or electromagnetic effects
\2 They are divided into three catagories, pure substances (a single compound of any mixture of states), homogeneous mixture of different compounds (mixture of different compounds of the same state or a solution), and heterogeneous mixtures (mixtures of multiple homogeneous multiple compound mixtures of different states)
\3 It has been found experimentally that systems of a single state can be described by three coordinate equations of state, PVT
\2 These equations can describe some macroscopically small change in a measurement by a differential, though since they are macroscopic coordinates, it describes an amount greater than a few particles changing, rather than an infinitesimal amount
\2 Thus, for some function V = V(T, P), $dV = \frac{\partial V}{\partial T}(T, P)dT + \frac{\partial V}{\partial P}(T, P)dP$, abbreviated as $dV = \frac{\partial V}{\partial T})_PdT + \frac{\partial V}{\partial P})_TdP$
\3 The average coefficient of volume expansion is defined as the change in volume per unit volume divided by the change in temperature at constant pressure
\4 As the change in temperature, and thus volume, approaches 0, the differential coefficient of volume expansion, or the volume expansivity, $\beta = \frac{1}{V}(\frac{\partial V}{\partial T})$
\4 It is generally a positive number, though there are exceptions such as water for $0-4^o C$ melting point at which point water reaches maximum density
\4 This is due to hydrogen bonding of ice and surface tension of water and is generally a function of P and T
\4 It often varies negligibly with respect to P and T often within a small temperature and pressure range, such that it can be considered a constant over the range
\4 It included division by V to signify the relative/percentage change in volume, rather the absolute change
\3 The average bulk modulus is defined as $\frac{-V\Delta P}{\Delta V}$, due to the pressure increasing causing a decrease in volume, such that it is a positive value
\4 As the change in volume, and thus pressure, approaches 0 as temperature is constant, it is called the isothermal bulk modulus, $B = -V(\frac{\partial P}{\partial V})$
\4 Isothermal compressibility, or the reciprocal of isothermal bulk modulus, $\kappa = \frac{-1}{V}(\frac{\partial V}{\partial P})$, and is similarly often constant for a material over some PT range
\2 This can be similarly found for each equation of state of a different explicit variable, all of which are differentiated to determine an explicit exact differential equation, describing a slight change in state of equilibrium for that explicit differential in terms of the other variables
\3 Exact differentials are contrasted as the differential of an actual function to that of an inexact differential, denoted by $\delta$
\3 By combining two of exact differential equations equations, it follows that if y(x, z) and dz = 0, dx $\neq 0$, then $\frac{\partial x}{\partial y}\frac{\partial y}{\partial x} = 1$
\4 Further, if dz $\neq 0$ and dx = 0, then by extension, $\frac{\partial x}{\partial y}\frac{\partial y}{\partial z}\frac{\partial z}{\partial x} = -1$
\4 By extension of that, $\frac{\partial y}{\partial z}\frac{\partial z}{\partial x} = -\frac{\partial y}{\partial x}$
\3 Thus, by those theorems, $\frac{\partial P}{\partial T} = \frac{\beta}{\kappa}$, such that each partial derivative has a physical representation by the theorems
\4 As a result, a form $dP = \frac{\beta}{\kappa}dT - \frac{1}{\kappa V}dV$ of the state equation is found, which can be integrated for each in terms of the others
\3 In addition, it must be noted that by the chain rule, for some z(w(x, y), x), it is found that $\frac{\partial z}{\partial w}|_x\frac{\partial w}{\partial y}|_x = \frac{\partial z}{\partial y}|_x$
\1 Wires are a one-dimensional system, usually assumed that pressure and volume are constant, such that the coordinates are tension, $\tau$, length, L, and temperature, T,
\2 While it cannot be expressed universally by a single equation, for a constant temperature in the portion of elasticity such that Hooke's law is true, $\tau = -k(L - L_0)$, where $L_0$ is the length at no tension
\3 Since equations of state assume thermodynamic equilibrium, the tension must be balanced out by some external force
\2 For some infinitesimal equilibrium state change for $L = L(T, \tau)$, $dL = \frac{\partial L}{\partial T}(T, \tau)dT + \frac{\partial L}{\partial \tau}(T, \tau)d\tau$
\3 By the limit of the average linear coefficient of expansion, linear expansivity, $\alpha = \frac{1}{L}\frac{\partial L}{\partial T}$
\4 Metals tend to have a positive linear expansivity, but other materials such as rubber may be negative
\4 Linear expansivity is mainly constant for torque change, only changing for temperature, but over a small range, can be considered constant
\4 It is noted that the linear expansivity is derived to be related to the volume expansivity, such that it is $\frac{1}{3}\beta$, assuming the material is isotropic (same in each direction)
\4 Old thermostats were based on this, with two attached materials of different $\alpha$, such that it would curl as they expanded at different rates
\3 By the limit of average Young's modulus (Stress * Y = Strain), isothermal Young's modulus, $Y = \frac{L}{A}\frac{\partial \tau}{\partial L}$
\4 Isothermal Young's modulus is always positive, only changing for temperature, but can be considered constant until some maximum strain ($\frac{\Delta L}{L}$) as well
\1 Surfaces are a 2D system, with different properties from those of the underlying material, acting as a stretched membrane, with each point exerting a force opposite and perpendicular to the opposite side of the surface
\2 Surfaces are thus described by the area, A, the temperature, T, and the surface tension, $\gamma$, which is the force per unit length of the surface, at any point in any direction on the surface, with an equal and opposite force on the surface
\2 While the substance within the surface is relevant, the volume and pressure of it can be ignored, due to remaining fairly constant for small changes in the surface coordinates
\2 For pure liquids at equilibrium with their vapor phase, it is described by the Guggenheim equation of state, $\gamma = \gamma_0(1 - \frac{T}{T_c})^n$, where $T_c$ is critical temperature, $\gamma_0$ is surface tension at standard temperature, generally $20^o C$, and n is a constant, such that $1 \leq c \leq 2$
\3 As a result, as T approaches $T_c$, the surface tension approaches 0, after which there is unable to be any liquid form
\2 For oil films on liquid water, it is described by the equation $(\gamma - \gamma_w)A = aT$, where $\gamma - \gamma_w$ is called the surface pressure, acting as a 2D version of the ideal gas law, and where a is some constant
\2 For soap bubbles, it functions as two surfaces trapping a small amount of liquid between the two, while another common surface is a single layer of gas on a solid surface
\end{outline*}
\section{Chapter 3 - Work}
\subsection{Definitions}
\begin{outline*}
\1 External work is work done on or by the system as a whole, while internal work is work done by part of the system on another part, such as interaction of molecules, only the former able to be done by macroscopic thermodynamics
\2 External work can be viewed as the changing of macroscopic coordinates by the modification of the configuration/position of some external mechanical device or system, such as thermodynamic electrochemical cell systems powering a mechanical device
\3 Thus, assuming the system is in thermodynamic equilibrium and there is no change in the configuration of external systems, no external or internal work is done
\2 Work done on a system is denoted positively, while work done by the system is denoted negatively within thermodynamics and mechanics
\1 Mechanical external work leads to forces and torques within the system, causing external acceleration, internal turbulence, waves, or other forms of internal motion
\2 This can lead to a nonuniform temperature distribution as well as a lack of thermal equilibrium with external systems
\3 The lack of thermal and mechanical equilibrium can lead to spontaneous chemical reactions or gradients forming within the system
\2 As a result, mechanical external work can generally not be described thermodynamically, such that only an infinitesimal applied force, called a quasi-static process, such that it is infinitesimally close to equilibrium during the process
\3 Thus, it can be thought to be described by an equation of state, being an idealization of work on any thermodynamic system, since the time to move to a new equilibrium state is negligible
\3 This is used to be able to ignore minor forces acting on a system, giving the ideas of ideal devices without friction and resistance, and to be able to allow change of state equations to apply
\3 As a result, all processes are reversible, due to no dissipative forces affecting the system, such that energy is not lost from the system
\2 Quasi-static process theory only applies to systems with proper boundaries, such that thermodynamic equilibrium is able to be reached
\2 It is notated that since the unbalanced force is infinitesimal, the equation of state variables and the force are considered to be equal for the system and external systems
\1 While internal work does take place on a microscopic scale within the system, it is not the purview of thermodynamics, but rather within statistical mechanics, such that only external work is viewed on the macroscopic scale
\end{outline*}
\subsection{Hydrostatic System Work}
\begin{outline*}
\1 For some hydrostatic system in a closed cylinder with a piston to lower the height of the system, the pressure exerted by the system on the piston is P, the area of the piston A, such that $F_i = PA$, with some external force, $F_e$, only infinitesimally different
\2 For work done, lowering the volume of the system, $dW = F_edx = PAdx$, since the difference between the forces is infinitesimal
\3 This is under the assumption of a quasi-static system, such that dissipative processes or chemical movement/reactions would not have a large enough effect to render this invalid
\3 On the other hand, lack of mechanical equilibrium, such that there is not a standard pressure definition, would render this invalid
\3 This can be expanded to other hydrostatic systems of different shape, such that the underlying ideas are still valid
\2 As a result, since $Adx = -dV$ by the decreasing volume, such that $dW = -PdV$, often with work measured in liter-atm for convenience, such that 1 L-atm = 100 J
\3 This can be integrated for the work done by the external system, assuming the difference between the calculated work done and the actual is negligible, since the process is quasi-static, pressure is a function of temperature and volume by the equation of state
\3 Once the temperature pathway is found for the particular quasi-static process, it can be integrated purely in terms of V
\1 For the system, a PV diagram can be drawn for the relationship between pressure and volume as the piston moves, traversing the curve in opposite directions for expansion and contraction
\2 As a result, for some closed figure, it moves back to the original equilibrium state through other equilibrium states, called a cycle, such that the net work of the cycle is the difference between the work of the expansion and contraction temperature pathways
\2 It is noted that changing the direction of the cycle would change the sign of the line integral, due to depending on which path direction is traversed for expansion and contraction ($\ointPdV$ for a clockwise curve due to each pathway integral being negative)
\2 Due to the infinitesimal force differences and possible chemical and thermal movement which is ignored, the path does not have to reverse directly, but rather use a different path such that net work is done, as well as the rate of the multiple controlled changing conditions
\3 Thus, the path itself matters, such that for an isobaric process, in which pressure is constant, followed by an isochoric process in which volume is constant, it would be different in area from a direct path, such that it is path dependent
\3 Since the integral is not path-independent, due to infinite possible P(V) functions changing the outcome, $dW = PdV$ is not valid, due to infinite possible values, but rather only valid for a  specific path of P(V), denoted as an inexact differential, $\delta W = PdV$
\4 This means that $\delta W$ is not a small change in the function of work, but rather simply a small amount of work performed by some small change in dV, such that W is defined only in terms of the integral
\4 Work as a result, is not some function of thermodynamic coordinates, or a state function of the system, but rather a path function depending on the specific change in the system
\4 For an isothermal change as a result, the path is the isotherm on the PV plane, such that the work is positional rather than path based, where $\frac {\partial P}{\partial V}$ determines the relative work needed for a change in volume or pressure
\4 For a derivative of an inexact differential, it is noted that the path must still be determined for the differential, even with a designated change, assuming there are multiple paths for the change itself
\2 This is all explained by the fact that work for a non-conservative force is a multi-variable path function, rather than a single variable state function, only state if conditions are imposed
\3 This can also be written as the curl of the force being 0 for conservative forces, as a conservative vector field, which are path-independent on multiple variables as well
\1 For a length of wire system, $\delta W = \tau dL$, while for a surface film (two layers containing some liquid between), $\delta W = \gamma dA$, proved similarly for each system
\2 The equation is derived for a film, but is equally applicable for a single surface, though the area of the surface is doubled for a film, due to repeating
\end{outline*}
\section{Chapter 4 - Heat and the First Law of Thermodynamics}
\subsection{Internal Energy}
\begin{outline*}
\1 Closed systems can have a change of state both by external work done on the system, and by heat transfer from a hotter substance within the system, rather than mechanically
\2 Heat is thus defined as something transferred between a system and surroundings by temperature difference only, such that adiabatic walls are also called heat insulators and diathermic walls are heat conductor
\2 Heat transfer and work can often be viewed as snonymously occurring, such that it depends on what is defined as the system and surroundings to define which is taking effect
\1 Adiabatic work is work done within an adiabatic boundary performed by a coupled system acting on the adiabatically closed system, such that there is no transfer of heat into the system, but rather purely a mechanical energy transfer
\2 This can still result in a change in temperature, based on which parameters are allowed to vary, such as increasing the temperature of water through a resistor
\2 As a result, the first law of thermodynamics states that for any adiabatic change of state, the work done is equal for all adiabatic pathways connecting the states
\3 Similar to a conservative force system, a thermodynamic system can thus be described in terms of a potential energy function, such that $W_{adiabatic, i \to f} = U_f - U_i$, called the internal energy function, where W is the work done on the system
\3 It is noted that not all states can be reached by purely adiabatic work due to entropy
\3 This as a result defines the law of the conservation of energy, and states that for any system, there exists an energy function describing the difference between two states as the energy change
\4 It was noted that it was thought beta decay and the microscopic scale did not follow conservation of energy, due to less calculated after the decay, found to be due to the antineutrino for $\beta^-$, neutrino for $\beta^+$
\4 On an astronomical scale, it was found that dark energy and dark matter was necessary for conservation of energy to exist, but within general relativity in a nonconstant gravitational field, it was found to not apply
\2 Since internal energy is a state function, dU can be expressed as an exact differential in terms of the coordinates necessary to describe a state
\end{outline*}
\subsection{First Law of Thermodynamics}
\begin{outline*}
\1 For nonadiabatic work, to allow conservation of energy to be consistent since $\Delta U \neq W_{adi}$, energy must be transferred by another means, heat transfer
\2 Thus, heat is defined thermodynamically as energy transferred by non-mechanical means into/from a closed system is called heat, such that $\Delta U = Q + W$, where W is adiabatic work and Q is diathermal heat transfer, positive for both if entering the system, as the full first law
\3 This is true for all thermodynamic system changes, rather than just quasi-static process systems, due to not requiring an equation of state
\3 For an infinitesimal process, involving only infinitesimal changes in thermodynamic coordinates, it is written as $dU = \delta Q + \delta W$, or for quasi-static hydrostatic system, $dU = \delta Q + PdV$
\3 For a composite system with multiple components, $\delta W$ can be written as the sum of work done on each system within the overall system
\2 Thus, it can be seen from this the experimental idea that heat is energy transferred by a temperature gradient
\2 Temperature gradient is not the only relevant factor for transferred energy though, such that for an isochoric (constant volume) system due to a fixed diathermic boundary, it is internal energy transferred
\3 On the other hand, isobaric heat (constant pressure) due to a movable boundary, it is enthalpy, rather than internal energy transferred as heat
\3 Thus, heat is either internal energy or enthalpy based on the system conditions being transferred, though it is only applicable as a path quantity, rather than a state, $\delta Q$, similar to work
\2 Further, by this equation, it is evident that for an adiabatic system, the loss of heat from one section of the system is equal to the gain by another part of the system
\1 Heat capacity, more accurately called internal energy capacity, is the ability of a system to store energy, measured by the change in temperature at a point other than state changes
\2 It is measured with heat due to being easier to detect that work, such that $C = \frac{\delta Q}{dT}$
\2 The specific quantity of the heat capacity, or the unit mass quantity, called specific heat capacity, is a property of the material itself, rather than the object
\3 The molar heat capacity, c = $\frac{C}{n}$, where n is the number of moles, such that it is equal to the mass divided by the molecular mass multiplied by the molecules per mole (Avogadro's number, 6.022 * $10^23$)
\2 Heat capacity depends on the process pathway on the P-V graph, such that heat capacity at constant pressure is denoted $C_P(P, T)$, while heat capacity at constant volume is denoted, $C_V(V, T)$
\3 For some small range of variation in T and either P or V, the heat capacity is virtually constant within the range
\2 Heat capacity is measured by a resistance wire around a cylindrical sample, such that if both are the system, the electrical energy is doing work on the system, whereas if just the sample is, the electrical energy does work on the wire, moving as heat into the system
\3 The work required to produce the heat added to the system was originally called the mechanical equivalent of heat, later found just to be an alternate concept of work
\3 This is called a heating coil, used to test the statistical mechanics results of a system
\3 For some current in the wire, $\delta Q = \epsilon I dt$, building the calorimeter, coil, and connected thermometer based on the desired temperature range and material
\4 For low temperature  solids, it is suspended by a thin insulator, using a resistance thermometer placed in a drilled hole in the sample, using thin wires to avoid error
\3 The temperature as a function of time graph is drawn, measuring temperature without activating the circuit as the foreperiod, closing the switch, then after $\Delta t$, opening it and graphing the afterperiod, though often measuring resistance instead of temperature by the thermometer
\4 The foreperiod and afterperiod curves are extended to give the value in the center of the activated region, measuring the difference between as $\Delta T$
\4 Thus, $c_P = \frac{\epsilon I \Delta t}{n \Delta T}$, such that pressure is the constant variable due to atmospheric pressure keeping it constant
\2 Originally, the calorie was the amount of heat to raise the temperature of 1g of water by $1^o C$, though later specified as $14.5^o C$ to $15.5^o C$, measured as 4.186 J
\1 The first law combined with the fact that for U(T, V), $dU = \frac{\partial U}{\partial T}|_VdT + \frac{\partial U}{\partial V}|_TdV$, such that $\frac{\delta Q}{dT} = \frac{\partial U}{\partial T}|_V + (\frac{\partial U}{\partial V}|_T + P)\frac{dV}{dT}$
\2 By extension, if V is constant, the second term cancels out ($C_V = \frac{\delta Q}{dT}|_V = \frac{\partial U}{\partial T}|_V$)
\3 The left side of the equation is able to be used to test a theoretical internal energy function experimentally
\3 It is noted that it takes high amounts of pressure to hold volume constant, such that this is not an easy method of measurement though
\2 On the other hand, if pressure is constant, it becomes $C_P = \frac{\partial U}{\partial T}|_V + (\frac{\partial U}{\partial V}|_T + P)\frac{\partial V}{\partial T}|_P$
\3 This is due to the fact that since $\frac{\delta Q}{dT}$ is path-dependent, due to T relying on two other parameters, some path as a function of P and V must be chosen
\4 The only sections of the right hand side that are modified as a result of the path are P(T, V) and $\frac{dV(T, P)}{dT}$, the former held constant in this form, the latter becoming a partial
\4 While the pathway by which P and T vary for within different parts of the equation rely on separate changes/parameters, it only determines the bounds of the end result, while the pathway chosen determines for both
\3 This is further modified to show that $C_P = C_V + (\frac{\partial U}{\partial V}|_T + P)V\beta$, providing another function of internal energy found theoretically, in terms of measurable state functions
\end{outline*}
\subsection{Heat Flow}
\begin{outline*}
\1 Temperature difference between a system and surroundings creates a lack of thermodynamic equilibrium, with a nonuniform temperature, though quasi-static process of heat flow behaves similarly to that of quasi-static work
\2 For some body of relatively large mass to the main system, it functions as a heat reservoir, such that virtually any amount of heat flow will not measurably change thermodynamic coordinates
\2 Quasi-static heat reservoir contact would produce an isothermal change of state, due to the infinitesimal temperature gradient, adding energy to the system to produce other coordinate changes
\3 On the other hand, a series of infinitesimally different heat reservoirs would produce an overall non-isothermal change of state, able to be performed for a high amount of time, using heat reservoirs such that their temperature remains constant
\3 As a result, $\delta Q = CdT$, where the path of the heat capacity determines the path of Q, allowing it to be integrated
\1 Two parts of an adiabatically closed substance maintained at different temperatures at the edges have a continuous distribution of temperature in between, due to transporting energy between by heat conduction
\2 Thus, it is found experimentally that $\frac{\delta Q}{dt} = -KA\frac{dT}{dx}$, where $\frac{dT}{dx}$ is the temperature gradient, showing the rate of heat flow between the edges
\3 K is the thermal conductivity, such that a large valued object is a thermal conductor, a small value is a thermal insulator
\2 Thermal conductivity varies as a function of temperature, such that for a large gradient, it varies throughout the object, though if small enough, it can be considered constant
\3 For some small temperature difference, $K = \frac{-L}{A\DeltaT}\frac{\delta Q}{dt}$, due to the temperature gradient and length being
\2 It has been found that thermal conductivity of metal is reduced by a large amount by any impurities, as well as changes in pressure of gases below a certain level and composition
\3 Liquids have lower thermal conductivity, increasing as the temperature is raised, while metals tend to remain fairly constant until some minimum temperature where it decreases
\3 Nonmetals are poor conductors, similar to liquids, decreasing as temperature is raised, rising drastically at low temperatures
\3 Gases are especially bad conductors, independent of pressure after some maximum pressure, generally below 1 atm, increasing as the temperature rises
\2 Thermal conductivity in series is equal to the reciprocal of the sum of reciprocals, while in parallel is the sum, such that it is used in double pane glass with air between to lower the conductivity, due to air having a lower value
\1 Convection currents are flows of liquid or gas that absorb heat and mix with another fluid of a lower temperature, releasing the heat, or vice verse
\2 Convection currents by an external force/system is a forced convection, while if it is due to a combined density difference causing the flow along with a temperature difference, it is a natural convection
\2 For fluid in motion against some surface, there is some thin film of stationary fluid between, thinner the stronger the motion is
\3 Heat is thus able to be transferred both by conduction through the film and convection
\2 The convection coefficient h, is defined such that $\frac{\delta Q}{dt} = hA(T_{wall} - T_{fluid})$, where h takes into account both the surface conduction through the stationary fluid and the convection
\3 The coefficient depends on if the wall is flat/curved, vertical/horizontal, gas/liquid, density, viscosity, specific heat, and thermal conductivity of the fluid, laminar (low speed)/turbulent flow (high speed), and if evaporation, condensation, or formation of scale happens
\1 Heat can be transmitted through empty space through radiation, most relevantly thermal (IR and visible), which is transmitted due to the high temperature of the origin solid/liquid (gases behave separately)
\2 At higher temperatures, greater frequencies of visible light are emitted, and more total energy is emitted
\2 The radiant exitance, R, of the body is the radiant power per infinitesimal area, depending on the nature of the surface of the body and the temperature just as absorptivity and emissivity does
\2 Isotropic radiation is thermal radiation incident on a body equally from all directions, where the fraction of incident radiant power absorbed is absorptivity
\3 Total emissivity, $\epsilon$, is the fraction of power incident on the body that is emitted as thermal radiation, such that for thermal equilibrium, it is equal to absorptivity, depending on temperature and the material itself
\3 Ideal substances are those which absorbing and emit all radiation falling on it when at thermal equilibrium, called a blackbody, independent of the substance made from, approximated by a cavity with a small entrance hole and uniform high-temperature opaque walls
\4 This is due to the radiation reflecting around the cavity until fully absorbed, such that it is almost entirely absorbed and is isotropic
\2 Irradiance is the radiant power incident per infinitesimal area on the surface of the cavity, corresponding to radiant exitance
\3 Thus, for a blackbody with irradiance of H in thermal equilibrium with the cavity is placed within, such that radiant power absorbed per unit area of the blackbody is H$\epsilon$ = H
\3 Thus, radiant exitance of the blackbody, R, is equal to H, since the radiant power absorbed is equal to that emitted per unit area for the body, projected back to and from the cavity
\2 It has been found that the amount of radiation within the cavity is a function of the temperature produced by the thermal equilibrium and independent of the materials, detected by allowing a small amount of radiation to escape for measurement
\2 Radiant exitance of a non-blackbody depends on the surface material itself, as well as temperature, such that $R = \epsilon H = \epsilon R_{bb}$, called Kirchhoff's Law, used to measure thermal radiation emissivity of materials at some temperature
\2 There will only be a difference in the heat absorbed and emitted if there is a temperature difference between a body and its surroundings as a result
\3 Thus, $\frac{\delta Q}{dt} = A(\alpha(T) H - R) = A(\epsilon(T) R_bb(T_W) - \epsilon(T) R_bb(T) = A\epsilon(T)(R_bb(T_w) - R_bb(T))$ for some body in a cavity, such that it is isotropic, where the cavity is large enough for the change in temperature to be negligible, T is the temperature of the body, $T_w$ of the cavity, and A is surface area
\2 It was derived later thermodynamically that $R_bb(T) = \sigma T^4$, where $\sigma$ is the Stefan-Boltzmann constant, $5.67051 * 10^-8 \frac{W}{m^2 * K^4}$
\3 It follows that $\frac{\delta Q}{dt} = A\epsilon\sigma(T_W^4 - T^4)$
\3 The Stefan-Boltzmann constant can be determined by the nonequilibrium method, placing a shielded silver disk in a blackened copper hemisphere until there is steam, measuring the temperature $T_W$, uncovering the disk and measuring the temperature over time
\4 $\frac{dT}{dt}$ is then found, such that since pressure is constant, $\delta Q = C_P dT$ for the silver disk, where as a result, $\sigma = \frac{C_P}{A(T_W^4 - T^4)}\frac{dT}{dt}$
\3 It can also be found by the equilibrium method, placing a hollow blackened sphere suspended in a vessel of temperature $T_W$, adding electricity to the sphere at a constant power
\4 This is done until it reaches equilibrium where radiation emission is equal to the power supplied, such that assuming it acts as a blackbody, $\epsilon I = A\sigma(T^4 - T_W^4)$
\end{outline*}
\section{Chapter 5 - Ideal Gases}
\subsection{Ideal Gas Laws}
\begin{outline*}
\1 It has been found experimentally that $\frac{PV}{n} = Pv = A(1 + BP + CP^2 \dots)$, such that it is purely a function of pressure, called the virial expansion, where A, B, C are virial coefficients
\2 Until 40 atm, it is generally linear, such that only pressure is necessary, requiring more as pressure rises, though as pressure approaches 0, it is equal to A, only dependent on the temperature, rather than the gas itself 
\3 Virial coefficients are generally small, except at extremely low temperatures
\2 Since the ideal gas temperature assumes constant volume and the same amount of moles of gas for each measurement, $T = 273.16 K lim \frac{Pv}{Pv_{TP}}$
\3 As a result, R, the molar gas constant, is defined as $\frac{lim(Pv)_TP}{273.16 K}$, equal to 2.27102 kJ/mol*K, difficult to measure due to absorption of gas on walls
\3 Thus, the ideal gas law is derived such that lim(PV) = nRT and A = RT, such that $\frac{Pv}{RT} = 1 + BP + CP^2 + \dots$
\1 Adiabatic free/Joule expansion is the expansion in volume of a gas as there becomes a larger space to move into, such that no Q or W is done, and internal energy is constant
\2 Free expansion is noted not to be quasi-static, due to the fast modification from one state to another forcing it through nonequilibrium states
\2 The Joule coefficient, $\frac{\partial T}{\partial V}|_U$, testing with two diathermic vessels in a adiabatic water bath, measuring the change in gas temperature, found that the heat capacity of water was too high, and thus the coefficient was too difficult to measure directly
\3 It was tested with water due to too much difficulty in making the temperature of gas
\2 Since for U(T, V), $dU = \frac{\partial U}{\partial T}|_VdT + \frac{\partial U}{\partial V}|_TdV$, such that if no temperature change or internal energy change takes place, $dT = dU = 0$, then $\frac{\partial U}{\partial V}|_T = 0$
\3 Similarly, for U(T, P), $\frac{\partial U}{\partial P}|_T = 0$ in this case as well
\4 Thus, as the gas approaches an ideal gas, it is equal to 0 as pressure approaches 0, such that there is no change in temperature in that case
\3 As a result, if free expansion causes no change in temperature, internal energy of the gas is a function of temperature only for free expansion, used to measure this fact
\3 It is also done by measuring the $\frac{\partial u}{\partial P}|_T$, where u is the molar internal energy, measured by a container with n moles of gas at some pressure, connected to the atmosphere by a long coil, surrounding the container with liquid at atmospheric temperature
\4 The valve is opened to allow small amounts of gas to flow out, using a heating coil to keep all temperatures constant with the atmosphere as the gas expands
\4 As Q heat is absorbed by the gas from the coil, the work done by the gas is $W = -P_0(nv_0 - V)$, where $P_0$ is atmospheric pressure, V is container volume, and $v_0$ is molar volume at atmospheric temperature and pressure
\4 It is noted that $nv_0$ must be larger than the volume of the container, such that it expands when provided to the air
\4 $\Delta u(P, T)|_T = \frac{Q + W}{n}$, after accounting for contraction of container walls, putting it in terms of change in pressure instead of volume, plotting for different pressures
\3 On the other hand, it was noted that the experiment did not reach the ideal gas value of 0, and had the same issue of larger heat capacity of water, requiring precision in water temperature
\1 Ideal gases are thus those which have low enough pressure for the ideal gas law, and as well that internal energy is a function of temperature, rather than temperature and pressure
\2 Gases can be assumed to be ideal with increasing error as the pressure increases, though with only a few percent error below 2 atm
\2 For some quasi-static process, the ideal gas law can be written as PdV + VdP =  nRdT by the product rule, and as a result of the second part, it can be stated that $\frac{dU}{dT}|_V = C_V$, such that $\delta Q = C_VdT + PdV$
\3 By combination of these, it is found that $\frac{\delta Q}{dT} = C_V + nR - V\frac{dP}{dT}$, such that if pressure is set constant, $C_P = C_V + nR$ for an ideal gas
\3 Thus, $C_V, C_P$ are functions of temperature only for ideal gases, and $C_P \geq C_V$, due to the gas expanding work to expand against the equal, external pressure when pressure is constant, whereas for constant volume, there is no work done (PdV = W)
\3 This is combined with the first law of thermodynamics to find that internal energy of an ideal gas is a function purely of temperature
\2 It is also derived following that $\delta Q = C_PdT - VdP$ for an ideal gas, giving another form of the 1st law for an ideal gas
\end{outline*}
\subsection{Heat Capacity Measurement}
\begin{outline*}
\1 Heat capacities are measured by a gas in a thin-walled steel container, surrounded by a wire to add heat, measuring the rise in temperature at constant volume to get $C_V$, while $C_P$ is through a calorimeter allowing volume to vary rather than pressure
\2 The inlet/initial and outlet/final temperatures are measured, found that for all ideal gases, $C_V(T) < C_P(T)$, $c_P - c_V = R$, and $\frac{C_P}{C_V} = \gamma$
\2 Monatomic gases, such as noble gases and most metallic vapors, $C_V = c \approx \frac{3R}{2}$, and $C_P = c' \approx \frac{5R}{2}$
\2 Permanent diatomic gases, such as $H_2, O_2, N_2, NO, CO, F_2, Cl_2$, have constant values of $C_V \approx \frac{5}{2}R, C_P \appx  \frac{7}{2}R$
\3 On the other hand though, at low temperatures, $H_2$ uniquely acts as a monatomic gas
\3 For all temperatures, it may be written as $\frac{C_P}{R} = \frac{7}{2} + f(T)$, where $f(T) = (\frac{b}{T})^2 \frac{e^{\frac{b}{T}}}{(e^{\frac{b}{T}} - 1)^2}$ generally for some constant b, generally too difficult to calculate, such that it is measured experimentally
\2 Polyatomic gases and chemically active gases are those with values for $C_P, C_V$ varying with temperature
\1 For some quasi-static ideal gas adiabatic process, $\delta Q = 0$, such that since $\delta Q = C_VdT + PdV = C_PdT - VdP$, then $\frac{dP}{P} = -\gamma \frac{dV}{V}$, such that for some small temperature change, $\gamma$ is constant, such that $PV^\gamma = c$
\2 As a result, for some adiabatic process on a P-V graph, $\frac{\partial P}{\partial V}|_Q = -\gamma \frac{P}{V}$, such that since for a quasi-static isothermal process, $\frac{\partial P}{\partial V}|_T = \frac{P}{V}$ by the ideal gas law, the curve for the former is steeper than the latter
\1 Ruchhardt measured $\gamma$ by placing gas in a jar of some volume, with a glass tube attached with a metal ball and piston within, such that $P_{eq} = P_{atm} + \frac{mg}{A}$, by $F = PA$  where A is the area of the tube, assuming no friction
\2 With displacement, it will oscillate, such that $dV = Ady$ and $dP = \frac{dF}{A}$, acting as a restoring force, causing oscillation, moving quickly enough to prevent heat transfer, such that it is adiabatic and minor enough to be quasi-static
\2 As a result, by the product rule and the ideal gas quasi-static adiabatic formula, $dF = -\frac{\gamma PA^2}{V}dy$, such that it is simple harmonic motion, where $\tau = 2\pi \sqrt{\frac{m}{-\frac{F}{y}}} = 2\pi \sqrt{\frac{mV}{\gamma PA^2}}$, measuring the period to get $\gamma$
\2 A majority of the error originates from the assumption of the lack of friction within the glass tube
\3 Error from assumptions was later corrected, using a steel piston in the middle of the gas container, dividing it in two and displacing it, using an electromagnet to hold the piston in place against weight, to prevent friction, using a current to move it slightly
\3 It also used the real equation of state, measuring the frequency by AC and amplitude by microscope, with a non-ideal non-adiabatic equation to create more precision
\2 For some small displacement of the piston in the divided container at constant velocity, $\vec{v}_0$, the pressure wave front due to the concentrated density near moves at velocity $\vec{v}$, such that it moves faster than the piston
\3 Thus, the compressed portion of the gas at some time, t, had an original volume of V, mass of $m = \rho_0 V = \rho_0 A \vec{v} t$
\3 Since the portion of the original volume that was compressed increases as the compressed wave front moves faster than the piston, such that the pressure lessens as it moves, but the compressed mass increases toward the total mass ($\frac{dm}{dt} = \rho A\vec{v}$)
\3 On the other hand, the additional velocity is due to the equalization of pressure, added to the velocity of the piston, such that the change in momentum is purely from the change in mass ($\frac{dp}{dt} = \frac{dm}{dt} \vec{v}_0 = \rho A \vec{v} \vec{v}_0$
\3 The pressure from beyond the wave front is the normal pressure of the system, while the pressure from the piston is that added to the pressure exerted by the piston
\3 It follows that $F = A\Delta P = \rho A \vec{v} \vec{v}_0$, and $\frac{\Delta V}{V} = \frac{A\vec{v}_0 t}{A \vec{v} t}$, able to be related such that $\vec{v}^2 = \frac{-1}{\frac{\rho}{V}\frac{\Delta V}{\Delta P}}$
\4 $\frac{\Delta V}{V \Delta P}$ in this case is reversible adiabatic compressibility ($\kappa_S$), rather than isothermal, proven by the above system, measuring heat transfer from the middle of the compressed and rarefacted/uncompressed portions, $\frac{\lambda}{2}$
\4 It is found that the heat traveling by conduction in the time for the wave to travel $\frac{\lambda}{2}$, Q = $KA\frac{\Delta T \lambda}{\frac{\lambda}{2}2\vec{v}} = KA\frac{\Delta T}{\vec{v}}$
\4 Since the heat to raise the temperature by $\Delta T$ is Q = $\rho A \frac{\lambda}{2}c_V\Delta T$, it is only adiabatic if the heat transferred in that amount of time is insignificant in comparison to correcting the temperature difference, such that it is corrected as one system, rather than by heat
\4 Thus, it is found that this is true only if $\lambda >> \frac{2K}{\vec{v}\rho c_V}$, such that the distance the wave is traveling must be drastically smaller than realistic, such that it is adiabatic
\3 As a result, it is found that $\kappa_S = \frac{1}{\gamma P}$, such that $\vec{v}^2 = \frac{\gamma P v}{M} = \frac{\gamma RT}{M}$, where v is the molar volume, M is the molar mass
\3 This can be used to measure the speed of sound in a gas, set with a movable piston with powder inside, vibrating the piston after placing it in such a position to create a standing wave
\4 After the vibration begins, powder piles at the nodes, allowing measurement of half a wavelength between, multiplied by the frequency to find the velocity
\4 This can be made more accurate by an acoustic interferometer instead of a tube, with a piezoelectric crystal, and a receiver, varying the frequency with a constant distance, measuring the resonance to find the odes, accounting for viscosity, heat conduction, and boundary layer absorption
\4 The speed of sound, $\vec{v}(P)$ can thus be used to find R, as pressure approaches 0 such that the ideal gas law holds
\end{outline*}
\subsection{Kinetic Theory of Ideal Gases}
\begin{outline*}
\1 The theory was determined by Waterston and Kronig in 1859, finding that temperature is a function of particle motion for a monatomic gas (due to solid spherical atoms), assuming there is some high number of identical, chemically inert particles
\2 It is stated that the number of particles per mole is $6.0221 * 10^23$, Avogadro's number, where the atomic mass is used to calculate the total mass, with a mole as $22.4 * 10^3 mL$ for an ideal gas
\2 Ideal gases are assumed to be in perpetual random motion with no preferred direction of motion, with a inter-atomic distance far larger than the atomic radius, with no forces of attraction between collisions, with velocities ranging from 0 to the speed of light
\3 This can be integrated to infinity without error, due to such low density at higher velocities, and is assumed that the distribution at any particular velocity band is approximately constant
\2 The atoms are uniformly distributed ($\frac{N}{V}$ is constant), such that $dN = \frac{N}{V}dV$, when fulfilling thermodynamic macroscopic qualifications
\2 Atomic collisions are assumed to be elastic against a smooth wall of the container, where only the perpendicular velocity with respect to the wall is changed, negated as a result without different speed
\1 For some velocity vector from the origin, $\vec{v}$, to some elementary area, dA, on the surface of the sphere, such that for the sphere around the origin, $dA = (rd\theta)(rsin\theta d\phi)$
\2 The solid angle of a circle, as a 3D extension of angle, $d\Omega = \frac{dA}{r^2} = sin \theta d\theta d\phi$, the maximum being that of the entire sphere, $4\pi$ steradians (sr), such that it is the surface area for a circle of radius 1
\1 It is stated that there is no preferred atomic direction, it is written that $dN|_{\vec{v}, \theta, \phi} = dN|_\vec{v} \frac{d\Omega}{4\pi}$, where N is the total number of atoms, such that the atoms going in any direction in a specific solid angle is equal to the fraction of the total atoms going in that direction
\1 It is stated that atoms have no preferred location, such that for some portion of the wall, dA, some cylinder can be constructed of length $\vec{v}dt$, such that it is a short enough time that atoms won't collide, moving in the direction of $\vec{v}$
\2 Due to the cylinder being non perpendicular to the base, dV = $\vec{v}dt cos\theta dA$, such that it is stated that $dC|_{\vec{v}, \theta, dV, \phi} = dN|_{\vec{v}, \theta, \phi}\frac{dV}{V}$, such that the total atoms in that location moving in that direction divided by the total amount moving in that direction is equal to the percentage of volume
\1 Since collisions are perfectly elastic, $\Delta p$ per collision = $2m\vec{v}cos\theta$, such that per dt, the total change in momentum for collisions against dA = $m\vec{v}^2\frac{dN|_\vec{v}}{V}(\frac{1}{2\pi}\int^{2\pi}_0 d\phi \int^{\frac{\pi}{2}}_0 cos^2\theta sin\theta d\theta$
\2 Since the change in momentum due to collisions against some area is the definition of pressure, this is equal to $dP|_\vec{v} = \frac{m}{3V}dN|_\vec{v}$, integrated from 0 to $\infty$ by the earlier assumption
\2 By definition, the average square of atomic velocity, $<\vec{v}^2> = \frac{1}{N}\int \vec{v}^2 dN|_w$, such that $PV = \frac{Nm<\vec{v}^2>}{3}$
\2 Combining with the ideal gas law, produces the fact that $T = \frac{2N}{3nR}\frac{1}{2}m<\vec{v}^2> = \frac{2*<KE>}{3nR}$, where since for monatomic ideal gas atoms there are no rotational or vibrational kinetic energies, and due to lack of effect on other atoms, no potential, $<K>N = \sum KE$ = U
\3 Thus, it is shown that for a monatomic ideal gas, the internal energy and average kinetic energy are proportional purely to temperature
\3 $C_V$ and $C_P$ can then be calculated based on this function of internal energy, such that $C_V = \frac{3}{2}nR, C_P = \frac{5}{2}nR$, such that it is independent of temperature
\3 The ratio $k = 1.3807 * 10^{-23} J/K = \frac{R}{N_A}$, where $N_A$ is Avogadro's number, is the Boltzmann constant, providing the kinetic theory ideal gas law, PV = NkT
\1 The assumptions of non-interacting point-masses create error for real gases, such that van der Waals equation of state corrects for it, with a accounting for cohesive forces, b for volume occupied by being non-point masses
\end{outline*}
\section{Chapter 6 - The Second Law of Thermodynamics}
\begin{outline*}
\1 Work done in contact in a system to produce a higher temperature, but in contact with a heat reservoir of lower temperature, produces an unlimited amount of heat while work is done, without changing the state of the system itself, acting as a work engine
\2 Since the temperature of the system returns to the original due to losing heat to the sink, the work done is equal to the heat transmitted, such that there is 100\% efficiency
\1 The corresponding method to produce unlimited heat without a change of state is a heat engine, made up of a high temperature reservoir and a low temperature one, each connected to the system performing the work
\2 It then undergoes a cycle, or series of processes, to return it to the original state, measuring thermal efficiency, $\eta = \frac{|W|}{|Q_H|}$, which by the first law, $\eta = 1 - \frac{|Q_L}{Q_H}$, such that it is only 100 if no heat goes to the lower reservoir
\2 Heat engines are divided into internal-combustion, converting the fuel to high pressure and temperature by compressing, and external-combustion, using high temperature surroundings to convert the fuel to high pressure and temperature
\1 The gasoline engine is made up of a piston connected to a rotating mechanism, rotating in one direction as it moves, using a six-process, four-stroke (piston-moving) cycle 
\2 The intake stroke moves the piston outward, such that the volume moves from 0, pulling gasoline vapor and air from the open intake valve, due to being depressurized compared with atmospheric outside, rotating the mechanism manually
\2 The compression stroke moves the piston inward after closing the intake valve, increasing the pressure and temperature by decreasing the volume, after which combustion by a spark plug produces extremely high temperature and pressure (combustion), rotating manually
\2 The power stroke is caused by the combustion products expanding, pushing the piston, and rotates the mechanism by the pressure increase, after which the exhaust valve is opened to lower temperature and pressure and release some products (exhaust)
\2 The exhaust stroke then lowers the volume back to 0, pushing the remaining products back through the exhaust valve, rotating the mechanism manually
\1 The Otto cycle is an idealized gasoline engine, in which there is no error introduced by conduction of heat through the engine, friction, turbulence, and chemical reaction between the vapor and air
\2 This engine is assumed to have air acting as an ideal gas, with constant heat capacities, and purely reversible quasi-static processes
\2 The intake stroke is an isobaric atmospheric pressure ($P_0$) process, moving the volume from 0 to V, such that it is described by the ideal gas law, $P_0V = nRT_1$, where $T_1$ is the temperature outside the engine, describing the state after intake
\2 The compression stroke is an adiabatic process, such that it moves from $V_1, T_1 \to V_2, T_2$, described by $T_1V_1^{\gamma - 1} = T_2V_2^{\gamma - 1}$, where $V_1 > V_2$
\2 The combustion is an isochoric increase of temperature and pressure due to external heat added from the hot reservoir, assumed to be a series of slightly different hot reservoirs, such that the change in temperature is quasi-static
\3 This gives the equation that $|Q_H| = \int^{T_3}_{T_2}C_VdT = C_V(T_3 - T_2)$
\2 The power stroke is an adiabatic process from some temperature, $T_3$ after combustion to $T_4$, at a lower pressure and temperature, such that $T_3V_2^{\gamma - 1} = T_4V_1^{\gamma - 1}$, moving back to the volume before the compression
\2 The exhaust is the isochoric decrease in temperature and pressure due to external heat added to the sequence of cold reservoirs, where $|Q_L| = C_V(T_4 - T_1)$
\2 The exhaust stroke is an isobaric process at atmospheric pressure again, the reversal of the intake stroke
\2 It is found as a result of these equations that $\eta = 1 - \frac{T_1}{T_2} = \frac{T_4}{T_3}$, such that it is based on the temperatures before and after each adiabatic process
\3 In addition, $\eta = 1 - \frac{1}{(\frac{V_1}{V_2})^{\gamma - 1}}$, where $\frac{V_1}{V_2}$ is the compression ratio, such that complete efficiency would require no volume/gas
\4 For real engines, it is found that a compression ratio greater than 10 would cause spontaneous combustion, interfering in the timing of the engine, though used in diesel
\3 Otto gasoline engines are assumed to go from 300 K to 580 K, such that there is 48\% efficiency, though real gasoline engines tend to be about 25\%
\1 Diesel engines are internal-combustion engines as well, taking in only air during intake, spraying oil after compression, using the temperature of the air to trigger combustion, acting similarly to a gas engine otherwise
\2 The oil spraying rate is adjusted such that the combustion is slow enough that it is isobaric as the piston moves out to increase the volume for the power stroke
\2 Since air is used rather than gasoline, the exhaust stroke and intake strokes can be skipped,  such that the power of the engine is doubled
\3 This involves blowing fresh air into the cylinder to remove the remaining combustion products that didn't move out on their own
\2 An idealized diesel cycle, removing the same issues present during a real gasoline engine, acting similar to the Otto cycle, such that the only difference is during the combustion
\3 Since the combustion is isobaric instead of isochoric as heat enters from the hot reservoir, such that $|Q_H| = \int^{T_3}_{T_2} C_P dT = C_P(T_3 - T_2)$, such that $\eta = 1 - \frac{1}{\gamma}\frac{T_4 - T_1}{T_3 - T_2} = 1 - \frac{1}{\gamma}\frac{r^\gamma_E - 1}{r_E - 1}\frac{T_1}{T_2}$, where $r_E$ is the expansion volume ratio, $\frac{V_f}{V_i}$
\3 The expected ideal value is approximately 55\%, while the realistic value is approximately 32\%
\1 Steam engines are an external-combustion engine, used largely in nuclear powered maritime vehicles and submarines, functioning by a constant mass of water, using the feed-water pump to pump water up into the boiler, at which point a nearby hot reservoir heats and boils it
\2 Since the boiler has much higher pressure then the remainder, the pump is necessary, compressing it as it rises with only small changes in temperature and volume
\3 The boiling takes place isobarically, raising the temperature, after which it is allowed to flow into a cylinder, expending adiabatically against a piston to perform work
\3 Eventually, the temperature and pressure lower until that of the condenser as volume increases, after which it is allowed to flow into there near the cold sink, condensing into the original water temperature and pressure, below atmospheric levels
\2 Error is introduced to the engine by turbulence from the pressure differences causing flow, friction, or conduction of heat through the walls
\2 The idealized steam engine cycle is the Rankine cycle, starting from liquid water in the condenser, using the pump to compress and move it to the boiler adiabatically
\3 After, the water is isobarically superheated and vaporized to above boiling point and vapor saturation at that pressure, expanding in the boiler
\3 After, it adiabatically expands from superheated steam to saturated/equilibrium steam, performing work on the piston, and finally the isobaric and isothermal condensation of steam back into water by loss of heat
\2 Since $Q_L$ is nonzero, due to needing to take heat to condense the water back into liquid, the efficiency cannot be 100\%, historically far lower than diesel, but currently actually about 35\%
\1 Stirling engines are external combustion engines, invented to be less likely to explode than steam engines, used before the invention of internal combustion engines for low horsepower machines
\2 It has the overall advantage of being able to use any form of heat source, operates quietly, and does not produce any toxic byproducts, currently used for artificial heart power
\2 Ideal Stirling engines have two pistons on each side of a shaft with a fixed mass of helium in between, the left/expansion piston side in contact with high temperature reservoir, the right/compression piston with the low reservoir/atmosphere
\3 In the middle is the regenerator made of fine wire screens, allowing heat in and out into an internal reservoir by a fixed quantity reversibly
\2 The Stirling cycle is made up of the heat loss stage, in which the left piston fully compressed, the right piston compressing halfway isothermally, triggering heat loss through the cold sink, performing work to compress
\3 After, the left piston expands, and the right fully compresses isochorically forced through the regenerator to the other side, raising the temperature and pressure without work or volume change through heat provided
\3 After, the right piston remains static, as the left piston is expands due to the pressure, to allow such that the volume increases, absorbing heat such that it is isothermal, performing work
\3 Finally, the left piston compresses fully as the right piston expands fully, giving the same heat absorbed by the regenerator as moving back through, isochorically
\2 Error is introduced into the system by non-ideal gas, leakage of gas, heat transfer through walls, heat conducted from the regenerator to the device, or friction
\3 Realistic thermal efficiency of a Stirling engine is approximately 40\%
\2 The Ringbom Stirling engine was a modification using one piston instead of two, with a displacer/regenerator moving between the cylinder and piston
\1 The Second Law of Thermodynamics Kelvin-Planck statement states that it is impossible to construct a cyclical heat engine that will do no effect other than extraction of heat from a reservoir and performing equal work
\2 This is based on the fundamental cycles of a heat engine, relying on a process which takes heat from an high temperature reservoir, and a process which rejects heat to a lower temperature reservoir
\3 This can be restated as that it is impossible for a hot object to utilize mechanical work from internal energy of matter, cooling below its surrounding objects (100\% thermal efficiency)
\2 Theoretical machines which create energy, violating the first law, are perpetual motion machines of the first kind
\2 Those which use internal energy of a single heat reservoir only, converting heat into work alone, thus violating the second law, are perpetual motion machine of the second kind
\1 Refrigerators are the opposite of a heat engine, taking some small amount of heat from a cold reservoir, returning a higher amount by work done on the system, the working substance of the system called refrigerant
\2 It must be able to be cyclical, such that there is no change of state from a full cycle, able to convert an unlimited amount of work into heat
\2 The Clausius statement of the Second Law states that it is impossible to construct a refrigerator that produces no effect other than transferring heat from a low temperature reservoir to high temperature, such that there is no work needed
\3 This is proven to be the Second Law, by the fact that combining an engine and a refrigerator, the latter violating this, would allow violation of the Kelvin statement
\3 Further, for an engine that violates the second law, feeding the work into the refrigerator, the refrigerator would thus violate the Clausius statement
\1 Reversible processes are those which can be performed such that the system returns to its original state without any external changes
\2 Any isothermal process involving a dissipation of heat, to be reversible, would require the complete conversion of heat from a reservoir into work, thus impossible
\3 Any adiabatic process increasing the internal energy of a system from work, resulting in an increase in temperature, to be reversible, would require removing all added internal energy as heat, and converting it fully into work, also impossible
\3 These dissipative processes are considered to have external mechanical irreversibility, present in all mechanical devices, such that a machine which violates it is a perpetual motion machine of the third kind
\2 Internal mechanical irreversibility processes are those which convert internal energy into mechanical energy, then back to internal energy without external heat or work, unable to be reversed without adding energy to the system, such as free expansion
\3 This can be visualized as a process, which to be reversed, would require work, but also which all added energy would have to be removed fully as heat and fed back to the source of the work
\3 This can be viewed microscopically as being due to mechanical instability, causing the initial change, followed by dissipation causing the second change, making it irreversible
\2 External and internal thermal irreversibility are those involving a transfer of heat to/from a reservoir, or within a body, by a finite temperature difference, impossible to reverse by the Clausius statement
\2 Chemical irreversibility are those involving spontaneous chemical reactions, mixing of chemical substances, changes of phase, and transport of matter between contacting phases (solutions, osmosis)
\3 Mixing of substances can be viewed as the free expansion of two substances, such that they are shown to be irreversible, while others are proved by chemical thermodynamics
\2 By extension, it is shown that all natural processes are irreversible, featuring characteristics of lack of thermodynamic equilibrium during (not quasi-static) and dissipative effects
\3 Since heat reservoirs are defined as a body such that any amount of heat transfer would produce a negligible effect, the transfer of heat can be considered reversible for a heat reservoir
\4 This is due to acting quasi-statically and producing little enough work relative to internal energy of the body for dissipative effects to be negligible
\3 On the other hand, quasi-static processes do not imply lack of dissipation, in cases of heat being created from work, making it non-dissipative
\3 On the other hand, since none of the ideal cycles rely on this conversion of work into heat, it is found that they are reversible, assuming no friction or other dissipative processes
\end{outline*}
\section{Chapter 7 - Carnot Cycle and Thermodynamic Temperature Scale}
\begin{outline*}
\1 The Carnot cycle is a theoretical ideal, reversible heat engine, with the working substance beginning in thermal equilibrium with the low temperature, $T_L$
\2 Reversible adiabatic process is then preformed, such that the temperature rises to $T_H$, after which a reversible isothermal process is performed such that $|Q_H|$ is absorbed
\2 After which a reversible adiabatic process is performed, such that the temperature drops back to the low temperature, and a reversible isothermal process is performed, rejecting $|Q_L|$ to the cold reservoir
\2 This is distinct from a real heat engine, in that since it is isothermal rather than isochoric or isobaric, it doesn't require a series of reservoirs to be quasi-static, but rather a single
\1 For a gas, this is drawn simply on a P-V graph, while for a reversible state change for added heat, which is isobaric and isothermal, with reversible adiabatic temperature change, it is written differently
\2 In this case, it is noted that the change in volume for temperature change is drastically larger for gases than liquids, but the change in pressure is approximately the same
\3 It is found that for an Carnot engine, any system can be graphed on some two coordinate plane, such that it is signified in a diagram by some rectangle, with R written inside to show reversibility
\2 The work done by the heat engine as a result is simply modified by changing the length of the isothermal process, and as a result, the amount of heat allowed to enter
\1 Any reversible heat engine only requiring two reservoirs for operation and reversibility must be Carnot engines, unlike other cycles, which require a series of reservoirs to be reversible, providing a definition for Carnot engines
\2 Carnot engines used in the reverse of the cycle are called Carnot refrigerators, distinct from other cycles, due to perfectly reversing the process formed by the engine, impossible due to dissipative effects
\2 Thus, Carnot's Theorem states that no heat engine operating between two given reservoirs can be more efficient than a Carnot engine operating between the same two reservoirs
\3 This is proven by joining a Carnot refrigerator with an irreversible heat engine, such that by contradiction of the 2nd Law, it must be more efficient
\3 As a result of this, the Carnot cycle does not depend on the actual substances which make up the Carnot engine, but rather have a fixed efficiency between two reservoirs
\3 This provides a bound for the efficiency of a heat engine, as the maximum due to being able to absorb all heat at the high temperature, and release all heat at the low temperature, rather than by a series of reservoirs
\1 The thermodynamic absolute temperature scale is the ideal, independent of the working substance, such as a constant-volume ideal gas thermometer, thus since a Carnot engine is independent, it makes a thermodynamic absolute scale
\2 Efficiency of a Carnot engine is $\mu_R = 1 - \frac{|Q_L|}{|Q_H|}$, as some function of the two temperatures, such that $f(T_H, T_L) = \frac{1}{1 - \mu_R} = \frac{|Q_H}{|Q_L|}$
\2 Since two Carnot engines, passing heat through some arbitrary heat sink, is equivalent to one bypassing it in terms of the initial and final heat, and total work, since the middle heat sink cancels out, $\frac{|Q_H|}{|Q_L|} = \frac{\phi(T_H)}{\phi(T_L)}$
\2 As a result, $\phi(T)$ is defined as the thermodynamic temperature function, independent of any substance, not even relying on gas properties like an ideal gas system, as the ratio of the heats
\3 For a logarithmic $\phi(T)$, varying from $-\infty (Q = 0)$ to $\infty (Q \to \infty)$, advantageous due to moving absolute zero far from the freezing point of water
\4 The triple point of water was set at $0^o$, increasing linearly for each factor of 10, as a more symmetrical numerical system, denoted $^oL$, called the logarithmic thermodynamic scale
\3 On the other hand, the common scale remains Kelvin or Rankine, applied to thermodynamic measurement systems
\2 For an absolute linear scale, $\phi(T)$ can just be written as T on the other hand, such that the triple point can be set as before, such that $T = 273.16 K \frac{|Q|}{|Q_{TP}|}$, such that heat is a thermometric property for a Carnot thermodynamic scale
\end{outline*}
\section{Chapter 8 - Entropy}
\subsection{Definition of Entropy}
\begin{outline*}
\1 For some reversible process plotted on a smooth curve on a two-coordinate graph, versus a piecewise reversible continuous process, drawn made up of isothermal and adiabatic processes, with the same amount of work done from the same initial and final states
\2 Since the change in internal energy is equal, and the work done is equal, the heat transfer is equal for the process, able to ignore the adiabatic portions
\2 Since adiabatic and isothermal lines are unable to intersect due to being parallel relatively, any number of isothermal and adiabatic alternating processes can be written for the same process
\2 For two isothermal reactions between the same adiabatic curves, it forms a Carnot cycle, such that $\frac{|Q_1|}{T_1} = \frac{|Q_2|}{T_2}$, going in opposite directions, such that the heat signs are opposite
\3 Thus, for reversible Carnot cycles, these reactions can be summed, such that $\sum_n \frac{Q_n}{T_n} = \oint_R \frac{\delta Q}{T} = 0$, such that any reversible cycle can be thought of as a series of Carnot cycles such that this equation is true, called the Claudius Theorem
\1 By Claudius's Theorem, for any two reversible paths making a closed cycle, $\int_{R_1}_i^f \frac{\delta Q}{T} + \int_{R_2}_f^i \frac{\delta Q}{T} = 0$
\2 Since each path is reversible, $\int_{R_2}_f^i \frac{\delta Q}{T} = -\int_{R_2}_i^f \frac{\delta Q}{T}$, such that the integral of $\frac{\delta Q}{T}$ for each reversible path with the same initial and final state is equal
\2 Thus, the state function of this integral is called entropy, S, such that $\delta Q_R = TdS$, allowing an exact differential form of heat for a reversible process
\3 T-S diagrams can thus be plotted similar to a P-V diagrams for a reversible process, such that for some T(S) curve, the region under the heat done, forming a rectangle for a Carnot cycle
\3 In addition, for some reversible adiabatic process, it must be isentropic, such that the entropy remains constant as well
\1 Caratheodory proved the Second Law mathematically, based on the concept that around any point, $P_0$, there exists points not accessible along solution curves of A(x, y, z)dx + B(x, y, z)dy + C(x, y, z)dz = 0 if the equation is able to be integrated
\2 This is able to be integrated if there exists $\Lambda(x, y, z), F(x, y, z)$, such that $Adx + Bdy + Cdz = \Lambda dF$, both facts holding for any amount of dimensions
\2 For some system of thermodynamic coordinates A, B, C, $\delta Q = Adx + Bdy + Cdz$, as form of the first law, such that there are states of the system inaccessible by purely quasi-static adiabatic pathways, as a form of the second law ($\delta Q = 0$)
\3 Thus, the second law is only valid if there exists some functions, T, S, such that $\delta Q = TdS$ under this form, such that by the second law, $\delta Q_R = TdS$ must be valid
\1 For an ideal gas as a result, it is found that $dS = \frac{\delta Q_R}{T} = C_P\frac{dT}{T} - \frac{V}{T}dP = C_P \frac{dT}{T} - nR\frac{dP}{P}$
\2 As a result, if $C_P$ is constant over the temperature range, $S - S_r = C_P ln\frac{T}{T_r} - nR ln\frac{P}{P_r}$, or $S = C_PlnT - nRlnP + S_0$, where r is the reference point for a relative entropy scale, allowing an entropy table to be calculated
\2 The general function for entropy of an ideal gas as a result by this and the similar formula for volume, is $\Delta S = \int^T_{T_r}C_P \frac{dT}{T} - nR ln\frac{P}{P_r} = \int^T_{T_r}C_V \frac{dT}{T} - nR ln\frac{V}{V_r}$ 
\3 For a reversible isobaric process as a result, $\frac{\partial T}{\partial S}|_P = \frac{T}{C_P}$ and for a reversible isochoric process $\frac{\partial T}{\partial S}|_V = \frac{T}{C_V}$, dependent only on the material and the temperature, rather than other coordinates
\1 For some reversible process, the entropy change of the universe is both that due to the process and the reservoir, such that since it is quasi-static, the temperature of each is equal
\2 As a result, the entropy of the system and the reservoirs are inverses, such that for a reversible process, there is no change in entropy of the universe
\end{outline*}
\subsection{Entropy Change}
\begin{outline*}
\1 For some irreversible process between equilibrium states, since entropy is a state function, the change in entropy is equal to that of a reversible process with the same initial and final equilibrium states
\2 External mechanical irreversibility processes with isothermal transformation of work into internal energy of a reservoir has no change on the system due to work rather than heat and no temperature change, while since the reservoir gains heat, that is the universal entropy change
\3 For a system with an adiabatic conversion of work into internal energy of a system, the entropy change is $\Delta S = \int^{T_f}_{T_i} C_P \frac{dT}{T} > 0$, such that if $C_P$ is constant, $\Delta S = C_P ln\frac{T_f}{T_i}$
\2 Internal mechanical irreversibility is found by free expansion, such that $\Delta S = _R\int_{V_i}^{V_f} nR \frac{dV}{V} = nR ln\frac{V_f}{V_i}$, such the entropy in the system increased
\3 While in a free expansion, the heat exchange, and thus entropy, would be 0, it is approximated as an isothermal increase in volume, such that the entropy can be calculated by the entropy equation
\2 External thermal irreversibility, such that there is a finite temperature difference, have heat flow, where the same amount of heat is transferred from high temperature to low, such that entropy increases
\2 It is seen fundamentally thus, for some irreversible process, the change in entropy must be positive, such as chemical irreversibility of diffusion (acting as two free expansions)
\2 This is proven by a theoretical system of a reversible engine feeding heat into an irreversible engine with an internal temperature difference/cold reservoir, feeding from an external hot reservoir
\3 In a cycle, such that there are no internal changes, $W = \oint dQ_{H} \leq 0$, since it cannot produce work without a cold reservoir, requiring dissipation to get back to the initial state, such that neither engine is able to produce work
\3 If the work done was 0, it would be moving back to the original state, returning all heat taken in, such that it would be reversible, thus $\oint dQ_H < 0$
\3 Since the first engine is a Carnot engine, the thermodynamic temperature ratio is valid, such that $\frac{dQ_H}{T_H} = \frac{-dQ_L}{T_M} = \frac{dQ_I}{T_M}$, since the heat lost is taken by the irreversible engine
\3 Thus, it is found that $\oint \frac{dQ_I}{T_M} < 0$ for an irreversible engine cycle, where $T_M$ is the temperature it is entering the system, rather than the temperature of the system/internal reservoir
\2 As a result, the Clausius statement of the second law of thermodynamics states that $\oint \frac{\delta Q}{T} \leq 0$
\3 It is noted though that the temperature of the entropy statement change is that of the heat entering the irreversible engine, rather than the temperature within the irreversible engine, since the temperatures are not equal due to not being quasi-static/reversible
\1 Since entropy is a state function, for some closed loop, reversible or irreversible, $\oint dS = 0$, while since $_I\oint \frac{\delta Q}{T} < 0 = _I\oint dS$
\2 As a result, it can be generally stated for all processes that $dS \geq \frac{\delta Q}{T}$, such that two out of isentropic, reversible, and adiabatic imply the third, but one does not imply the others
\1 For some nonequilibrium state moving toward an equilibrium state, the nonequilibrium can be broken into incremental volume/mass elements, such that it can be viewed as infinitesimal equilibrium states 
\1 The Entropy Principle states that for any irreversible process, the entropy of the universe increases, already shown for irreversible nonadiabatic processes, acting as an alternate form of the Second Law
\2 For some irreversible adiabatic process, $\Delta S = S_f - S$, it can be brought through a combination of a reversible isothermal and two reversible adiabatic processes to bring to the initial state
\3 Since for the reversible adiabatic process, there is no change in entropy, $\Delta S$ is equal to the negative of the change in entropy of the reversible isothermal process
\3 In addition, $\Delta S$ is then equal to the negative of the heat gained then divided by the temperature during the process after the first reversible adiabatic process
\3 As a result, during a full cycle, the only action done is the absorption of heat during the isothermal process, followed by work done on the system during the irreversible adiabatic process, such that there cannot be heat absorbed, but rather $\geq 0$
\3 Thus, the change in entropy during the irreversible process $\geq 0$, and cannot be 0 since that would imply reversibility
\2 For some nonhomogeneous composition, temperature, and pressure system, for an irreversible adiabatic process, the system is assumed to be able to be divided into infinitesimal parts, each with definite entropy and coordinates
\3 Thus, each section can be viewed as a homogeneous system, such that the total entropy of the system increases for each
\3 This assumes it can be broken up into infinitesimal parts and reversible processes for chemical reactions can be found, both of which relying on experimental proof
\2 For some object being cooled from the surrounding temperature $T_i$ to $T_f$ by a Carnot refrigerator, the minimum work for a real refrigerator can be calculated
\3 $\Delta S_{body} + \Delta S_{reservoir} = S_f - S_i + \frac{Q + W}{T_i} = 0$, such that it provides the minimum work for a real process, using an entropy table to determine
\1 It can be seen from the process of irreversible, isothermal state changes or by free expansion, that an increase in entropy in the system manifests as a decrease in the atomic order/structure, describing a state just as entropy does
\2 In the conversion of work to heat, this is viewed as the conversion of ordered kinetic energy to random microscopic kinetic energy
\2 Thus, entropy can also be viewed as a mechanism of time, such that entropy must always increase over time, used as a measurement
\1 The Third Law of Thermodynamics states that at absolute 0 $(0^o K)$, the entropy of the system is set to 0
\end{outline*}
\section{Miscellaneous Thermodynamics Notes}
Boundary layer is what changes temperature, rather than the wall itself, such that it is what modifies? Convection and conduction simulaneously? Ice water == resistor in series - Homework ice problem
Oscillating electric charge emits an electromagnetic waves/radiation, emitting many types in a sample due to random motion
Opaque bodies have 0 transmission fraction, transparent have 1, white bodies have reflective fraction of 1
Planck radiation law solved the lack of classical theory infinite 5000K radiation, leading to photons and quantum mechanics
Higher frequency peak as temperature increases

Air is a poor conductor, such that atmosphere has chemical change, radiation, and convection (particle flow) for heat change, not conduction, processes within it can be considered adiabatic, such that for problem 10, $M\Delta ng = M\rho A\Delta yg = \frac{MPAg\Delta y}{RT}$

From above, $F = P(y + \Delta y)A$, from below F = P(y)A, gravity from above, such that at equilibrium, it can be solved for the density and pressure with height
29 g/mol - air average
-dT/dy = -9.77 K/km

Heat is conversion of ordered work into unordered entropy on a microscopic scale, such that the production of heat from non-internal energy is dissipative, even if it is due to prior internal energy
Lack of nondissipative processes is fundemental law of nature, due to view as entropy, but not as a result of first two laws of thermodynamics, under which perpertual of third kind is allowed
Engines converted ordered energy to heat to ordered energy and heat, but increase overall universal entropy, even if geothermal machine passign heat somehow

Carnot cycle is a theoretical, but practically createable, and reversible engine, due to including no inherently irreversible processes, unlike other engines

Irreversibility cannot be seen from a PV diagram, due to only paying attention to the system for dissipative, thus just changing the values, and for non-quasi-static, pressure and volume at any points except the extremes are meaningless

$dS = C_V dT/T + P/T dV$ by first law
$\frac{\partial T}{\partial S}|_V = \frac{T}{C_V}, \frac{\partial T}{\partial S}|_P = \frac{T}{C_P}$, since $C_V < C_P$, the slope of the former is greater on a TS diagram 

\section{Freegarde Chapter 1 - Wave Motion Essence}
\begin{outline*}
\1 Physics can be viewed by Lagrangian Particle Theory or Euler Field Theory, as a duality of perspectives, most apparent on a quantum scale
\2 As a result, when considering a dynamic/time-dependent system, it becomes Kinetic Theory and Wave Theory respectively
\1 Waves are defined microscopically as a collective bulk disturbance, created at a point as a delayed response to the disturbance at adjacent points
\2 The disturbance progression to adjacent points is called the propagation, going through the medium
\2 The medium does not need to be linear, homogeneous, with sinusoidal or periodic propagation to classify as a wave
\2 Positions in the medium are defined by coordinates, each position requiring at least one additional variable to describe the disturbance
\3 The coordinates and disturbance variables are each wavefunctions, defined with respect to time
\1 Waves must be viewed in terms of both cause and effect, such that it is defined macroscopically as a time-dependent field effect due to finite speed of propagation of a causal effect
\2 Thus, it can be viewed as the solution to systems in dynamic equilibrium, more generalized than static equilibrium theory
\2 Electric Coulomb waves can thus be viewed as a result of a rotating dipole acting on a single charge at some distance
\3 If the dipole radius is assumed to be far less than the distance from the charge, it can be simplified such that $E(t) = \frac{2kq}{r_0^3}a(t - \frac{r_0}{c}$
\4 a(t) is the perpendicular height of each side of the dipole, while $r_0$ is the distance from the center of the dipole to the particle
\4 The time delay due to the relativistic limitations on which the change in force is received create the time lag/retardation, allowing the wave
\4 The magnetic component of electromagnetic waves are found by a Lorenz transformation taking into account the motion of the charges and dipoles themselves
\3 Gravitational waves have a similar wave effect for a gravitational single body, such that $g = \frac{Gm}{r_0^3}a(t - \frac{r_0}{c})$
\4 On the other hand, since the rotation of the causal body requires a third body to rotate with it similar to a dipole, it results for similar bodies in a higher order (inverse fourth root law), canceling out the effect to some degree
\4 Thus, it is found that g = $\frac{2Gma_0^2}{r_0^4}sin(2\omega t)$ assuming the distance between the two causal bodies is vastly smaller
\2 As a result, waves can be viewed as some function dependent on an action at a prior time at some other object
\1 It was believed until the 1900s that objects required a medium to influence each other in a vacuum, leading to the idea of the aether
\2 As a result of relativity and electromagnetism, it became apparent that there is no ether, such that forces can be sent through a vacuum
\2 On the other hand, for waves, there remains the concern of diffraction in a vacuum, with apparently nothing deflecting, and the concept of radiation, sending energy through a vacuum alone
\3 Current theories create no issues with these concepts, though they remain difficult to conceptualize
\1 Transverse wave motions, such as surface tension, electromagnetic, or gravitational, where the disturbance is perpendicular to the direction of wave propagation
\2 Longitudinal waves are those where the disturbance is parallel to the direction of propagation, such as sound, generally due to pressure differences on each side
\3 On the other hand, gravitational or electromagnetic waves can be created longitudinal rather than transverse
\2 Spin waves are those with both transverse and longitudinal components, such as seismology
\2 Waves can also be scalar quantity displacement, such as thermal waves (based on heat transfer), quantum wavefunction, or chemical composition waves (based on a reaction-diffusion system)
\2 Waves are also not required to be moving through a continuous medium, as long as they fulfill the definition
\end{outline*}
\section{Freegarde Chapter 2 - Basic Wave Equations}
\subsection{General Concepts}
\begin{outline*}
\1 Wave equations are partial differential equations which relate the wavefunction (derivatives of wave displacement) to time and position, derived first from the general physical situation
\1 It is found that for traveling waves, the shape remains the same though the location of the disturbance changes with time
\2 This is due to the lack of loss of energy during propagation, such that there is no dissipation/damping of the wave
\2 As a result, it is found that $\psi(x, t) = \psi(x - \delta x, t - \delta t) = \psi(x - v\delta t, t - \delta t)$, since $\delta x = v \delta t$
\3 Thus, if $\delta t = t$, $\psi(x, t) = \psi(x - vt, 0) = \psi(u)$, such that the variables must appear in that combination for a forward traveling wave
\3 Thus, since the wavefunction is the same value over a specific curve, this single variable function is found for a traveling wave
\3 This all follows from the method of viewing waves as the motions of individual particles, collectivized, such that x determines which particle is being viewed, while t is at which time
\2 In addition, both sides can be subtracted by $\phi(x, t - \delta t)$, by the x-t relation, such that $\frac{\partial \phi}{\partial t} = -v \frac{\partial \phi}{\partial x}$
\3 This assumes that $\phi$ is continuous, or at least can be approximated as a continuous function
\3 Further, it is found similarly that $\frac{\partial^n \phi}{\partial t^n} = (-v)^n \frac{\partial^n \phi}{\partial x^n}$, though the equation would have to be modified due to the sign of n changing
\1 Generally, wave motion may be done relating the derivative with respect to time at some point to that relating the derivative with respect to location at that time
\2 This is then used to determine the general form of the wave equation, followed by inputting specific parameters for the solution
\end{outline*}
\subsection{Long String Waves}
\begin{outline*}
\1 For some string with tension T, assuming gravity effect is negligible, such that for some unit length mass M with length $\delta x$, tension is acting on both sides
\2 The angle is assumed to be slightly varying on the sides of the length, with small enough total displacement that $cos\theta_1 \appx cos\theta_2 \appx 1$
\3 Since the tension on the side in the direction of the wave is toward the positive wave function, while it is toward the negative for the other at some point, $F_{perpendicular} = T(sin\theta_2 - sin\theta_1)$
\3 When this distance is infinitesimal, the change in angle for each side is infinitesimal, moving the wave toward the slightly steeper direction
\2 By definition of the angle, $tan\theta = \frac{\partial \psi}{\partial x}$ for each end of $\delta x$
\3 Since the angles are small enough that $cos\theta \appx 1$, it can be stated that $sin\theta \appx tan\theta$
\3 As a result by this and Newton's Law, $F_n = T(\frac{\partial \psi}{\partial x}_{x_2} - \frac{\psi}{\partial x}_{x_1}) = M\delta x (\frac{\partial^2 \psi}{\partial t^2}|_x)$
\4 This is due to the fact that the acceleration at any one point is equal to the wave function second derivative with respect to time
\3 Thus, as divided by the length and taken to the limit as the length approaches 0, it is found that $\frac{\partial^2 \psi}{\partial t^2} = \frac{T}{M}\frac{\partial^2 \psi}{\partial x^2}$
\1 Using the formula for traveling waves in terms of one variable combined with the chain rule and the above formula, it is found that $v^2 \frac{d^2 \psi}{du^2} = \frac{T}{M} \frac{d^2 \psi}{du^2}$
\2 Thus, for a traveling string wave, $v = \pm \sqrt{\frac{T}{M}}$, such that the velocity depends on the tension and density
\1 Since the wave equation has linearity, such that superposition applies to it, the general solution is equal to the linear combination of individual solutions, spanning the solution set
\2 Thus, the two respective solutions are for a positive and negative velocity traveling waves, such that it is equal to the solution to a wave with positive and negative velocity
\3 Thus, this forms the idea that since the velocity can be in either direction when assuming a standing wave, all waves on the string are made up of two standing wave directions
\2 Thus, it is found that $\psi(x, t) = \psi(x \pm vt) = \psi_+(x - |v|t) + \psi_-(x + |v|t)$, to provide the forward and backward wave components, taking the specific solution at t = 0
\3 This forms $\psi(x, 0) = \psi_+(x) + \psi_-(x)$ and by the derivative with respect to t, $\frac{\partial \psi}{\partial t}(x, 0) = \frac{\partia \psi_+ (x - |v|t)}{\partial t}(x, 0) + \frac{\partial \psi_- (x + |v|t)}{\partial t}(x, 0)$
\4 The first equation serves to provide the initial condition for a unique solution
\3 By the chain rule, the second equation becomes $\frac{\partial \psi}{\partial t}(x, 0) = |v|(\frac{d\psi_-(x)}{dx} - \frac{d\psi_+(x)}{dx})$
\2 For some guitar string connected at two ends, [0, l], pulled at some point $x_0$ to initial height $a_0$ at t = 0, with initial velocity of 0
\3 By integrating the velocity formula for t = 0, $\psi_-(x) = \psi_+(x) + c$, such that at t = 0, $\psi(x, 0) = 2\psi_+(x) + c = 2\psi_-(x) - c$
\1 By these formulas for $\psi_\pm$ in terms of the initial wavefunction of some point, $\psi(x, t) = \frac{1}{2}\psi(x - |v|t, 0) + \frac{1}{2}\psi(x - |v|t, 0)$
\2 This assumes that x is not at the endpoints at some nonzero time, since it is undefined for a negative displacement or a displacement greater than the length
\2 This equations can be supplemented by the fact that $\psi(0, t) = \psi(l, t) = 0$ for any time t, acting as a boundary condition
\2 This is used to find the restriction that for any x, t, $\psi_+(x - |v|t) = -\psi_-(2l - x - |v|t) = \psi_+(x - |v|t - 2l)$, such that the component waves are triangular/sawtooth
\end{outline*}
\end{document}
