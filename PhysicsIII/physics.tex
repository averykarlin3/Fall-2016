\documentclass[11 pt, twoside]{article}
\usepackage{textcomp}
\usepackage[margin=1in]{geometry}
\usepackage[utf8]{inputenc}
\usepackage{color}
\usepackage{indentfirst} %Comment out for no first paragraph indent
\usepackage[parfill]{parskip}
\usepackage{setspace}
\usepackage{tikz}
\usepackage{amsmath}
\usepackage{amsfonts}
\usepackage{amssymb}
\usepackage{enumitem}
\usepackage{outlines}

\usepackage{fancyhdr}
\pagestyle{fancy}
\cfoot{\hyperlink{content}{\thepage}}
\lhead{}
\chead{}
\rfoot{}
\lfoot{}
\rhead{}
\renewcommand{\headrulewidth}{0pt}
\renewcommand{\footrulewidth}{0pt}


\usepackage{hyperref}
\hypersetup {
	colorlinks,
	citecolor=black,
	filecolor=black,
	linkcolor=black,
	urlcolor=black
}

\newcommand{\sepitem}{0pt} %Added room between items on the list, not including a list and its sublist
\newcommand{\seppar}{1pt} %Between items and lists overall

\setenumerate[1]{itemsep=\sepitem, parsep=\seppar}
\setenumerate[2]{itemsep=\sepitem, parsep=\seppar}
\setenumerate[3]{itemsep=\sepitem, parsep=\seppar}
\setenumerate[4]{itemsep=\sepitem, parsep=\seppar}

\newenvironment{outline*}
{
	\begin{outline}[enumerate]
	}
	{\end{outline}
}

\newcommand{\foot}[1]{\hyperlink{#1}{$_#1$}}

\begin{document}

\title{Introduction to Physics III: Thermodynamics, Waves, and Relativity}
\author{Avery Karlin}
\date{Fall 2016}
\newcommand{\textbook}{Heat and Thermodynamics by Zemansky and Dittman, 7th Edition}
\newcommand{\teacher}{Dr. Johnson}

\maketitle
\newpage
\hypertarget{content}{\tableofcontents}
\vspace{11pt}
\noindent
\underline{Primary Textbook}: \textbook\\
\underline{Secondary Textbook}: Introduction to the Physics of Waves by Freegarde \\
\underline{Secondary Textbook}: Special Relativity by Helliwell \\
\underline{Teacher}: \teacher
\newpage

\section{Chapter 1 - Temperature and the Zeroth Law of Thermodynamics}
\subsection{Definition of Thermodynamics}
\begin{outline*}
\1 The study of a natural system requires the creation of a boundary, separating a section of space and matter, or the system, from its surroundings, called closed if no matter is able to cross the boundary, open if there is an exchange of matter
\1 Systems are then studied on either a macroscopic/human scale or a microscopic/molecular scale
\2 Systems are described on a macroscopic scale by composition, mass, volume, and temperature, and other aggregate properties, acting as macroscopic coordinates for the macroscopic description
\3 Macroscopic coordinates have the properties of assuming nothing special about the matter structure, fields, or radiation, require few to describe a system, suggested by sensory observation, and can be directly measured
\3 While mechanics describes the external properties of a system and its energy, thermodynamics describes the internal properties
\2 Systems are described on a microscopic scale by statistical mechanics, describing the populations/number of particles in each energy state at equilibrium, and the interactions of particles with each other by collisions and fields, and with other systems in an ensemble
\3 The equilibrium state is the state of the highest probability, and the system is assumed to have some high number of particles
\3 Microscopic descriptions have the properties of making assumptions about the structure of matter, fields, and radiation, requires many quantities, is based mathematical models rather than observation, and must be calculated rather than measured
\2 The scales must reconcile to the same conclusion, with the macroscopic view as the average over some amount of time of the microscopic characteristics
\3 Since the microscopic view requires assumptions and models rather than observations, it is able to change as the result of increased data, unlike the macroscopic view which is used to test the assumptions
\1 Thermodynamics is the study of the macroscopic properties of nature, especially the temperature of the system, and the identification of thermodynamic relations based on the fundamental laws
\2 Mechanical coordinates used within classical mechanics are from the mechanical/external energy of the system to measure the movement of the system overall, while thermodynamics uses macroscopic coordinates to deal with the internal energy of the system, called thermodynamic coordinates
\2 Systems described by thermodynamic laws are thermodynamic systems, found within each discipline
\1 Thermodynamics is the branch of physics which describes the relationships between all forms of energy, conserved in an isolated system by the Law of Conservation of Energy
\2 Heat is energy in transit from one body at higher temperature to another with lower temperature, with 1 calorie as the heat to raise the temperature of 1 g of water by $1^oC$ at 1 atm, equal to 4.184 J
\end{outline*}
\subsection{Thermal Equilibrium and Temperature}
\begin{outline*}
\1 For some thermodynamic system of constant mass and composition, where to describe the system requires two coordinates, X and Y
\2 The system is at equilibrium state if the coordinates would remain constant if the external conditions are unchanged
\2 Equilibrium states depend on the proximity to other systems and the boundary within systems (adiabatic if any equilibrium state is present on both, such as wood, while diathermic if there must be specific combinations of equilibrium states, such as metal)
\3 Thermal equilibrium is achieved by multiple systems separated by a diathermic wall when they have reached a combined system equilibrium
\3 Diathermic walls are thus boundaries that allow heat transfers, while adiabatic do not allow heat transfers, both not permitting matter transfers (assuming it does not break due to stress from the systems)
\2 The Zeroth Law of Thermodynamics states that if two systems are each in thermal equilibrium with another system simultaneously, then they are at thermal equilibrium with each other as well
\3 It is called this due to the next two relying on this definition, found experimentally as a fundamental law of the discipline, creating the concept of temperature as a universal quantity
\1 Temperature is designed as the macroscopic measure of hotness from an object observed, and microscopically as the average kinetic energy of particles, defined as some number $\geq 0$ in science (Kelvin), ignoring the terms of coldness
\2 For any given state of a system, there exists a curve, or isotherm, of the locus of all states of another system which would be in thermal equilibrium with the first system, experimentally found to generally be continuous over some region of the curve
\3 The same can be found for the other system at each set of coordinates, such that by the zeroth law, each isotherm has a corresponding isotherm in the other system, such that all points on those lines are at equilibrium with each other
\3 The property which determines if a system is at thermal equilibrium is called temperature, since all that is needed is a correspondence of equilibrium between two systems and a reference, such that it is a scalar, rather than a vector if there was not a transitive property
\2 Thus, for systems at the same temperature, they are on corresponding isotherms and thus at thermal equilibrium, such that temperature is a measure of thermal equilibrium
\3 The converse of this is also true, such that thermal equilibrium implies same temperature
\1 Temperature equality can be determined by a capillary of mercury, equally filled and sized, measuring the equality of the height the mercury rises, experimentally found to signify equilibrium, called a thermoscope
\2 Thermometers measure the temperature on an empirical scale, using some rule to assign a value to each isotherm, finding some path on the isotherm plane, with the varying variable as the thermometric property, the thermometric function, $\theta(X)$ to get the temperature
\3 The traditional mercury volume thermometer relies on the asymetrical oscillation as energy increases, such that it is more likely to be at the larger side of oscillation, creating an average larger volume
\4 This is defined by $U(r) = 4\epsilon((\frac{\omega}{r})^{12} - (\frac{\omega}{r})^6)$, such that $\epsilon$ is the minimum energy, r is the distance between molecules, and $\omega^{\frac{2}{3}} = r_0$, where $r_0$ is the minimum distance between molecules
\4 All objects expand outward as a result, such that there would be increased space in a cavity as well, expanding only outward
\2 The historic scale mercury pressure thermometer (constant volume) is based on a linear relationship, $\theta(X) = aX + c$, without a constant defining an absolute temperature scale (absolute zero at $0^o)$
\3 The same relation when applied to other thermometers or other thermometer systems of the same type (the international standard being hydrogen pressure) produces other scales, with some defined coefficient for each
\2 Thus, when a thermometer is placed in contact with a chosen standard system in a reproducible state, called the fixed point temperature
\3 Until 1954, the world standard was Celsius, but which was too difficult to measure accurately due to water surrounding ice and sensitivity to minor pressure fluctuations
\3 After, the Kelvin system is based on the triple point of water (three state equilibrium, called the standard fixed point), able to measured accurately, as an absolute system, assigned 273.16 K, equal to $0.01^o C$
\4 This is measured by putting water under such pressure in a sealed tube that it begins to boil, using a freezing mixture to form a layer of ice on the edge, after which the mixture is removed and water forms between the ice and the edge, still evaporating, at the triple point
\end{outline*}
\subsection{Thermometer Types}
\begin{outline*}
\1 For each type of thermometer, a test using the same equation would produce drastically different results from those of the standard at all values except the reference
\2 Hydrogen gas pressure thermometers tend to be negligibly different at standard pressures, acting as the reference for normal temperature and pressure
\1 Gas thermometers are made up of a measuring gas containing bulb, attached to a mercury column through a capillary, keeping the volume of the gas constant by adjusting the height of the mercury column
\2 The mercury column has two columns, one attached to the capillary, the other to the dead space, with a column of the measuring gas reaching from underneath the columns to the reservoir
\3 If high enough pressure, the mercury can rise into the dead space/nuisance volume at the top of the capillary, adjusting the mercury in the column to the point that the mercury fills the capillary column fully to keep volume constant
\2 Thus, at this point, the pressure of the main column is the atmospheric pressure added to the difference in heights between the columns
\3 It can then be measured when the bulb is surrounded by the triple point water and when surrounded by the measured system
\2 The pressures must be corrected for errors due to the gas being a different temperature than the bulb, the capillary not being uniform temperature, volume changes in any part as the temperature and pressure change
\3 It can also have error due to gradient in the capillary due to the diameter being close to the mean free path of the gas, gas adsorbed by the device, especially at low temperatures, and temperature and compressibility of the mercury creating error
\2 As a result, once corrected, the measuring gas behavior appears close to that of an ideal gas
\3 The basis of gas thermometers was the ideal-gas law, stating PV = nRT, where n is the number of moles of gas, T is the Kelvin temperature, and R is the molar gas constant
\3 This is able to be used for the reference, to derive the same relation between triple point pressure/temperature and measured pressure/temperature ($\frac{P}{P_TP} = \frac{\theta}{273.16 K}$), such that the temperature definition thus assumes an ideal gas for a Kelvin thermometer
\2 Helium is the best measuring gas, not diffusing through platinum at high temperatures, but not becoming a liquid until extremely low temperatures
\1 Vapor pressure thermometry depends only on the vapor pressure of the measuring gas at some temperature, thus easy to produce, and is easy to measure
\1 Platinum resistance thermometry is secondary, due to being fundamentally defined in terms of temperature, but with a broader range than gas thermometers
\2 This is done by winding a long, thin wire around a frame, loose enough to avoid strain when compressing, able to also be wrapped around the object being measured, drilled into the object, or bonded onto the surface
\2 Potentiometric resistance thermometers have zero direct current into the leads, while bridge circuits are balanced such that there is no alternating current through
\3 Bridge circuits are made possible as thermometers due to inductive voltage-dividers/ratio transformers and high sensitivity/signal-to-noise lock-in amplifiers
\2 Within the ideal measurement range, $R'(T) = R'_{TP}(1 + aT + bT^2)$, where a and b are constants, generally measured as $W(T) = 1 + aT + bT^2$, measuring the ratio of resistivity, which is less prone to material tension/composition error
\1 Radiation thermometry, including optical, radiation, infrared, and total radiation, are used to measure thermal/blackbody radiation, using Planck's radiation law relating temperature to radiance, depending only on the walls of a closed body, not the shape
\2 The dimensions must be larger than the wavelengths themselves, and there must be a small hole to allow radiation to escape to allow measurement, negligible enough to not disturb equilibrium
\2 Pyrometers were made to measure high temperatures without contact, comparing visible radiation over a small band with that of a standard, using a photoelectric detector, correcting for emissivity of the source at that band
\2 Total-radiation pyrometers have a larger range able to measure, but are less precise, measuring all radiation from the object, rather than just blackbody
\1 Thermocouples are an outdated method of measurement due to high inaccuracy of 0.2 K, made of two wires connected to copper wires in a melting point reference junction, connecting the two wires and measuring the EMF at various values
\2 $\epsilon = c_0 + c_1\theta + c_2\theta^2 + c_3\theta^3$, where $c_n$ are constants based on the materials of the thermocouple, though for a small range of temperature, $c_3$ approximates to 0
\2 The temperature range of the thermocouple depends on the materials used, advantageous because of the small, conductive mass allowing rapid temperature equilibrium with the material being measured
\2 The copper wires can be connected to electrical circuits for monitoring of electrical equipment temperatures or to a simple voltmeter for measurement
\end{outline*}
\subsection{Temperature Scales}
\begin{outline*}
\1 As the triple point pressure of a gas (due to lessening volume), approaches 0, the resulting temperature calculated is the ideal-gas temperature
\2 While thermometers depend on the specific measuring gas properties to provide the temperature, as it approaches the ideal gas temperature, it becomes independent of the measuring gas, though based on the properties of gases overall, rather than an individual gas, behaving ideally
\3 Due to this ideal specific gas-independent temperature, it is an ideal state of matter for measuring of temperature
\2 For the temperature region in which a gas thermometer may be used, the measured Kelvin temperature scale and ideal gas temperature scales are identical
\2 It is noted that the idea of lack of atomic motion at absolute zero cannot be assumed, due to relying on the complete equivalency of temperature and atomic motion on different scales 
\3 In addition, at absolute zero, there is some residual energy expected based on quantum mechanics, resulting in the zero-point energy quantity
\1 The Celsius scale was used prior to the Kelvin temperature scale until 1954, which slightly shifted the base from the ice point to the triple point of water (273.16 K) for accuracy and modifying to an absolute scale, with the same degree of magnitude between values
\2 Thus, the Celsius and Kelvin scales simply have a constant difference of the measured ice point, equal to 273.15 K
\2 As a result, the modern Celsius scale no longer has a fixed ice/steam point for water, but rather purely the triple point, such that Kelvin is the standard
\1 Fahrenheit and Rankine are based on a scale 5/9ths that of the Celsius and Kelvin, based on the triple point of water, such that the ice point is found to be $32^o F$, providing a relationship to Celsius
\1 The International Temperature Scale of 1990 was the creation of a practical scale for routine measurements and calibration of instruments, to ease measurement from time-consuming calibration by the single reference point,  providing a series of fixed points for comparison
\2 It also provides close measurement interpolation for the remainder, providing a close approximation to the Kelvin scale, by other thermometer types
\3 This uses polynomial approximations of constants to approximate the scale of another thermometer for that region
\2 At the minimum of 0.65 K to 3.2 K, it is defined by the $^3He$ vapor pressure-temperature relation, from 1.25 K to 2.1768 K to 5 K by $^4He$ of the same type
\3 $^3He$ fails below 0.3 K due to being too small to measure, and fails above 3.32 K due to the critical point (where only gas exists), while $^4He$ fails below 1 K due to superfluid behavior causing less variation with temperature and errors, and above 5.2 K due to the critical point
\2 From 3 K to 24.5561 K by either He gases as a constant-volume gas thermometer
\2 From 13.8033 K to 1234.93 K, it is defined by platinum resistance thermometers with fixed points within and deviation functions between, creating 11 sub-ranges
\2 Above that, it is measured by an optical pyrometer, using blackbody spectral radiance concentration ratios by Planck's radiation law, using only a single reference of the freezing point of gold, silver, or copper, replacing the thermocouple
\end{outline*}
\section{Chapter 2 - Simple Thermodynamic Systems}
\subsection{Thermodynamic Equilibrium}
\begin{outline*}
\1 Change of state of a system is when the coordinates change either spontaneously or due to outside influence, isolated if the former, such that there is no outside influence, rarely found practically
\2 Systems not in mechanical equilibrium (with no unbalanced force or torque in the system interior or the surroundings/no unequal pressure) will have a change of state until it is corrected
\2 Systems not in chemical equilibrium (unequal chemical potential) result in a transfer of matter through or out/in the system, or a spontaneous change of internal structure as a reaction to restore it
\2 Systems in thermal equilibrium are those with no change in the system's thermodynamic coordinates when separated diathermically, such that there is no heat transfer (meaning temperature is equal with surroundings and within the system)
\3 Otherwise, change of state will occur until thermal equilibrium is restored
\1 Thermodynamic equilibrium is when thermal, mechanical, and chemical equilibrium are found simultaneously, such that it can be described by thermodynamic macroscopic coordinates at a particular instant, though not over time
\2 Thus, when not at mechanical, chemical, and thermal equilibrium, coordinates like composition, temperature, and pressure cannot be used to apply to the system as a whole, as it moves through nonequilibrium states
\1 Equations of state relate the coordinates of a thermodynamic equilibrium system to each other, based on the individual specifics of a system, such that it is determined by molecular theory or experiment, not general theory, but can be assumed to exist for all thermodynamic equilibrium systems since all coordinates are not independent for a system
\2 At low pressures, the state equation of ideal gas is PV = nRT, or Pv = RT, where v is the molar volume ($\frac{V}{n}$)
\2 At high pressures, it is represented by the van der Waals equation, based on particle interactions and sizes, such that $(P + \frac{a}{v^2})(v - b) = RT$, where v is the molar volume
\3 a and b are gas-specific constants, since equations of state are defined by existence, rather than any particular form without constants
\3 a is the molecular attraction while b is the atomic short range repulsion of the system, such that it accounts for higher pressures in which gases behave less ideally 
\2 Nonequilibrium states cannot be described purely by coordinates as a whole, such that there is no equation of state for the system
\end{outline*}
\subsection{Simple Systems}
\begin{outline*}
\1 Simple systems are those which are described by three thermodynamic coordinates in an equation of state, with the special case of PVT systems called hydrostatic systems
\1 Hydrostatic systems are also defined as an isotopic (uniform) system of constant mass and composition that exerts uniform hydrostatic pressure on the surroundings if there is no gravitational or electromagnetic effects
\2 They are divided into three catagories, pure substances (a single compound of any mixture of states), homogeneous mixture of different compounds (mixture of different compounds of the same state or a solution), and heterogeneous mixtures (mixtures of multiple homogeneous multiple compound mixtures of different states)
\3 It has been found experimentally that systems of a single state can be described by three coordinate equations of state, PVT
\2 These equations can describe some macroscopically small change in a measurement by a differential, though since they are macroscopic coordinates, it describes an amount greater than a few particles changing, rather than an infinitesimal amount
\2 Thus, for some function V = V(T, P), $dV = \frac{\partial V}{\partial T}(T, P)dT + \frac{\partial V}{\partial P}(T, P)dP$, abbreviated as $dV = \frac{\partial V}{\partial T})_PdT + \frac{\partial V}{\partial P})_TdP$
\3 The average coefficient of volume expansion is defined as the change in volume per unit volume divided by the change in temperature at constant pressure
\4 As the change in temperature, and thus volume, approaches 0, the differential coefficient of volume expansion, or the volume expansivity, $\beta = \frac{1}{V}(\frac{\partial V}{\partial T})$
\4 It is generally a positive number, though there are exceptions such as water for $0-4^o C$ melting point at which point water reaches maximum density
\4 This is due to hydrogen bonding of ice and surface tension of water and is generally a function of P and T
\4 It often varies negligibly with respect to P and T often within a small temperature and pressure range, such that it can be considered a constant over the range
\3 The average bulk modulus is defined as $\frac{-V\Delta P}{\Delta V}$, due to the pressure increasing causing a decrease in volume, such that it is a positive value
\4 As the change in volume, and thus pressure, approaches 0 as temperature is constant, it is called the isothermal bulk modulus, $B = -V(\frac{\partial P}{\partial V})$
\4 Isothermal compressibility, or the reciprocal of isothermal bulk modulus, $\kappa = \frac{-1}{V}(\frac{\partial V}{\partial P})$, and is similarly often constant for a material over some PT range
\2 This can be similarly found for each equation of state of a different explicit variable, all of which are differentiated to determine an explicit exact differential equation, describing a slight change in state of equilibrium for that explicit differential in terms of the other variables
\3 Exact differentials are contrasted as the differential of an actual function to that of an inexact differential, denoted by $\delta$
\3 By combining two of exact differential equations equations, it follows that if y(x, z) and dz = 0, dx $\neq 0$, then $\frac{\partial x}{\partial y}\frac{\partial y}{\partial x} = 1$
\4 Further, if dz $\neq 0$ and dx = 0, then by extension, $\frac{\partial x}{\partial y}\frac{\partial y}{\partial z}\frac{\partial z}{\partial x} = -1$
\4 By extension of that, $\frac{\partial y}{\partial z}\frac{\partial z}{\partial x} = -\frac{\partial y}{\partial x}$
\3 Thus, by those theorems, $\frac{\partial P}{\partial T} = \frac{\beta}{\kappa}$, such that each partial derivative has a physical representation by the theorems
\4 As a result, a form $dP = \frac{\beta}{\kappa}dT - \frac{1}{\kappa V}dV$ of the state equation is found, which can be integrated for each in terms of the others
\1 Wires are a one-dimensional system, usually assumed that pressure and volume are constant, such that the coordinates are tension, $\tau$, length, L, and temperature, T,
\2 While it cannot be expressed universally by a single equation, for a constant temperature in the portion of elasticity such that Hooke's law is true, $\tau = -k(L - L_0)$, where $L_0$ is the length at no tension
\3 Since equations of state assume thermodynamic equilibrium, the tension must be balanced out by some external force
\2 For some infinitesimal equilibrium state change for $L = L(T, \tau)$, $dL = \frac{\partial L}{\partial T}(T, \tau)dT + \frac{\partial L}{\partial \tau}(T, \tau)d\tau$
\3 By the limit of the average linear coefficient of expansion, linear expansivity, $\alpha = \frac{1}{L}\frac{\partial L}{\partial T}$
\4 Metals tend to have a positive linear expansivity, but other materials such as rubber may be negative
\4 Linear expansivity is mainly constant for torque change, only changing for temperature, but over a small range, can be considered constant
\3 By the limit of average Young's modulus, isothermal Young's modulus, $Y = \frac{L}{A}\frac{\partial \tau}{\partial L}$
\4 Isothermal Young's modulus is always positive, only changing for temperature, but can be considered constant over some small range as well
\1 Surfaces are a 2D system, with different properties from those of the underlying material, acting as a stretched membrane, with each point exerting a force opposite and perpendicular to the opposite side of the surface
\2 Surfaces are thus described by the area, A, the temperature, T, and the surface tension, $\gamma$, which is the force per unit length of the surface, at any point in any direction on the surface, with an equal and opposite force on the surface
\2 While the substance within the surface is relevant, the volume and pressure of it can be ignored, due to remaining fairly constant for small changes in the surface coordinates
\2 For pure liquids at equilibrium with their vapor phase, it is described by the Guggenheim equation of state, $\gamma = \gamma_0(1 - \frac{T}{T_c})^n$, where $T_c$ is critical temperature, $\gamma_0$ is surface tension at standard temperature, generally $20^o C$, and n is a constant, such that $1 \leq c \leq 2$
\3 As a result, as T approaches $T_c$, the surface tension approaches 0, after which there is unable to be any liquid form
\2 For oil films on liquid water, it is described by the equation $(\gamma - \gamma_w)A = aT$, where $\gamma - \gamma_w$ is called the surface pressure, acting as a 2D version of the ideal gas law, and where a is some constant
\2 For soap bubbles, it functions as two surfaces trapping a small amount of liquid between the two, while another common surface is a single layer of gas on a solid surface
\end{outline*}
\section{Chapter 3 - Work}
\subsection{Definitions}
\begin{outline*}
\1 External work is work done on or by the system as a whole, while internal work is work done by part of the system on another part, such as interaction of molecules, only the former able to be done by macroscopic thermodynamics
\2 External work can be viewed as the changing of macroscopic coordinates by the modification of the configuration/position of some external mechanical device or system, such as thermodynamic electrochemical cell systems powering a mechanical device
\3 Thus, assuming the system is in thermodynamic equilibrium and there is no change in the configuration of external systems, no external or internal work is done
\2 Work done on a system is denoted positively, while work done by the system is denoted negatively within thermodynamics and mechanics
\1 Mechanical external work leads to forces and torques within the system, causing external acceleration, internal turbulence, waves, or other forms of internal motion
\2 This can lead to a nonuniform temperature distribution as well as a lack of thermal equilibrium with external systems
\3 The lack of thermal and mechanical equilibrium can lead to spontaneous chemical reactions or gradients forming within the system
\2 As a result, mechanical external work can generally not be described thermodynamically, such that only an infinitesimal applied force, called a quasi-static process, such that it is infinitesimally close to equilibrium during the process
\3 Thus, it can be thought to be described by an equation of state, being an idealization of work on any thermodynamic system, since the time to move to a new equilibrium state is negligible
\3 This is used to be able to ignore minor forces acting on a system, giving the ideas of ideal devices without friction and resistance, and to be able to allow change of state equations to apply
\3 As a result, all processes are reversible, due to no dissipative forces affecting the system, such that energy is not lost from the system
\2 Quasi-static process theory only applies to systems with proper boundaries, such that thermodynamic equilibrium is able to be reached
\2 It is notated that since the unbalanced force is infinitesimal, the equation of state variables and the force are considered to be equal for the system and external systems
\1 While internal work does take place on a microscopic scale within the system, it is not the purview of thermodynamics, but rather within statistical mechanics, such that only external work is viewed on the macroscopic scale
\end{outline*}
\subsection{Hydrostatic System Work}
\begin{outline*}
\1 For some hydrostatic system in a closed cylinder with a piston to lower the height of the system, the pressure exerted by the system on the piston is P, the area of the piston A, such that $F_i = PA$, with some external force, $F_e$, only infinitesimally different
\2 For work done, lowering the volume of the system, $dW = F_edx = PAdx$, since the difference between the forces is infinitesimal
\3 This is under the assumption of a quasi-static system, such that dissipative processes or chemical movement/reactions would not have a large enough effect to render this invalid
\3 On the other hand, lack of mechanical equilibrium, such that there is not a standard pressure definition, would render this invalid
\3 This can be expanded to other hydrostatic systems of different shape, such that the underlying ideas are still valid
\2 As a result, since $Adx = -dV$ by the decreasing volume, such that $dW = -PdV$, often with work measured in liter-atm for convenience, such that 1 L-atm = 100 J
\3 This can be integrated for the work done by the external system, assuming the difference between the calculated work done and the actual is negligible, since the process is quasi-static, pressure is a function of temperature and volume by the equation of state
\3 Once the temperature pathway is found for the particular quasi-static process, it can be integrated purely in terms of V
\1 For the system, a PV diagram can be drawn for the relationship between pressure and volume as the piston moves, traversing the curve in opposite directions for expansion and contraction
\2 As a result, for some closed figure, it moves back to the original equilibrium state through other equilibrium states, called a cycle, such that the net work of the cycle is the difference between the work of the expansion and contraction temperature pathways
\2 It is noted that changing the direction of the cycle would change the sign of the line integral, due to depending on which path direction is traversed for expansion and contraction ($\ointPdV$ for a clockwise curve due to each pathway integral being negative)
\2 Due to the infinitesimal force differences and possible chemical and thermal movement which is ignored, the path does not have to reverse directly, but rather use a different path such that net work is done, as well as the rate of the multiple controlled changing conditions
\3 Thus, the path itself matters, such that for an isobaric process, in which pressure is constant, followed by an isochoric process in which volume is constant, it would be different in area from a direct path, such that it is path dependent
\3 Since the integral is not path-independent, due to infinite possible P(V) functions changing the outcome, $dW = PdV$ is not valid, due to infinite possible values, but rather only valid for a  specific path of P(V), denoted as an inexact differential, $\delta W = PdV$
\4 This means that $\delta W$ is not a small change in the function of work, but rather simply a small amount of work performed by some small change in dV, such that W is defined only in terms of the integral
\4 Work as a result, is not some function of thermodynamic coordinates, or a state function of the system, but rather a path function depending on the specific change in the system
\4 For an isothermal change as a result, the path is the isotherm on the PV plane, such that the work is positional rather than path based, where $\frac {\partial P}{\partial V}$ determines the relative work needed for a change in volume or pressure
\4 For a derivative of an inexact differential, it is noted that the path must still be determined for the differential, even with a designated change, assuming there are multiple paths for the change itself
\2 This is all explained by the fact that work for a non-conservative force is a multi-variable path function, rather than a single variable state function, only state if conditions are imposed
\3 This can also be written as the curl of the force being 0 for conservative forces, as a conservative vector field, which are path-independent on multiple variables as well
\1 For a length of wire system, $\delta W = \tau dL$, while for a surface film (two layers containing some liquid between), $\delta W = \gamma dA$, proved similarly for each system
\2 The equation is derived for a film, but is equally applicable for a single surface, though the area of the surface is doubled for a film, due to repeating
\end{outline*}
\section{Chapter 4 - Heat and the First Law of Thermodynamics}
\subsection{Internal Energy}
\begin{outline*}
\1 Closed systems can have a change of state both by external work done on the system, and by heat transfer from a hotter substance within the system, rather than mechanically
\2 Heat is thus defined as something transferred between a system and surroundings by temperature difference only, such that adiabatic walls are also called heat insulators and diathermic walls are heat conductor
\2 Heat transfer and work can often be viewed as snonymously occurring, such that it depends on what is defined as the system and surroundings to define which is taking effect
\1 Adiabatic work is work done within an adiabatic boundary performed by a coupled system acting on the adiabatically closed system, such that there is no transfer of heat into the system, but rather purely a mechanical energy transfer
\2 This can still result in a change in temperature, based on which parameters are allowed to vary, such as increasing the temperature of water through a resistor
\2 As a result, the first law of thermodynamics states that for any adiabatic change of state, the work done is equal for all adiabatic pathways connecting the states
\3 Similar to a conservative force system, a thermodynamic system can thus be described in terms of a potential energy function, such that $W_{adiabatic, i \to f} = U_f - U_i$, called the internal energy function, where W is the work done on the system
\3 It is noted that not all states can be reached by purely adiabatic work due to entropy
\3 This as a result defines the law of the conservation of energy, and states that for any system, there exists an energy function describing the difference between two states as the energy change
\4 It was noted that it was thought beta decay and the microscopic scale did not follow conservation of energy, due to less calculated after the decay, found to be due to the antineutrino for $\beta^-$, neutrino for $\beta^+$
\4 On an astronomical scale, it was found that dark energy and dark matter was necessary for conservation of energy to exist, but within general relativity in a nonconstant gravitational field, it was found to not apply
\2 Since internal energy is a state function, dU can be expressed as an exact differential in terms of the coordinates necessary to describe a state
\end{outline*}
\subsection{First Law of Thermodynamics}
\begin{outline*}
\1 For nonadiabatic work, to allow conservation of energy to be consistent since $\Delta U \neq W_{adi}$, energy must be transferred by another means, heat transfer
\2 Thus, heat is defined thermodynamically as energy transferred by non-mechanical means into/from a closed system is called heat, such that $\Delta U = Q + W$, where W is adiabatic work and Q is diathermal heat transfer, positive for both if entering the system, as the full first law
\3 This is true for all thermodynamic system changes, rather than just quasi-static process systems, due to not requiring an equation of state
\3 For an infinitesimal process, involving only infinitesimal changes in thermodynamic coordinates, it is written as $dU = \delta Q + \delta W$, or for quasi-static hydrostatic system, $dU = \delta Q + PdV$
\3 For a composite system with multiple components, $\delta W$ can be written as the sum of work done on each system within the overall system
\2 Thus, it can be seen from this the experimental idea that heat is energy transferred by a temperature gradient
\2 Temperature gradient is not the only relevant factor for transferred energy though, such that for an isochoric (constant volume) system due to a fixed diathermic boundary, it is internal energy transferred
\3 On the other hand, isobaric heat (constant pressure) due to a movable boundary, it is enthalpy, rather than internal energy transferred as heat
\3 Thus, heat is either internal energy or enthalpy based on the system conditions being transferred, though it is only applicable as a path quantity, rather than a state, $\delta Q$, similar to work
\2 Further, by this equation, it is evident that for an adiabatic system, the loss of heat from one section of the system is equal to the gain by another part of the system
\1 Heat capacity, more accurately called internal energy capacity, is the ability of a system to store energy, measured by the change in temperature at a point other than state changes
\2 It is measured with heat due to being easier to detect that work, such that $C = \frac{\delta Q}{dT}$
\2 The specific quantity of the heat capacity, or the unit mass quantity, called specific heat capacity, is a property of the material itself, rather than the object
\3 The molar heat capacity, c = $\frac{C}{n}$, where n is the number of moles, such that it is equal to the mass divided by the molecular mass multiplied by the molecules per mole (Avogadro's number, 6.022 * $10^23$)
\2 Heat capacity depends on the process pathway on the P-V graph, such that heat capacity at constant pressure is denoted $C_P(P, T)$, while heat capacity at constant volume is denoted, $C_V(V, T)$
\3 For some small range of variation in T and either P or V, the heat capacity is virtually constant within the range
\2 Heat capacity is measured by a resistance wire around a cylindrical sample, such that if both are the system, the electrical energy is doing work on the system, whereas if just the sample is, the electrical energy does work on the wire, moving as heat into the system
\3 The work required to produce the heat added to the system was originally called the mechanical equivalent of heat, later found just to be an alternate concept of work
\3 This is called a heating coil, used to test the statistical mechanics results of a system
\3 For some current in the wire, $\delta Q = \epsilon I dt$, building the calorimeter, coil, and connected thermometer based on the desired temperature range and material
\4 For low temperature  solids, it is suspended by a thin insulator, using a resistance thermometer placed in a drilled hole in the sample, using thin wires to avoid error
\3 The temperature as a function of time graph is drawn, measuring temperature without activating the circuit as the foreperiod, closing the switch, then after $\Delta t$, opening it and graphing the afterperiod, though often measuring resistance instead of temperature by the thermometer
\4 The foreperiod and afterperiod curves are extended to give the value in the center of the activated region, measuring the difference between as $\Delta T$
\4 Thus, $c_P = \frac{\epsilon I \Delta t}{n \Delta T}$, such that pressure is the constant variable due to atmospheric pressure keeping it constant
\2 Originally, the calorie was the amount of heat to raise the temperature of 1g of water by $1^o C$, though later specified as $14.5^o C$ to $15.5^o C$, measured as 4.186 J
\1 The first law combined with the fact that for U(T, V), $dU = \frac{\partial U}{\partial T}|_VdT + \frac{\partial U}{\partial V}|_TdV$, such that $\frac{\delta Q}{dT} = \frac{\partial U}{\partial T}|_V + (\frac{\partial U}{\partial V}|_T + P)\frac{dV}{dT}$
\2 By extension, if V is constant, the second term cancels out ($C_V = \frac{\delta Q}{dT}|_V = \frac{\partial U}{\partial T}|_V$)
\3 The left side of the equation is able to be used to test a theoretical internal energy function experimentally
\3 It is noted that it takes high amounts of pressure to hold volume constant, such that this is not an easy method of measurement though
\2 On the other hand, if pressure is constant, it becomes $C_P = \frac{\partial U}{\partial T}|_V + (\frac{\partial U}{\partial V}|_T + P)\frac{\partial V}{\partial T}|_P$
\3 This is due to the fact that since $\frac{\delta Q}{dT}$ is path-dependent, due to T relying on two other parameters, some path as a function of P and V must be chosen
\4 The only sections of the right hand side that are modified as a result of the path are P(T, V) and $\frac{dV(T, P)}{dT}$, the former held constant in this form, the latter becoming a partial
\4 While the pathway by which P and T vary for within different parts of the equation rely on separate changes/parameters, it only determines the bounds of the end result, while the pathway chosen determines for both
\3 This is further modified to show that $C_P = C_V + (\frac{\partial U}{\partial V}|_T + P)V\beta$, providing another function of internal energy found theoretically, in terms of measurable state functions
\end{outline*}
\subsection{Heat Flow}
\begin{outline*}
\1 Temperature difference between a system and surroundings creates a lack of thermodynamic equilibrium, with a nonuniform temperature, though quasi-static process of heat flow behaves similarly to that of quasi-static work
\2 For some body of relatively large mass to the main system, it functions as a heat reservoir, such that virtually any amount of heat flow will not measurably change thermodynamic coordinates
\2 Quasi-static heat reservoir contact would produce an isothermal change of state, due to the infinitesimal temperature gradient, adding energy to the system to produce other coordinate changes
\3 On the other hand, a series of infinitesimally different heat reservoirs would produce an overall non-isothermal change of state, able to be performed for a high amount of time, using heat reservoirs such that their temperature remains constant
\3 As a result, $\delta Q = CdT$, where the path of the heat capacity determines the path of Q, allowing it to be integrated
\1 Two parts of an adiabatically closed substance maintained at different temperatures at the edges have a continuous distribution of temperature in between, due to transporting energy between by heat conduction
\2 Thus, it is found experimentally that $\frac{\delta Q}{dt} = -KA\frac{dT}{dx}$, where $\frac{dT}{dx}$ is the temperature gradient, showing the rate of heat flow between the edges
\3 K is the thermal conductivity, such that a large valued object is a thermal conductor, a small value is a thermal insulator
\2 Thermal conductivity varies as a function of temperature, such that for a large gradient, it varies throughout the object, though if small enough, it can be considered constant
\3 For some small temperature difference, $K = \frac{-L}{A\DeltaT}\frac{\delta Q}{dt}$, due to the temperature gradient and length being
\2 It has been found that thermal conductivity of metal is reduced by a large amount by any impurities, as well as changes in pressure of gases below a certain level and composition
\3 Liquids have lower thermal conductivity, increasing as the temperature is raised, while metals tend to remain fairly constant until some minimum temperature where it decreases
\3 Nonmetals are poor conductors, similar to liquids, decreasing as temperature is raised, rising drastically at low temperatures
\3 Gases are especially bad conductors, independent of pressure after some maximum pressure, generally below 1 atm, increasing as the temperature rises
\2 Thermal conductivity in series is equal to the reciprocal of the sum of reciprocals, while in parallel is the sum, such that it is used in double pane glass with air between to lower the conductivity, due to air having a lower value
\1 Convection currents are flows of liquid or gas that absorb heat and mix with another fluid of a lower temperature, releasing the heat, or vice verse
\2 Convection currents by an external force/system is a forced convection, while if it is due to a combined density difference causing the flow along with a temperature difference, it is a natural convection
\2 For fluid in motion against some surface, there is some thin film of stationary fluid between, thinner the stronger the motion is
\3 Heat is thus able to be transferred both by conduction through the film and convection
\2 The convection coefficient h, is defined such that $\frac{\delta Q}{dt} = hA(T_{wall} - T_{fluid})$, where h takes into account both the surface conduction and the convection
\3 The coefficient depends on if the wall is flat/curved, vertical/horizontal, gas/liquid, density, viscosity, specific heat, and thermal conductivity of the fluid, laminar (low speed)/turbulent flow (high speed), and if evaporation, condensation, or formation of scale happens
\1 Heat can be transmitted through empty space through radiation, most relevantly thermal (IR and visible), which is transmitted due to the high temperature of the origin solid/liquid (gases behave separately)
\2 At higher temperatures, greater frequencies of visible light are emitted, and more total energy is emitted
\2 The radiant exitance, R, of the body is the radiant power per infinitesimal area, depending on the nature of the surface of the body and the temperature just as absorptivity and emissivity does
\2 Isotropic radiation is thermal radiation incident on a body equally from all directions, where the fraction of incident radiant power absorbed is absorptivity
\3 Total emissivity, $\epsilon$, is the fraction of power incident on the body that is emitted as thermal radiation, such that for thermal equilibrium, it is equal to absorptivity, depending on temperature and the material itself
\3 Ideal substances are those which absorbing and emit all radiation falling on it when at thermal equilibrium, called a blackbody, independent of the substance made from, approximated by a cavity with a small entrance hole and uniform high-temperature opaque walls
\4 This is due to the radiation reflecting around the cavity until fully absorbed, such that it is almost entirely absorbed and is isotropic
\2 Irradiance is the radiant power incident per infinitesimal area on the surface of the cavity, corresponding to radiant exitance
\3 Thus, for a blackbody with irradiance of H in thermal equilibrium with the cavity is placed within, such that radiant power absorbed per unit area of the blackbody is H$\epsilon$ = H
\3 Thus, radiant exitance of the blackbody, R, is equal to H, since the radiant power absorbed is equal to that emitted per unit area for the body, projected back to and from the cavity
\2 It has been found that the amount of radiation within the cavity is a function of the temperature produced by the thermal equilibrium and independent of the materials, detected by allowing a small amount of radiation to escape for measurement
\2 Radiant exitance of a non-blackbody depends on the surface material itself, as well as temperature, such that $R = \epsilon H = \epsilon R_{bb}$, called Kirchhoff's Law, used to measure thermal radiation emissivity of materials at some temperature
\2 There will only be a difference in the heat absorbed and emitted if there is a temperature difference between a body and its surroundings as a result
\3 Thus, $\frac{\delta Q}{dt} = A(\alpha(T) H - R) = A(\epsilon(T) R_bb(T_W) - \epsilon(T) R_bb(T) = A\epsilon(T)(R_bb(T_w) - R_bb(T)$ for some body in a cavity, such that it is isotropic, where the cavity is large enough for the change in temperature to be negligible, T is the temperature of the body, $T_w$ of the cavity, and A is surface area
\2 It was derived later thermodynamically that $R_bb(T) = \sigma T^4$, where $\sigma$ is the Stefan-Boltzmann constant, $5.67051 * 10^-8 \frac{W}{m^2 * K^4}$
\3 It follows that $\frac{\delta Q}{dt} = A\epsilon\sigma(T_W^4 - T^4)$
\3 The Stefan-Boltzmann constant can be determined by the nonequilibrium method, placing a shielded silver disk in a blackened copper hemisphere until there is steam, measuring the temperature $T_W$, uncovering the disk and measuring the temperature over time
\4 $\frac{dT}{dt}$ is then found, such that since pressure is constant, $\delta Q = C_P dT$ for the silver disk, where as a result, $\sigma = \frac{C_P}{A(T_W^4 - T^4)}\frac{dT}{dt}$
\3 It can also be found by the equilibrium method, placing a hollow blackened sphere suspended in a vessel of temperature $T_W$, adding electricity to the sphere at a constant power
\4 This is done until it reaches equilibrium where radiation emission is equal to the power supplied, such that assuming it acts as a blackbody, $\epsilon I = A\sigma(T^4 - T_W^4)$
\end{outline*}
\section{Chapter 5 - Ideal Gases}
\subsection{Ideal Gas Laws}
\begin{outline*}
\1 It has been found experimentally that $\frac{PV}{n} = Pv = A(1 + BP + CP^2 \dots)$, such that it is purely a function of pressure, called the virial expansion, where A, B, C are virial coefficients
\2 Until 40 atm, it is generally linear, such that only pressure is necessary, requiring more as pressure rises, though as pressure approaches 0, it is equal to A, only dependent on the temperature, rather than the gas itself 
\3 Virial coefficients are generally small, except at extremely low temperatures
\2 Since the ideal gas temperature assumes constant volume and the same amount of moles of gas for each measurement, $T = 273.16 K lim \frac{Pv}{Pv_{TP}}$
\3 As a result, R, the molar gas constant, is defined as $\frac{lim(Pv)_TP}{273.16 K}$, equal to 2.27102 kJ/mol*K, difficult to measure due to absorption of gas on walls
\3 Thus, the ideal gas law is derived such that lim(PV) = nRT and A = RT, such that $\frac{Pv}{RT} = 1 + BP + CP^2 + \dots$
\1 Adiabatic free/Joule expansion is the expansion in volume of a gas as there becomes a larger space to move into, such that no Q or W is done, and internal energy is constant
\2 Free expansion is noted not to be quasi-static, due to the fast modification from one state to another forcing it through nonequilibrium states
\2 The Joule coefficient, $\frac{\partial T}{\partial V}|_U$, testing with two diathermic vessels in a adiabatic water bath, measuring the change in gas temperature, found that the heat capacity of water was too high, and thus the coefficient was too difficult to measure directly
\3 It was tested with water due to too much difficulty in making the temperature of gas
\2 Since for U(T, V), $dU = \frac{\partial U}{\partial T}|_VdT + \frac{\partial U}{\partial V}|_TdV$, such that if no temperature change or internal energy change takes place, $dT = dU = 0$, then $\frac{\partial U}{\partial V}|_T = 0$
\3 Similarly, for U(T, P), $\frac{\partial U}{\partial P}|_T = 0$ in this case as well
\4 Thus, as the gas approaches an ideal gas, it is equal to 0 as pressure approaches 0, such that there is no change in temperature in that case
\3 As a result, if free expansion causes no change in temperature, internal energy of the gas is a function of temperature only for free expansion, used to measure this fact
\3 It is also done by measuring the $\frac{\partial u}{\partial P}|_T$, where u is the molar internal energy, measured by a container with n moles of gas at some pressure, connected to the atmosphere by a long coil, surrounding the container with liquid at atmospheric temperature
\4 The valve is opened to allow small amounts of gas to flow out, using a heating coil to keep all temperatures constant with the atmosphere as the gas expands
\4 As Q heat is absorbed by the gas from the coil, the work done by the gas is $W = -P_0(nv_0 - V)$, where $P_0$ is atmospheric pressure, V is container volume, and $v_0$ is molar volume at atmospheric temperature and pressure
\4 It is noted that $nv_0$ must be larger than the volume of the container, such that it expands when provided to the air
\4 $\Delta u(P, T)|_T = \frac{Q + W}{n}$, after accounting for contraction of container walls, putting it in terms of change in pressure instead of volume, plotting for different pressures
\3 On the other hand, it was noted that the experiment did not reach the ideal gas value of 0, and had the same issue of larger heat capacity of water, requiring precision in water temperature
\1 Ideal gases are thus those which have low enough pressure for the ideal gas law, and as well that internal energy is a function of temperature, rather than temperature and pressure
\2 Gases can be assumed to be ideal with increasing error as the pressure increases, though with only a few percent error below 2 atm
\2 For some quasi-static process, the ideal gas law can be written as PdV + VdP =  nRdT by the product rule, and as a result of the second part, it can be stated that $\frac{dU}{dT}|_V = C_V$, such that $\delta Q = C_VdT + PdV$
\3 By combination of these, it is found that $\frac{\delta Q}{dT} = C_V + nR - V\frac{dP}{dT}$, such that if pressure is set constant, $C_P = C_V + nR$ for an ideal gas
\3 Thus, $C_V, C_P$ are functions of temperature only for ideal gases, and $C_P \geq C_V$, due to the gas expanding work to expand against the equal, external pressure when pressure is constant, whereas for constant volume, there is no work done (PdV = W)
\2 It is also derived following that $\delta Q = C_PdT - VdP$ for an ideal gas, giving another form of the 1st law for an ideal gas
\end{outline*}
\subsection{Heat Capacity Measurement}
\begin{outline*}
\1 Heat capacities are measured by a gas in a thin-walled steel container, surrounded by a wire to add heat, measuring the rise in temperature at constant volume to get $C_V$, while $C_P$ is through a calorimeter allowing volume to vary rather than pressure
\2 The inlet/initial and outlet/final temperatures are measured, found that for all ideal gases, $C_V(T) < C_P(T)$, $C_P - C_V = R$, and $\frac{C_P}{C_V} = \gamma$
\2 Monatomic gases, such as noble gases and most metallic vapors, $C_V = c \approx \frac{3R}{2}$, and $C_P = c' \approx \frac{5R}{2}$
\2 Permanent diatomic gases, such as $H_2, O_2, N_2, NO, CO, F_2, Cl_2$, have constant values of $C_V \approx \frac{5}{2}R, C_P \appx  \frac{7}{2}R$
\3 On the other hand though, at low temperatures, $H_2$ uniquely acts as a monatomic gas
\3 For all temperatures, it may be written as $\frac{C_P}{R} = \frac{7}{2} + f(T)$, where $f(T) = (\frac{b}{T})^2 \frac{e^{\frac{b}{T}}}{(e^{\frac{b}{T}} - 1)^2}$ generally for some constant b, generally too difficult to calculate, such that it is measured experimentally
\2 Polyatomic gases and chemically active gases are those with values for $C_P, C_V$ varying with temperature
\1 For some quasi-static ideal gas adiabatic process, $\delta Q = 0$, such that since $\delta Q = C_VdT + PdV = C_PdT - VdP$, then $\frac{dP}{P} = -\gamma \frac{dV}{V}$, such that for some small temperature change, $\gamma$ is constant, such that $PV^\gamma = c$
\2 As a result, for some adiabatic process on a P-V graph, $\frac{\partial P}{\partial V}|_Q = -\gamma \frac{P}{V}$, such that since for a quasi-static isothermal process, $\frac{\partial P}{\partial V}|_T = \frac{P}{V}$ by the ideal gas law, the curve for the former is steeper than the latter
\1 Ruchhardt measured $\gamma$ by placing gas in a jar of some volume, with a glass tube attached with a metal ball and piston within, such that $P_{eq} = P_{atm} + \frac{mg}{A}$, by $F = PA$  where A is the area of the tube, assuming no friction
\2 With displacement, it will oscillate, such that $dV = Ady$ and $dP = \frac{dF}{A}$, acting as a restoring force, causing oscillation, moving quickly enough to prevent heat transfer, such that it is adiabatic and minor enough to be quasi-static
\2 As a result, by the product rule and the ideal gas quasi-static adiabatic formula, $dF = -\frac{\gamma PA^2}{V}dy$, such that it is simple harmonic motion, where $\tau = 2\pi \sqrt{\frac{m}{-\frac{F}{y}}} = 2\pi \sqrt{\frac{mV}{\gamma PA^2}}$, measuring the period to get $\gamma$
\2 A majority of the error originates from the assumption of the lack of friction within the glass tube
\3 Error from assumptions was later corrected, using a steel piston in the middle of the gas container, dividing it in two and displacing it, using an electromagnet to hold the piston in place against weight, to prevent friction, using a current to move it slightly
\3 It also used the real equation of state, measuring the frequency by AC and amplitude by microscope, with a non-ideal non-adiabatic equation to create more precision
\end{outline*}
\subsection{Kinetic Theory of Ideal Gases}
\begin{outline*}
\1 The theory was determined by Waterston and Kronig in 1859, finding that temperature is a function of particle motion for a monatomic gas, assuming there is some high number of identical, chemically inert particles
\2 It is stated that the number of particles per mole is $6.0221 * 10^23$, Avogadro's number, where the atomic mass is used to calculate the total mass, with a mole as $22.4 * 10^3 mL$ for an ideal gas
\2 Ideal gases are assumed to be in perpetual random motion with no preferred direction of motion, with a inter-atomic distance far larger than the atomic radius, with no forces of attraction between collisions, with velocities ranging from 0 to the speed of light
\3 This can be integrated to infinity without error, due to such low density at higher velocities, and is assumed that the distribution at any particular velocity band is approximately constant
\2 The atoms are uniformly distributed ($\frac{N}{V}$ is constant), such that $dN = \frac{N}{V}dV$, when fulfilling thermodynamic macroscopic qualifications
\2 Atomic collisions are assumed to be elastic against a smooth wall of the container, where only the perpendicular velocity with respect to the wall is changed, negated as a result without different speed
\1 For some velocity vector from the origin, $\vec{v}$, to some elementary area, dA, on the surface of the sphere, such that for the sphere around the origin, $dA = (rd\theta)(rsin\theta d\phi)$
\2 The solid angle of a circle, as a 3D extension of angle, $d\Omega = \frac{dA}{r^2} = sin \theta d\theta d\phi$, the maximum being that of the entire sphere, $4\pi$ steradians (sr), such that it is the surface area for a circle of radius 1
\2 
\end{outline*}
\section{Notes}
Boundary layer is what changes temperature, rather than the wall itself, such that it is what modifies? Convection and conduction simulaneously? Ice water == resistor in series - Homework ice problem
Oscillating electric charge emits an electromagnetic waves/radiation
Opaque bodies have 0 transmission fraction, transparent have 1, white bodies have reflective fraction of 1
Planck radiation law solved the lack of classical theory infinite 5000K radiation, leading to photons and quantum mechanics
\end{document}
