\documentclass[11 pt, twoside]{article}
\usepackage{textcomp}
\usepackage[margin=1in]{geometry}
\usepackage[utf8]{inputenc}
\usepackage{color}
\usepackage{indentfirst} %Comment out for no first paragraph indent
\usepackage[parfill]{parskip}
\usepackage{setspace}
\usepackage{tikz}
\usepackage{amsmath}
\usepackage{amsfonts}
\usepackage{amssymb}
\usepackage{enumitem}
\usepackage{outlines}

\usepackage{fancyhdr}
\pagestyle{fancy}
\cfoot{\hyperlink{content}{\thepage}}
\lhead{}
\chead{}
\rfoot{}
\lfoot{}
\rhead{}
\renewcommand{\headrulewidth}{0pt}
\renewcommand{\footrulewidth}{0pt}


\usepackage{hyperref}
\hypersetup {
	colorlinks,
	citecolor=black,
	filecolor=black,
	linkcolor=black,
	urlcolor=black
}

\newcommand{\sepitem}{0pt} %Added room between items on the list, not including a list and its sublist
\newcommand{\seppar}{1pt} %Between items and lists overall

\setenumerate[1]{itemsep=\sepitem, parsep=\seppar}
\setenumerate[2]{itemsep=\sepitem, parsep=\seppar}
\setenumerate[3]{itemsep=\sepitem, parsep=\seppar}
\setenumerate[4]{itemsep=\sepitem, parsep=\seppar}

\newenvironment{outline*}
{
	\begin{outline}[enumerate]
	}
	{\end{outline}
}

\newcommand{\foot}[1]{\hyperlink{#1}{$_#1$}}

\begin{document}

\title{}
\author{Avery Karlin}
\date{}
\newcommand{\textbook}{}
\newcommand{\teacher}{}

\maketitle
\newpage
\hypertarget{content}{\tableofcontents}
\vspace{11pt}
\noindent
\underline{Primary Textbook}: \textbook\\
\underline{Teacher}: \teacher
\newpage

\section{Chapter 1 - Introduction}
\begin{outline*}
\1 Computer systems are viewed from a bottom up approach of how transistors run basic programs and a top down approach of how complicated programs are converted into simple logic commands
\2 This is used to understand commands a computer does well and does badly to program more efficiently, and to understand how changes in technology change the speed of computing
\2 C is used due to being high level enough to write large programs, but low level enough to allow direct modification of bits
\1 The concept of abstraction is central to computer science, focusing on a higher level, rather than the component ideas to save time and mental effort, assuming the details work
\2 On the other hand, it must go in turn with deconstruction, breaking down an abstract idea into more concert sub-ideas, the opposite of abstraction, in case there is a problem
\1 The concept of viewing hardware and software as joint components of a single system, which must both be taken into account to design either, is another central idea, to design the most effective components
\1 Processors/CPUs are the primary unit of a computer, used to direct the processing of information and perform the calculations required to process it, though there are other components to make use easier
\2 Originally, they were made of large boards covered in integrated circuit packages, but now are just a single silicon microprocessor chip with millions of transistors
\1 The only limitations of computers are time and amount of memory, but otherwise all computers can do the same tasks, though not at the same pace
\2 This is due to computers being universal computing devices, as digital machines which could be increased in precision unlike analog/physical machines, but which were not made for individual tasks
\2 Turing in 1937 proposed the Turing machine, which would be able to carry out all computations of some type, and later began to define what computation is, abstracting tasks by a black box model, showing the task, input, and output, with no specification about how it is performed
\3 Turing's thesis states that a Turing machine can do all computations, such that improvements to it do not change the amount of computations, making a universal Turing machine able to simulate all different Turing machines
\2 Computers/universal Turing machines are able to do any computations, due to being programmable
\1 Computer problems must be converted into voltages to influence the flow of electrons which the computer is made of, made of a series of methods to allow carrying out of complex tasks
\2 The levels of transformation are the levels of choice to convert the problem into an electron flow for the computer, starting with the statement in a natural/human language, which has too much ambiguity to give directly to the computer
\2 The first transformation i
\end{outline*}
\end{document}
